We can gain further insight to clusterization in our generic model,
from Carath\'eodory's Theorem and the concentration on the first $k$
eigenvectors of $\fwd \prcov \fwd^*$. We present a short discussion
adapting arguments presented in \cite[Chapter
  3]{silveyOptimalDesign1980} and \cite[Section
  5.2.3]{pronzatoPazman2013}.

Consider $\opt$ a D-optimal design under our generic model.  As
instructed by Theorem \ref{thm:char}, we ignore all but the first $k$
eigenvectors of $\fwd \prcov \fwd$. Thus, we replace $\hilo$ with
$\hilo^{(k)}$ --- the $k$-dimensional subspace spanned by the first
$k$ eigenvectors of $\fwd^*\prcov\fwd$. Let
\begin{equation*}
  \mathcal{M} := \conv \{\meas \meas^* : \meas\in \hilo^{(k)}, \|\meas\|=1\},
  %\hilo^{(k)}\},
\end{equation*}
where $\conv$ denotes the convex hull of a set. The set $\mathcal{M}$
contains only positive-definite operators on a $k$-dimensional vector
space. Hence $\mathcal{M}$ lives in a $k(k+1)/2$-dimensional vector
space. Since $\opt^*\opt = \sum_{i=1}^m \meas_i\meas_i^*$ for $\meas_i
\in \hilo^{(k)}$, it is easy to verify that $\frac1m \opt^*\opt \in
\mathcal{M}$.  Recall Carath\'eodory's Theorem:
\begin{theorem*}[Carath\'eodory]
  Let $X \subseteq \mathbb{R}^n, X \neq \phi$ and denote $\conv (X)$
  the convex hull of $X$. For every $x \in \conv (X)$, $x$ is a convex
  combination of at most $n+1$ vectors in $X$.
\end{theorem*}
Carath\'eodory's Theorem implies that there exist $\meas_i$ and
$\alpha_i$ such that
\begin{equation*}
  \opt^*\opt = \sum_{i=1}^I \alpha_i \meas_i\meas_i^*,
\end{equation*}
where $\|\meas\|=1, \sum\alpha_i = m, \alpha_i \geq 0$ and $I =
\frac{k(k+1)}{2} + 1$. We can thus write $\opt$ as:
%(note that we do not require $\meas_i$ to be orthogonal to each
%other):

\[
\opt =
\left[
  \begin{array}{ccc}
    \horzbar & \sqrt{\alpha_i} \meas^*_1 & \horzbar \\
    \horzbar & \sqrt{\alpha_2} \meas_2^* & \horzbar \\
             & \vdots    &          \\
    \horzbar & \sqrt{\alpha_I} \meas_I^* & \horzbar \\
  \end{array}
\right].
\]

Unfortunately, $\opt$ is not a valid design, since its rows do not
have unit norm. Still, the above representation of $\opt$ is useful:
If $m > \frac{k(k+1)}{2} + 1$, then $\alpha_i > 1$ for some $1\leq i
\leq I$.  Thus, we can view $\opt$ as a clustered design, since it
places weight $>1$ on a single measurement vector.
