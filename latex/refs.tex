\documentclass{article}
\def\papertitle{Measurement Clusterization in Bayesian D-optimal Designs in Infinite Dimensions}
\def\authors{Yair Daon}
\def\journal{Bayesian Analysis}
\def\doi{12345}


% Define title defaults if not defined by user
\providecommand{\lettertitle}{Author Response to Reviews of}
\providecommand{\papertitle}{Title}
\providecommand{\authors}{Authors}
\providecommand{\journal}{Journal}
\providecommand{\doi}{--}

\input{preamble}
\usepackage{amsmath}
\usepackage{amssymb}
\usepackage{amsfonts}
\usepackage{amsthm}
%% \usepackage{amsaddr}


\usepackage{thm-restate}
\usepackage{tikz,pgfplotstable}
\usetikzlibrary{patterns}
\pgfplotsset{compat=1.9} % set to 1.8 to get old behaviour
%\usepackage{hyperref}
\usepackage{xcolor}

\newcommand{\R}{\ensuremath{\mathbb{R}}}
\newcommand{\eps}{\varepsilon}
\newcommand{\M}{\mathcal{M}}

\newcommand{\der}{\text{\textup{d}}}
\newcommand{\diag}{\textup{diag}}

\providecommand{\x}{}
\renewcommand{\x}{\mathbf{x}}
\newcommand{\y}{\mathbf{y}}

\newcommand{\hil}{\mathcal{H}}
\newcommand{\hilp}{\mathcal{H}_p}
\newcommand{\hilo}{\mathcal{H}_o}
\newcommand{\obs}{\mathcal{O}}
\newcommand{\pobs}{\mathcal{P}}
\newcommand{\fwd}{\mathcal{F}}

\newcommand{\obsm}{\widehat{\obs}}
\newcommand{\Sigmam}{\widehat{\Sigma}}
\newcommand{\postcovm}{\widehat{\Gamma_{\textup{post}}}}
\newcommand{\uu}{\mathbf{u}}
\newcommand{\tar}{\Psi}
\DeclareMathOperator*{\argmin}{arg\,min}
\DeclareMathOperator*{\argmax}{arg\,max}
\DeclareMathOperator*{\conv}{conv}


% Definitions for second chapter
\newcommand{\data}{\mathbf{d}}
\newcommand{\param}{\mathbf{m}}
\newcommand{\normal}{\mathcal{N}}
\newcommand{\pr}{\mu_{\textup{pr}}} %Prior measure
%\newcommand{\post}{\mu_{\textup{post}}^{\data, \obs}} % Posterior measure
\newcommand{\post}{\mu_{\textup{post}}} % Posterior measure for slides
\newcommand{\prmean}{\param_{\textup{pr}}} % Prior mean
\newcommand{\postmean}{\param_{\textup{post}}} % Posterior mean
\newcommand{\postcov}{\Gamma_{\textup{post}}} % Posterior covariance
\newcommand{\prcov}{\Gamma_{\textup{pr}}} % Prior covariance
\newcommand{\modcov}{\Gamma_{\textup{model}}} % Model covariance
\newcommand{\tmp}{\mathcal{G}}
\newcommand{\meas}{\mathbf{o}}
\newcommand{\ev}{\mathbf{e}} % eigenvector 
\newcommand{\func}{\mathbf{a}}
\newcommand{\tr}[1]{\textup{tr}\left \{#1 \right \} }
\newcommand{\ttr}[1]{\textup{tr}\ #1}
\newcommand{\rank}{\textup{rank}\ }
\newcommand{\des}{\eta} % vector of design parameters
\newcommand{\sigsqr}{\sigma^2}

%% \newcommand{\opt}{\mathfrak{D}}
\newcommand{\opt}{\mathcal{D}}
\newcommand{\postopt}{\mu_{\textup{post}}^{\data, \opt}} % Posterior measure

%% %\usepackage{comment}% http://ctan.org/pkg/comment
%% %% %% %\excludecomment{proof}
%% %% \excludecomment{figure}
%% %% \let\endfigure\relax


%% %\overfullrule=0pt

\theoremstyle{plain}
\newtheorem{theorem}{Theorem}
\newtheorem{corollary}[theorem]{Corollary}
\newtheorem{lemma}[theorem]{Lemma}
\newtheorem{proposition}[theorem]{Proposition}
\theoremstyle{definition}
\newtheorem{definition}[theorem]{Definition}
\newtheorem{example}[theorem]{Example}
\newtheorem{conjecture}[theorem]{Conjecture}
\theoremstyle{remark}
\newtheorem{remark}[theorem]{Remark}

\usepackage{xr}
\externaldocument{ba}

\begin{document}

% Make title
{\Large\bf \lettertitle}\\[1em]
{\huge \papertitle}\\[1em]
{\authors}\\
{\it \journal, }\texttt{doi:\doi}\\
\hrule

% Legend
\hfill {\bfseries RC:} \textbf{\textit{Reviewer Comment}},\(\quad\) AR: \emph{Author Response}, \(\quad\square\) Manuscript text

I appreciate the many thoughtful insightful comments by all of
you. These comments have helped me considerably improve the exposition
of my paper and added new insights as well. I sincerely appreciate the
option to revise and resubmit. Please see below my answers to the
comments made by editor in chief (Section \ref{eic}), associate editor
(Section \ref{ae}) and three referees (Sections \ref{ref1}, \ref{ref2}
and \ref{ref3}).



\section{Editor in Chief}\label{eic}
\RC Personally I like the paper, but it represents a somehow unusual
submission for Bayesian Analysis, and I am worried about it being of
sufficient general interest for the broad readership of the
journal. Still, in agreement with the AE, we decided to give the
author the possibility to revise the paper. However, I must stress
that there is no guarantee of a successful outcome, which will solely
depend on the quality of the revision. In particular:

\RC the Introduction is to be made accessible to the whole BA
    readership;
    
\AR  I have considerably extended the introduction.
   
    
\RC the motivation and the practical statistical implications are to
be developed and discussed;

\AR I added a section on the implications:

%\begin{quote}
  %Our answer to Question \ref{q:generic} implies that encountering
clusterization should be expected in many different problems across
many different scientific fields. Researchers that encounter
clusterization should not be surprised or wary. In Our answer to
Question \ref{q:why}, we explain what our view of the cause of
clusterization is. It appears, the cause is generic: a D-optimal
design reduces uncertainty for a select set of prior covariance
eigenvectors --- those with the most prior uncertainty, i.e.~those the
practitioner cares about the most! We believe practitioners should not
try to avoid measurement clusterization. Rather, practitioners should
take repeated measurements (e.g.~in MRI and borehole tomography),
increase apparatus sensitivity (e.g.~in EIT), or take consecutive
measurements (e.g.~in the 1D heat equation). Overall, we show that
measurement clusterization is a natural and (almost) inevitable part
of Bayesian D-optimal designs (but see disclaimer below).

One interesting implication of the analysis presented here is that
clusterization can serve as an evidence to the number of relevant
eigenvectors. Since leading eigenvectors typically correspond to slow
variations in space and/or time, clusterization could be used to
estimate the number of relevant degrees of freedom, and even to reduce
the complexity of a computational model, e.g.~by dropping
discretization points.

It is important to note that we do not view clustered designs as
undesirable, nor do we believe D-optimal designs should be avoided at
all. Nothing in our analysis indicates that we should expect any
pathological behavior when utilizing D-optimal designs. On the
contrary: we show that D-optimal designs (clustered or not) reduce
uncertainty of prior covariance eigenvectors where it is
highest. Thus, in a sense we hope to make precise in a future study,
we expect convergence of for D-optimal designs to be fastest.

For example, \cite{tekentrup2020} provides convergence analysis for
various designs with differing space filling properties. She showed
that convergence rate depends on how these designs fill space. We
expect better convergence rates for D-optimal designs; Even though
D-optimal desgins demonstrate clusterization, they do so because they
explore space in a way tailored to the problem studied. Consequently,
D-optimal designs cluster because they first aim to measure what
matters most; see the discussion following Theorem \ref{thm:char} for
details.



We expect D-optimal designs to judicious exploration of
design space, as implemented by a D-optimal design to give better
convergence compared to randomly sampling the design
space \cite{knapik2011}.




Clusterization is a peculiar phenomenon and it is perfectly reasonable
for someone to argue against the D-optimality criterion based on the
fact that it results in clustered designs. We have seen, however that
there is a perfectly reasonable explanation for clusterization. We
have shown that clusterization is an inevitable consequence of having
a problem with some modes where uncertainty decays faster than others.

Lastly, we believe that when clusterization arises, it should serve as
a warning sign to practitioners. In the inverse problem of the 1D heat
equation, clusterization occurs primarily because Laplacian
eigenvectors \emph{do not} decay in $\Omega$. Consequently, measuring
$u(x_1, T)$ at some point $x_1 \in \Omega$ provides information about
$u(x_2,T)$ for distant points $x_2 \in \Omega$. Intuitively, this
should not occur: for small $T$, the heat distribution at $x_1$ should
give very little knowledge on the heat distribution at $x_2$. This
behavior stems from a well-known property of the heat equation: it
allows information to spread \emph{instantly} across the computational
domain \cite{renardy2006PDE}. In reality, heat (and information)
propagate at finite speeds. Of course, the known physical barrier for
information spread is the speed of light, but we expect heat to spread
considerably slower: heating an Olympic pool at one end should have no
immediate effect on the temperature at the other end.

Our choice of prior is also a potential major contributor to
clusterization. Our choice of Gaussian prior similarly implies
information is shared between distant locations in $\Omega$. Thus, we
suggest refraining from choosing Gaussian priors with inverse
Laplacian covariance operators. Rather, non-Gaussian priors could be
employed instead \cite{hosseini2017, hosseini2019}.

We believe that the emergence of clusterization in this context is
thus non-physical, arising from the way the inverse problems we
consider are phrased. Clusterization therefore indicates that the
underlying mathematical / Bayesian model is overly permissive and
fails to capture crucial physical constraints of the problem. We
suggest that when clusterization occurs, practitioners should consider
alternative models where information is localized in space and travels
at finite speed in the medium. Such models may not only provide more
physically accurate and meaningful results but may also mitigate the
issue of clusterization.

%\end{quote}
   
    
\RC the contribution has to be better contextualized with respect to
the literature in both inverse problems and D-optimal designs;

    
\RC numerical evidence and code, mentioned in the paper but actually
not provided, need to be made available;
    
\AR I added a section describing numerical experiments and their
results. I also added documentation to the accompanying
\href{https://github.com/yairdaon/OED}{repository} and made sure all
scripts there run.
    
\RC the mathematical derivations have to be watertight and some
clarifications are needed in this respect (see Report of Referee 3).
  
\AR I am not sure which specific comment of Reviewer 3 this refers to,
please see my answers to Reviewer 3 below.
   

\section{Associate Editor}\label{ae}
\RC The three referees found merits and interest in your work, but also
pointed out several major issues to be addressed. Of particular
concern to me, the paper's exposition jumps too quickly into the
formulation / maths without giving a proper intuition and
context. This makes the paper less accessible to readers who aren't
already quite familiar with inverse problems (which are introduced
with very little preamble, and it's not even immediately explicit what
are the parameters being learned) and optimal experimental design.

\AR I have added considerably to the introduction. See revised
manuscript and diff file.


\RC A referee makes valid points as to whether clusterization of
measurements is a problem in general, when in fact in can be an
appealing feature in some settings, and another referee asks whether
D-optimal designs should be abandoned altogether. The exposition
needs to discuss these high-level issues, and introduce the related
literature and results appropriately, to give the work proper
context.



\RC Altogether, I think that your work has the potential to make a
good contribution to Bayesian analysis, but as it currently stands
significant work is needed.



\section{Reviewer 1}
\RC This paper explains a common practical issue: the cause of
clusterization in Bayesian D-optimal designs in infinite dimensions
and why adding a correlated measurement process decluster the design
points. Its theoretical results are valuable to the field of inverse
problems/uncertainty quantification and the conclusions should be
useful to practitioners in this field.

\AR I appreciate the kind words.


\RC However, I feel changing the assumption from independent to correlated
measurements is quite artificial: people tweak a data generation
process assumption to solve an optimal design problem, and the author
just takes the assumption as it is.

\AR If I understand correctly --- I completely agree. I added text
emphasizing that I do not endorse using correlated errors:


\begin{quote}
  It is important to note that we do not view clustered designs as
  undesirable, nor do we believe it should be avoided at
  all. Clusterization is a peculiar phenomenon and it is perfectly
  reasonable for someone to argue against the D-optimality criterion
  based on the fact that it results in clustered designs. We have
  seen, however that there is a perfectly reasonable explanation for
  clusterization. We have shown that clusterization is an inevitable
  consequence of having a problem with some modes where uncertainty
  decays faster than others.
\end{quote}

\RC Since the author proves the clusterization issue under the independent
measurement assumption, I expect the author to go deeper and explain
why an ordinary data generation assumption + a common optimal design
leads to a “suboptimal” design?

\AR In my view, this is a result of the unlocalized structure of
Laplacian eigenvectors. I believe this is more of a problem with our
(linear) physical models and choice of (Gaussian) priors. See
discussion below reproduced from the revised manuscript (but note that
I do not endorse clustered designs)


\begin{quote}
  Lastly, we believe that when clusterization arises, it should serve
  as a warning sign to practitioners. In the inverse problem of the 1D
  heat equation, clusterization occurs primarily because Laplacian
  eigenvectors \emph{do not} decay in $\Omega$. Consequently,
  measuring $u(x_1, T)$ at some point $x_1 \in \Omega$ provides
  information about $u(x_2,T)$ for distant points $x_2 \in
  \Omega$. Intuitively, this should not occur: for small $T$, the heat
  distribution at $x_1$ should give very little knowledge on the heat
  distribution at $x_2$. This behavior stems from a well-known
  property of the heat equation: it allows information to spread
  \emph{instantly} across the computational domain
  \cite{renardy2006PDE}. In reality, heat (and information) propagate
  at finite speeds. Of course, the known physical barrier for
  information spread is the speed of light, but we expect heat to
  spread considerably slower: heating an Olympic pool at one end
  should have no immediate effect on the temperature at the other end.

  Our choice of prior is also a potential major contributor to
  clusterization. Our choice of Gaussian prior similarly implies
  information is shared between distant locations in $\Omega$. Thus, we
  suggest refraining from choosing Gaussian priors with inverse
  Laplacian covariance operators. Rather, non-Gaussian priors should be
  employed instead \cite{hosseini2017, hosseini2019}.
  
  We believe that the emergence of clusterization in this context is
  thus non-physical, arising from the way the inverse problems we
  consider are phrased. Clusterization therefore indicates that the
  underlying mathematical / Bayesian model is overly permissive and
  fails to capture crucial physical constraints of the problem. We
  suggest that when clusterization occurs, practitioners should consider
  alternative models where information is localized in space and travels
  at finite speed in the medium. Such models may not only provide more
  physically accurate and meaningful results but may also mitigate the
  issue of clusterization.
\end{quote}


\RC Not just by the mathematical proof, but from an information theory
point of view. For example, does the redundancy points mean additional
points does not reduce the variance of the estimation at all? If
that’s the case, how does correlated measurement assumption “help”
with it?

\AR I think this is the same misunderstanding I commented on above;
For any nonzero observation error $\sigma^2 > 0$, repeated
measurements will indeed result in an increase of the design
criterion.




\section{Reviewer 2}\label{ref2}
\subsection{Measurement Clusterization vs Repetition}
\RC I will start by saying measurement clusterisation is not
well-defined in this paper. On page 1, it is referred to as two sets
of measurements that are almost indistinguishable from one
another. However, the setup for Corollary 11 implicitly defines
measurement clusterisation as repeated measurements. The latter
definition matches the references of \cite{fedorov1996} and
\cite{nyberg2012}. So I will assume for the rest of this review that,
measurement clusterisation refers to repeated measurements.

\AR I appreciate the comment, I fixed the relevant line in the
manuscript.


\RC Nearly all textbooks on classical design of experiments
(e.g.~\cite[Section 1.2.4]{morris2011}) will feature a section with
the three words randomisation, blocking and repetition: the important
principles of design of experiments.  In these texts, repeated
measurements are considered to be a beneficial property of a
design. Interestingly, when optimal designs are found, they are often
found to have repeated measurements. The paper does not explain why
the author thinks repeated measurements (measurement clusterisation)
is an undesirable property.

\AR\label{rep} I do not view clusterization as an undesired property, I only
mentioned many authors consider it to be undesirable. I have added a
paragraph contrasting repetition and replication with clusterization
and why I believe they may be different:


\begin{quote}
  Clusterization should not be confused with replication. Replication
  requires that the experimentalist executes multiple trials under
  circumstances that are \emph{nominally identical} \cite[Section
    1.2.4]{morris2011}. Replication is commonly viewed as a beneficial
  and even necessary aspect of optimal experimental design
  \cite{fisher1949design, morris2011,
    schafer2001replication}. Similarly to a design implementing
  replication, a clustered design reduces the signal-to-noise ratio of
  the repeated measurements \cite{telford2007brief}. The difference is
  that a clustered design takes repeated measurements at the expense
  of other quantities not measured at all.

  For example, \cite{fisher1949design}, suggested repeating his famous
  milk and tea experiment in order "to be able to demonstrate the
  predominance of correct classifications in spite of occasional
  errors". Unfortunately, in the experiments we consider, replication
  is impossible. For example, in the MRI problem, we cannot generate
  an individual nominally identical to the one we wish to scan.

  To further illustrate the difference between replication and
  clusterization, consider an experiment measuring the effect of
  rainfall on grass growth \cite{fay2000rainfall}. The experiment
  involved four rainfall manipulation "treatments" (i.e.~simulating
  different timing and quantity of rainfall), each replicated three
  times over a different plot of land. Indeed, it seems reasonable for
  researchers to replicate the phenomenon they are trying to study. A
  clustered design in such an experiment would imply the researchers
  should take a repeated measurement \emph{on the same plot}, at the
  expense of measuring grass growth in other plots. To conclude our
  short discussion on replication vs.~clusterization: these are
  fundamentally different concepts and even though replication is
  quite intuitive, it is inapplicable to the inverse problems we
  consider in this manuscript.
\end{quote}



\RC If one looks at, for example, \cite{fedorov1996} and
\cite{nyberg2012}, it is an undesirable property because they are
interested in sensor location, and two or more sensors cannot be in
the same location. However, there are many experiments where repeated
measurements are possible, and are therefore optimal

\AR I believe that even in an application where repeated experiments
are allowed they may be unintuitive. For example, in a medical imaging
application --- repeatedly measuring a "slice" in an MRI scan might
seem suboptimal to some.


\RC (if the model is correct, which it won’t be).

\AR I think this is a great point. I believe clusterization may imply
a model is somehow inadequate:

\begin{quote}
  When clusterization arises, we believe it should serve as a warning
  signal to practitioners. In the inverse problem of the 1D heat
  equation, clusterization occurs primarily because Laplacian
  eigenvectors do not decay in $\Omega$. Consequently, measuring
  $u(x_1, T)$ at some point $x_1 \in \Omega$ provides information
  about $u(x_2,T)$ for distant points $x_2 \in \Omega$. Intuitively,
  this should not happen: for small $T$, the heat distribution at
  $x_1$ should give very little knowledge on the heat distribution at
  $x_2$. This behavior stems from a well-known property of the heat
  equation: it allows information to spread instantly across the
  computational domain \cite{renardy2006PDE}. In reality, heat (and
  information) propagate at finite speeds. Of course, the known
  physical barrier for information spread is the speed of light, but
  we expect heat and information to spread considerably slower. Thus,
  clusterization may suggest that our physical model itself is at
  fault.
  
  Our choice of prior is perhaps an even greater contributor to
  clusterization. Our choice of Gaussian prior similarly implies
  information is shared between distant locations in $\Omega$. Thus,
  we suggest refraining from choosing Gaussian priors with inverse
  Laplacian covariance operators. Rather, non-Gaussian priors should
  be employed instead \cite{hosseini2017, hosseini2019}. So,
  clusterization may suggest we give more attention to the choice of
  prior.
\end{quote}



\RC An example, is a chemical engineering experiment where one wishes
to learn the relationship between a series of variables and yield of a
chemical reaction. The experiment will involve specifying the
variables, and measuring the yield (the observation) after a specified
period of time. This process is then duplicated with a potentially
different set of variables.  It is perfectly possible to have repeated
variables.

\AR I agree that the problem with clustered designs depends on the
application. However, in the chemical engineering experiment
mentioned, it is possible that an optimal design will require two
simultaneous measurements of the same experiment. This is an example
of a clustered design that seems unintuitive.

 
\RC The author needs to make clear that they are considering
experiments where repeated measurements are impossible. Admittedly,
this consideration is implicit in the paper, for example, the examples
given on page 1. However, it needs to be made explicit in the Abstract
and the first paragraph of the paper.

\AR I do not refrain from considering experiments where repeated
measurements are possible. I am giving insight on why such designs
arise, since many authors view repeated measurements as non-intuitive.

  
\RC The paper explains why repeated measurements can be optimal
(answer to Question 3). The author should expand on this and draw the
link with how repeated measurements are, typically, regarded as
desirable in classical design of experiments for finite
dimensions.

\AR Thank you for this suggestion. I believe my previous answer above
also answers this comment. It is reproduced below:

\begin{quote}
  Clusterization should not be confused with replication. Replication
  requires that the experimentalist executes multiple trials under
  circumstances that are \emph{nominally identical} \cite[Section
    1.2.4]{morris2011}. Replication is commonly viewed as a beneficial
  and even necessary aspect of optimal experimental design
  \cite{fisher1949design, morris2011,
    schafer2001replication}. Similarly to a design implementing
  replication, a clustered design reduces the signal-to-noise ratio of
  the repeated measurements \cite{telford2007brief}. The difference is
  that a clustered design takes repeated measurements at the expense
  of other quantities not measured at all.

  For example, \cite{fisher1949design}, suggested repeating his famous
  milk and tea experiment in order "to be able to demonstrate the
  predominance of correct classifications in spite of occasional
  errors". Unfortunately, in the experiments we consider, replication
  is impossible. For example, in the MRI problem, we cannot generate
  an individual nominally identical to the one we wish to scan.

  To further illustrate the difference between replication and
  clusterization, consider an experiment measuring the effect of
  rainfall on grass growth \cite{fay2000rainfall}. The experiment
  involved four rainfall manipulation "treatments" (i.e.~simulating
  different timing and quantity of rainfall), each replicated three
  times over a different plot of land. Indeed, it seems reasonable for
  researchers to replicate the phenomenon they are trying to study. A
  clustered design in such an experiment would imply the researchers
  should take a repeated measurement \emph{on the same plot}, at the
  expense of measuring grass growth in other plots. To conclude our
  short discussion on replication vs.~clusterization: these are
  fundamentally different concepts and even though replication is
  quite intuitive, it is inapplicable to the inverse problems we
  consider in this manuscript.
\end{quote}


\RC See for example, the discussion of the relationship between
Carath\'eodory’s Theorem and the number of support points in \cite[page
  139]{pronzatoPazman2013}. The support points are the unique design
points, so if this is less than $m$, then there are repeated design
points.
  
\AR I appreciate this insightful comment and reference. The arguments
presented there do not directly apply to the setup I suggest in my
paper. Thus, I have made some modifications:

\begin{quote}
  We can gain further insight to clusterization in our generic model,
  from Carath\'eodory's Theorem and the concentration on the first $k$
  eigenvectors of $\fwd \prcov \fwd^*$. We present a short discussion
  adapting arguments presented in \cite[Chapter 3]{silvey1980} and
  \cite[Section 5.2.3]{pronzatoPazman2013}.

  Consider $\opt$ a D-optimal design under our generic model.  As
  instructed by Theorem \ref{thm:char}, we ignore all but the first
  $k$ eigenvectors of $\fwd \prcov \fwd$. Thus, we replace $\hilo$
  with $\hilo^{(k)}$ --- the $k$-dimensional subspace spanned by the
  first $k$ eigenvectors of $\fwd^*\prcov\fwd$. Let
  \begin{equation*}
    \mathcal{M} := \conv \{\meas \meas^* : \meas\in \hilo^{(k)}, \|\meas\|=1\},
    %\hilo^{(k)}\},
  \end{equation*}
  where $\conv$ denotes the convex hull of a set. The set
  $\mathcal{M}$ contains only positive-definite operators on a
  $k$-dimensional vector space. Hence $\mathcal{M}$ lives in a
  $k(k+1)/2$-dimensional vector space. Since $\opt^*\opt =
  \sum_{i=1}^m \meas_i\meas_i^*$ for $\meas_i \in \hilo^{(k)}$, it is
  easy to verify that $\frac1m \opt^*\opt \in \mathcal{M}$.  Recall
  Carath\'eodory's Theorem:
  \begin{theorem*}[Carath\'eodory]
    Let $X \subseteq \mathbb{R}^n, X \neq \phi$ and denote $\conv (X)$
    the convex hull of $X$. For every $x \in \conv (X)$, $x$ is a convex
    combination of at most $n+1$ vectors in $X$.
  \end{theorem*}
  Carath\'eodory's Theorem implies that there exist $\meas_i$ and
  $\alpha_i$ such that
  \begin{equation*}
    \opt^*\opt = \sum_{i=1}^I \alpha_i \meas_i\meas_i^*,
  \end{equation*}
  where $\|\meas_i\|=1, \sum\alpha_i = m, \alpha_i \geq 0$ and $I =
  \frac{k(k+1)}{2} + 1$. We can thus write $\opt$ as:
  %(note that we do not require $\meas_i$ to be orthogonal to each
  %other):
  
  \[
  \opt =
  \left[
    \begin{array}{ccc}
      \horzbar & \sqrt{\alpha_i} \meas^*_1 & \horzbar \\
      \horzbar & \sqrt{\alpha_2} \meas_2^* & \horzbar \\
      & \vdots    &          \\
      \horzbar & \sqrt{\alpha_I} \meas_I^* & \horzbar \\
    \end{array}
    \right].
  \]
  
  Unfortunately, $\opt$ is not a valid design, since its rows do not
  have unit norm. Still, the above representation of $\opt$ is useful:
  If $m > \frac{k(k+1)}{2} + 1$, then $\alpha_i > 1$ for some $1\leq i
  \leq I$.  Thus, we can view $\opt$ as a clustered design, since it
  places weight $>1$ on a single measurement vector.
\end{quote}

%% Firstly, the argument in \cite{pronzatoPazman2013} relies on the
%% assumption that $\theta$ --- the parameter we seek to infer --- is
%% $p$-dimensional. However, the setup I suggest is
%% infinite-dimensional. E.g.~in the generic model I propose, the
%% parameter is in $\hilo$, which is assumed an infinite-dimensional
%% Hilbert space. Similarly, the initial condition $u_0$ we try to infer
%% in the inverse problem of the heat equation is in some function space
%% over $\Omega=[0,1]$. Even if we discretize $\Omega$ and consider a
%% finite-dimensional problem, the number of discretization points would
%% be at least a hundred. So even the bound $m \leq p=100$ suggested in
%% \cite[Section 5.3.1]{pronzatoPazman2013} is not very meaningful when
%% we allow only $\approx 6$ measurements. Indeed, discretization points
%% are cheap and measurements are expensive so the number of
%% discretization points will always be considerably larger than the
%% number of allowed measurements. Thus, I do not see how the bound
%% suggested will be informative. Respectfully I choose not to explore
%% your suggestion in the manuscript.

%% We can bypass the fact that $p$ tends to be large by recalling that
%% Theorem \ref{thm:char} implies

\subsection{Context}
\RC The paper lacks discussion on context. I read the answers to the
three questions and thought “So what?”  What are the implications to
statistical practice, or experimental science, of these findings?

\AR I really appreciate this comment. I added a section highlighting
the implications of this study. Since it is too long to reproduce
here, see Section \ref{subsub:implications} in the revised manuscript.


\RC The presentation of the paper does not help. The results are
presented with no explanation or intuition. It might be better to
relegate much of the mathematical detail to Supplementary Material and
use the main manuscript for exposition.

\AR Thanks for this comment. I considerably expanded the
  introduction and delegated many proofs to the Supplementary.


\subsection{Avoiding measurement clusterisation}
\RC The approaches to avoid measurement clusterisation, described on page
3, are described as pragmatic and "fundamentally altering the optimal
design problem".

\RC Firstly, this is not strictly true. Suppose measurements correspond to
time, it is not possible to take two or more observations at the same
time, and there should be a minimum time period between measurements.
Implementing this minimum time period is not pragmatic, rather it is
characterising the application correctly. One could imagine a similar
thing occurs with the electrode locations on the skin in impedance
tomography: the physical size of the sensor imposes a minimum
distance between sensors.

\AR To a certain extent I agree: the impossibility of taking identical
measurements is indeed natural. However, this is not encoded in the
mathematical formulation of the problem. One can then take one of two
approaches: either accept what the mathematics suggest, even if it
counterintuitive, or find ways to circumvent the mathematics because
the result seems non-intuitive. I expanded this discussion in the
revised manuscript. See below and the discussion preceding it in
Section \ref{subsub:implications} of the revised manuscript. 

\begin{quote}
  Given the above discussion, it is our belief that the methods
  suggested by other authors to avoid clusterization are merely
  overlooking the problem. We do not view clustered designs as
  inherently bad on their own. Rather, we suggest that if a clustered
  design arises, then a practitioner should revisit their probabilistic
  and physical modeling assumptions. If the practitioner is confident in
  their assumptions, then clustered designs should be avoided only to
  the extent necessitated by the physical measuring apparatus.
\end{quote}


\RC Secondly, the paper seems to suggest “imposing correlations
between observations” as a solution. However, isn’t this imposition
pragmatic as well? I suppose it is argued that the $\obs \eps'$ term
is accounting for model error: but it is still pragmatic: the data
acquisition given by equation (1) is not the true data-generating
process.

\AR I do not support imposing correlations in the paper, I just
used correlations to show how my framework can be used to rigorously
prove what is widely accepted. Other referees also understood that the
paper endorses imposing correlations and I accept that this is my
error in writing. See text answering previous comment.


\subsection{Minor comments}
\RC The author uses two different forms of asterisk: one for
optimality of the design (e.g. Definition 3) and one for the
adjoint. Suggest changing the one for optimality.

\AR I changed $\obs^\star$ to $\opt$.

%% \begin{quote}
%%   \begin{definition}\label{def:d_optimality}
%%     We say \DIFdelbegin \DIFdel{\( \obs^{\star} \)}\DIFdelend
%%     \DIFaddbegin \DIFadd{alive}\DIFaddend

%%     \(\opt\) is \emph{D-optimal} if \(\opt =
%%     \argmax_{\obs} \tar(\obs)\), where entries of \(\obs \in (\hilo^*)^m\)
%%     are constrained to some allowed set of measurements in \(\hilo^*\).
%%   \end{definition}
%% \end{quote}
  
\RC I do not like the use of the term measurements as used in the
paper. The measurements are the $\data$’s since these are a result of
measuring a physical quantity. However, the paper refers to the
quantities that are controlled as the measurements.

\AR I was not able to find a better term to replace "measurements", so
I stuck to it.

  
\RC Proof of Proposition 12 uses $\eps$ as part of the regularisation
trick but this has been used previously for error (e.g. equation (1)).
  
\AR I changed it to $\nu$. You will not find the revised proof in the
manuscript since I moved it to the Supplementary.




\section{Reviewer 3}
\RC The paper studies Bayesian D-optimal design (that is, the task of
allocating measurements in order to maximise the Kullback-Leibler
divergence between the prior and posterior distribution) in an inverse
regression type model in Hilbert spaces with Gaussian measurements
errors. For Gaussian priors, the paper provides a characterisation of
the optimality criterion and the optimal design. Some conclusions
about the phenomenon of clusterisation of measurements known in
literature are then drawn.


\RC While the results in the paper provide some novel insights on the
topic, I believe the current version of the manuscript is far from the
standard required by Bayesian Analysis. Let me raise immediately my
main concern: on p.4 the author describes the generality of
measurements clusterisation as the first main question being explored,
which is answered via, quoting, "randomized numerical simulations that
exhibit clusterization more than 95\% of the time (see the code in
supplementary material)". However, in the version of the supplementary
materials I had access to there is no further mention of these
numerical simulations and results; only a link to a Github repository
is provided, which is however not functioning.

\AR I made sure the \href{https://github.com/yairdaon/OED}{repository}
is functioning and well-documented.


\RC Evidently, this issue needs to be fixed before any future
consideration can be given to the manuscript. For example, a detailed
description of the repeated experiments could be provided either in
the main article or in the supplement, with tables indicating the
various values of the parameters for the simulations and the precise
percentages obtained. Also, the code needs to be publicly available
for reproducibility.

\AR I appreciate the suggestion. I added a section reproducing the
abovementioned numerical experiments to the main article:


\begin{quote}
  In the proof of Theorem \ref{thm:char} we utilize Lemma
  \ref{lemma:free} to construct D-optimal designs. We implement this
  construction with the goal of testing numerically how prevalent are
  clustered designs. To this end, we would like to generate random
  prior eigenvalues $\lambda_j$, fix $m$ and $k$, find a D-optimal
  $\opt^*\opt$ and then utilize the construction of
  Theorem~\ref{thm:char} and Lemma~\ref{lemma:free} to find $\opt$.


  To simplify things, we directly sample rank $\opt^*\opt$. We iterate
  over the number of measurements $m \in \{4,\dots, 24\}$, and for
  every $m$ we then iterate over $k:=\rank \obs^*\obs \in \{2,\dots,
  m-1\}$. For each pair $m,k$ we repeat the following steps $N=5000$
  times:
  \begin{enumerate}
  \item Generate random diagonal $D\in \mathbb{R}^{k\times k}$ with
    entries $\log (d_i) \sim \mathcal{N}(50,15)$ and normalize so that
    $\ttr D = m$.
  \item Conjugate $D$ by a random orthogonal matrix to form a positive
    semi-definite $M := UDU^t \in \mathbb{R}^{k\times k}$. This $M$
    represents $\opt^*\opt$.
  \item Apply the construction of Lemma \ref{lemma:free} to calculate
    $A$ such that $AA^t = M$, where $A$ has unit norm
    columns. I.e.~find the optimal design $\opt$.
  \item Since $A$ corresponds to $\opt$, its columns correspond to
    measurement vectors. We call $A$ "clustered" if $A$ has two or
    more identical columns (up to some numerical precision threshold).
  \end{enumerate}
  We then calculate the fraction of clustered designs of the simulations
  we ran, for each pair $m,k$. Clusterization occurred at high rates
  ($>99.9\%$) whenever $m-k > 1$; see Fig.~\ref{fig:sim_AAt}. Hence, in
  these simulations, clusterization is a generic property. However, when
  $m-k = 1$, clusterization does not occur. We do not why this is so and
  further investigation into this phenomenon is out of scope for the
  current study.
  
  Full results are located in the \texttt{simulations.csv} file within
  the accompanying \href{https://github.com/yairdaon/OED}{repository}.
  Code implementing the experiments described above is located in module
  \texttt{zeros.py} of said repository. Runtime should be less than 30
  minutes on any reasonably modern laptop (it took 12 minutes on the
  author's laptop).
\end{quote}

\begin{figure}
    \centering
    \includegraphics[height=0.5\textwidth]{figs/simulations.png}
    \caption{Fraction of clustered $A$ for $AA^t = M$ and $M$
      generated randomly (see text and repository for details on
      generating $M$). It is evident that when $m-k \geq 2$ clusterization
      is ubiquitous, whereas for lower $m-k$ clusterization does not
      occur.}
  \label{fig:sim_AAt}
\end{figure}

\subsection{Further comments}
\RC I have a number of further comments in regards to the results and the
overall presentation of the manuscript.


\RC D-optimal design vs. statistical recovery rates: being more
familiar with the statistical inverse problem literature than with
D-optimal design, the clusterisation phenomenon demonstrated in this
paper raises the question as to whether consistent recovery of the
unknown ($\param$ in the notation of this paper) is possible under
this choice. All the results I am aware of either assume white
noise/equally spaced design (e.g.~\cite{knapik2011}) or measurement
locations sampled uniformly at random (e.g.~\cite{nickl2023}). There
are also some results with more general design under conditions on the
fill-distance of the grid (e.g.~\cite{teckentrup2020}). All of these
specifically prevent clusterisation, which is key to the statistical
analysis. Hence, it is not clear to me even if D-optimal design should
be pursued at all, if it indeed it leads to clusterisation as implied
by the present paper. I think a discussion on this would be
illuminating for the reader and help draw a connection to the broader
statistical inverse problems literature.

\AR I added a discussion on this subject. I do not argue against the
use of clustered designs imply that clustered designs, but I believe
that when they arise it should be a warning signal to a practitioner
regarding the validity of their physical model and Bayesian
assumptions:

\begin{quote}
  
  
\end{quote}

\RC Figure 1: It is hard to discern whether the measurements locations
are perfectly overlapping or just very closely placed. Perhaps using
dots with numbers here would help the readability.
  
\AR It is impossible to visually discern the measurements, even when
resolution is increased. These measurements are not identical
numerically, but they are identical to five decimal digits. Since I
find these points via optimization over point location in
$\Omega=[0,1]$ I view a numerical error $<10^{-5}$ as reasonable.


\RC Experiment with the 1D heat equation: the experiment showcases
some interesting phenomena, but also raises many questions. For
instance: I would expect the dimensionality of the working domain
(here the unit interval [0, 1]) to play a role, since larger dimension
allow for higher freedom in placing the design points.

\RC Further, a dimensionality effect should also arise directly from
Theorem 1 via the prior covariance eigenvalues: in the experiment the
prior covariance operator is set to be equal to the inverse Laplace
operator $(-\Delta)^{-1}$. In d-dimensional domains, its eigenvalues
$\lambda_j$ are known to follow Weyl’s asymptotics, growing as
$j^{2/d}$, which should then impact the index after which the
eigenvalues are ‘thresholded’ by the procedure.

\AR These are two great points. Indeed, the growth of eigenvalues of
the Laplacian is different in higher dimensions. This is why in higher
dimensions we would have to take as prior \(u_0 \sim
\mathcal{N}(\param_0, (-\Delta)^{-\gamma})\), for some \(\gamma >
d/2\) in order to maintain regularity of prior realizations. I decided
not to include a discussion on the choice of priors in higher spatial
dimensions because (1) it would require too much exposition and make
me digress from the main points I wanted to convey and (2) I view
implementing experiments in higher dimensions as out of the scope of
the current study.



\begin{quote}
  We choose a prior for the initial condition $u_0 \sim \mathcal{N}(0,
  (-\Delta)^{-1})$, with homogeneous Dirichlet boundary
  condition. This choice ensures the posterior is well-defined and
  prior realizations are well-behaved, see \cite[Theorem 3.1 and Lemma
    6.25]{Stuart10} for details.
\end{quote}

  
\RC Also, it should be relatively easy here to also study empirically
the effect of the correlated error model in preventing
clusterisation. Investigating all these issues in the context of the
presented numerical set up would considerably enlarge the breath of
the present work.

\AR I added relevant simulations to a new section Numerical
Experiments:


\begin{quote}
  In order to verify the results of Section
  \ref{section:non_vanishing}, we run simulations of the inverse
  problem of the 1D heat equation with nonvanishing model error
  \(\modcov = \prcov^2 \). Indeed, including model correlation pushes
  measurements apart, see Fig.~\ref{fig:corr_errors}. Code generating
  Fig.~\ref{fig:corr_errors} is located in module
  \texttt{clusterization.py} in the accompanying
  \href{https://github.com/yairdaon/OED}{repository}.
\end{quote}
\begin{figure}
  \centering
  \includegraphics[height=0.5\textwidth]{figs/dst_modelError4.png}
  \caption{Model correlation mitigates clusterization. We add a
    model correlation term to the error terms in the 1D heat
    equation inverse problem. Lo and behold, measurements are not
    close anymore and are pushed away thanks to the model error
    term.}
    \label{fig:corr_errors}
\end{figure}

  
\RC I found very hard to understand Theorem 1 immediately at the end
of Section 1, as all the necessary background is introduced only
later. Since the same result is also stated and proved in Section 5, I
would consider cutting it from Section 1. The paragraphs in p.4 and 5
already do a good job presenting the results, and the repetition of
the formal statement does not seem necessary here.

\AR I removed the statement of Theorem 1 from the introduction.


\RC Denoting the optimal design in Theorem 1 by simply $\obs$ is
somewhat confusing, cf. the equation in the second item. Perhaps a
separate notation such as $\bar{\obs}$ would help readability.

\AR Thank you for the comment, I completely agree. Optimal designs are
now denoted $\opt$.

  
\RC Section 1.2 seems altogether unnecessary; the inverse regression
model considered in the paper are a modelisation of many real-world
phenomena, and they are routinely studied in statistical papers.

\AR I am glad you view this part as unnecessary, but I feel the model
might feel too abstract to some practitioners. Thus, I choose to keep
this part to avoid complaints from other readers.



\RC What does ‘strongly smoothing’ operator means in the third line of Section 2.1?

\AR Basically I meant that its eigenvalues decrease "quickly". I
understand this is not a common or well-defined term so I removed the
relevant sentence.

  
\RC Section 4: the conclusion reached at the end of the section on
p.15 seem to imply that in the noiseless limit, repeating any
measurement does not improve the objective criterion, if correlated
error model are present.

\AR Indeed, this is very intuitive: when no observation error is
present, a repeated measurement does not tell us anything new.


\RC Is it clear however that adding any (non-repeated) measurement
always strictly increases it? I.e., there could be situations where
adding one measurement does not change the value of the optimised
criterion?

\AR Thanks for this insightful comment! Such cases are possible but I
view them as pathological. I added a short discussion reproduced
below:



\begin{quote}
  It is worth noting that by the nonnegativity of the KL divergence,
  $\tar$ cannot decrease upon adding measurements. However, we can
  construct examples where the posterior does not change upon taking a
  new measurement e.g.~if the prior variance vanishes on some
  eigenvector and a measurement is taken on said eigenvector. We do
  not expect a measurement to generate no information gain whatsoever
  in any realistic scenario, and ignore such pathologies.
\end{quote}

\RC Also, can the conclusion drawn here be extended to the case of
noisy observations $\sigma > 0$?

\AR I do not expect this conclusion to hold: if noise is
    present, even a repeated measurement will result in some
    information gain and an increase in design criterion, since the
    observation error is effectively reduced by a factor of
    $1/\sqrt{2}$ for the repeated measurement.


  
\RC At the end of Section 4 on p.15 it is mentioned that $\tar$ is not
defined for $\sigma^2= 0$, but in the latter case, could not a
Gaussian posterior measure be still defined as in Gaussian process
regression, e.g~\cite{rasmussen2006}].

\AR Indeed, one can define a posterior, so a KL divergence from said
posterior to prior could be calculated and $\tar$ evaluated. However,
taking $\sigma = 0$ in Theorem 1 (the theorem of
\cite{AlexanderianGloorGhattas14}) gives $\tar \equiv \infty$. I did
not study this subject further.

  
\RC Theorem 1 is expressed in terms of the prior covariance matrix,
which is a user specified quantity. Hence, it seems to me that all
sort of design behavior under the D-optimal criterion can occur by
engineering the prior. It would perhaps be informative to study in
more details some representative example of inverse problem and
Gaussian prior. For example, the recovery of the initial condition
with the ‘Mat\'ern-like’ Gaussian prior laid out in the supplement are
good candidates.

\AR the Mat\'ern-like prior actually arises by using a Laplacian-like
operator as a prior \cite{rue2011}. So a study of this example (with
and without correlated errors) is already included in the paper. I am
afraid implementing another inverse problem is out of scope for the
present paper since I was not able to install common packages for
inverse problems \cite{attia2023pyoed, VillaPetraGhattas16,
  VillaPetraGhattas18, VillaPetraGhattas21} on my machine.
  

\RC For these it is still not clear how the conclusion from Theorem 1
translates into the Figure 1, also because I am unsure about the
applicability of the result to point evaluations.

\AR Indeed, point evaluations $\delta_x$ are not in any function space
I consider in the paper since my analysis is limited to Hilbert
spaces. Point evaluations can be approximated well in
e.g.~$L^2(\Omega)$ so in my opinion this is not a deal breaker. The
only caveat is that upon approximating point evaluations e.g.~by
$R(x;t) = t\mathbb{1}_{-t/2, t/2}(x)$ we will need to change the norm
constraints to the something different (i.e.~$\|R(\cdot;t\|$) but that
does not change the analysis. I barely mentioned this issue in the
paper since I think it is confusing and adds little context. See,
however, a footnote I added which is reproduced below:
  


\begin{quote}
  The alert reader will likely ask how do we reconcile point
  measurements $\delta_x$ as suggested by the formulation of the 1D
  heat equation with working in Hilbert spaces. We don't. We follow
  standard practice in the literature and restrict our analysis to
  Hilbert spaces. We can satisfy ourselves with the fact that point
  evaluations could be approximated in a standard Hilbert space like
  $L^2(\Omega)$.
\end{quote}
  
\RC In particular, how the conclusion that with $m = 4$ measurements
the D-optimal procedure will aim to ignore the third and fourth
eigenvalue/function was drawn?

\AR I explain this in more detail in the paper:


\begin{quote}
  Building on Theorem \ref{thm:char}, we can now give a compelling
  explanation to the measurement clusterization we observed for the
  inverse problem of the heat equation (Fig.~\ref{fig:eigenvectors}).
    
  Consider $\fwd$ and $\prcov$ from \emph{the inverse problem of the
  heat equation}. As before, we denote the eigenvalues of
  $\fwd\prcov\fwd^*$ by $\lambda_j$. We input these eigenvalues into
  our \emph{generic} model, and find a D-optimal design $\opt$ for our
  generic model using Theorem \ref{thm:char}. In our generic model,
  the measurements we take are best utilized in reducing uncertainty
  for the first $k$ eigenvectors. So, a D-optimal design arising from
  our \emph{generic model} completely avoids measuring eigenvectors
  $k+1$ and above.

  Of course, in a real life problem --- such as the inverse problem of
  the 1D heat equation --- it is likely impossible to find measurement
  locations for which all eigenvectors $k+1$ and above are
  zero. However, if the eigenvalues of $\fwd\prcov\fwd^*$ decay
  quickly (recall the square-exponential decay for eigenvalues of the
  1D heat equation in eq.\eqref{eq:decay}), a D-optimal design will
  try to balance measuring a small number (i.e.~$k$) of the leading
  eigenvectors.

  The abovementioned balance is explored in
  Fig.~\ref{fig:eigenvectors}. We allow $m=4$ measurements in $\Omega
  = [0,1]$ and observe that D-optimal measurement locations are
  clustered at $x_1 = 0.31$ and $x_2 = 0.69$. Upon close inspection of
  the scaled eigenvectors of $\fwd \prcov \fwd^*$, we first observe
  that eigenvectors $3$ and above have negligible prior
  amplitude. Since we only have $m=4$ measurements at our disposal, we
  interpret these results, following Theorem \ref{thm:char}, as
  implying we should only care about measuring the first and second
  eigenvectors. Then, we note the D-optimal $x_1,x_2$ present a
  compromise between the amplitude of the first and second
  eigenvectors. For example, a measurement at $x=0.5$ would have
  ignored the second eigenvector altogether, since the second
  eigenvector is zero at $x=0.5$.

  Now we can understand measurement clusterization for the inverse
  problem of the heat equation. A D-optimal design attempts to measure
  the first $k$ eigenvectors of $\fwd \prcov \fwd^*$. But there may be
  (spatial) limitations on where these $k$ eigenvectors have large
  amplitude. For the inverse problem of the heat equation there are
  two spatial locations that present a good compromise between the
  amplitudes of the first and second eigenvectors, namely $x_1$ and
  $x_2$ --- see Fig.~\ref{fig:eigenvectors}. We have $m=4$
  measurements at our disposal but only two spatial locations that are
  a good compromise between the first and second scaled
  eigenvectors. Thus, clusterization arises as a consequence of the
  pigeonhole principle.

\end{quote}
\begin{figure}\label{fig:eigenvectors}
  \centering
  \includegraphics[width=\textwidth]{figs/eigenvectors_dst_scaled.png}
  \caption{D-optimal measurement locations ($m=4$ measurements) and
    weighted eigenvectors for finding the initial condition of the 1D
    heat equation. Measurement locations and weighted eigenvectors are
    plotted over the computational domain $\Omega = [0, 1]$
    (x-axis). Measurement clusterization occurs approximately at
    $0.31$ and $0.69$. These two locations are a compromise between
    the magnitudes of the first and second eigenvectors, which are the
    eigenvectors that a D-optimal design aims to measure. Allocating
    $m=4$ measurements into two locations results in clusterization,
    according to the pigeonhole principle.}
  \label{fig:why}
\end{figure}
 



\bibliographystyle{apalike}
\bibliography{../../lib.bib}

\end{document}
