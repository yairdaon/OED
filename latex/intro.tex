\section{Introduction}\label{section:intro}
Measurements play a fundamental role in generating observations and
are indispensable to any inference process. In the realm of data
science, selecting the optimal set of measurements holds particular
significance when inferring parameters of a physical process. Unlike
many other fields, where measurements are fixed, in this context, we
have the freedom to choose which measurements to take. This freedom
should be taken advantage of to choose measurements that enhance
accuracy, reduce costs, or achieve both objectives
simultaneously. Whether measurements involve specifying electrode
locations on the skin in electric impedance tomography
\cite{horesh2010impedance}, determining certain wavelengths in MRI
\cite{horesh2008mri}, or positioning sensors for detecting
ground-reflected waves in the search for oil
\cite{horesh2008borehole}, the selection of optimal measurements,
referred to as the problem of \emph{optimal design}, becomes
crucial. To achieve this, researchers optimize specific \emph{design
criteria}, with A- and D-optimality being two of the most widely
recognized and extensively studied design criteria
\cite{Chaloner1995}.


Surprisingly, A- and D-optimal designs have been observed to yield
remarkably similar measurements in certain cases \cite{fedorov1996,
  hooker2009, fedorov2012, Ucinski05, neitzel2019sparse}. To
illustrate this point, we consider a toy inverse problem: inferring
the initial condition of the 1D heat equation (details in Section
\ref{section:example}). In Figure \ref{fig:clusterization
  illustration} D-optimal measurement locations are shown for
different numbers of measurements. Notably, for six measurements, a
D-optimal design yields two sets of measurements that are almost
indistinguishable from one another. This intriguing phenomenon is
known as \emph{measurement clusterization} \cite{Ucinski05}, and we
refer to a design that exhibits measurement clusterization as a
\emph{clustered design}.


\begin{figure}
  \begin{tikzpicture}[thick, scale=1.3, every node/.style={scale=0.99}]
    \begin{axis}
      %% title={Posterior Pointwise Standard Deviations and D-Optimal
      %% Measurement Locations}
      [xmin = 0, xmax = 3.14, xlabel = {$x \in \Omega$}, ylabel =
        posterior pointwise std, ymin = 0,
      %compat = 1.3,
      % ymax   = 130,
      % ytick = \empty,
      legend cell align=left,
      % legend style={at={(0.45,0.2)}}
      legend pos= outer north east 
      ]
      % \draw[black!30!white, thin] (50,0) -- (50,130);
      % 
      %% \addplot [thin, black, mark=none] table{stdv-heat-sens1-var1.txt};
      %% \addlegendentry{1 measurements};
      
      %% \addplot [thin, blue, mark=none] table{stdv-heat-sens2-var1.txt};
      %% \addlegendentry{2 measurements};
      
      %% \addplot [thin, red, mark=none] table{stdv-heat-sens3-var1.txt};
      %% \addlegendentry{3 measurements};
      
      \addplot [thin, green, mark=none] table{stdv-heat-sens4-var1.txt};
      \addlegendentry{4 measurements};
      
      \addplot [thin, purple, mark=none] table{stdv-heat-sens5-var1.txt};
      \addlegendentry{5 measurements};
      
      \addplot [thin, cyan, mark=none] table{stdv-heat-sens6-var1.txt};
      \addlegendentry{6 measurements};

  
      %% \addplot [black,  only marks, mark=x, mark size=1.5] 
      %% table{locs-heat-sens1-var1.txt}; 
      %% \addplot [blue,   only marks, mark=x, mark size=1.5]
      %% table{locs-heat-sens2-var1.txt}; 
      %% \addplot [red,    only marks, mark=x, mark size=1.5]
      %% table{locs-heat-sens3-var1.txt};
      \addplot [green,  only marks, mark=*, mark size=1.5] 
      table{locs-heat-sens4-var1.txt}; 
      \addplot [purple, only marks, mark=*, mark size=1.5] 
      table{locs-heat-sens5-var1.txt}; 
      \addplot [cyan,   only marks, mark=*, mark size=1.5] 
      table{locs-heat-sens6-var1.txt}; 
  
      
    \end{axis}
  \end{tikzpicture}
  \caption{Measurement clusterization for the inverse problem of
    inferring the initial condition of the 1D heat equation. Posterior
    pointwise standard deviations (lines) are plotted over the
    computational domain $\Omega = [0, \pi]$, for varying numbers of
    measurements. Measurement locations (circles) were chosen
    according to the D-optimality criterion. Measurement
    clusterization occurs for six measurements: Only four measurement
    locations are visible, since two pairs of measurements are so
    close together they are indistinguishable.}
  \label{fig:clusterization illustration}
\end{figure}


Researchers widely agree that measurement clusterization is
undesirable \cite{fedorov1996, hooker2009, fedorov2012, Ucinski05,
  neitzel2019sparse}, prompting the exploration of various remedies to
address this issue. One approach involves merging ``close''
measurements \cite{fedorov2012}; however, this strategy merely
overlooks the phenomenon of measurement clusterization. An alternative
solution lies in \emph{clusterization-free design}s, where measurement
locations are deliberately chosen to be distant from one another. This
can be achieved by imposing distance constraints between measurements
or by introducing correlated errors that account for both observation
error and model misspecification \cite{Ucinski05}. For instance, in
the context of time-series analysis for pharmacokinetic experiments,
measurement clusterization can be mitigated by incorporating the
modeling of auto-correlation time within the noise terms
\cite{hooker2009}.


In spatial problems involving choice of measuremnts within a domain
$\Omega \subseteq \mathbb{R}^d, d=1,2,3$, many researchers circumvent
the problem of measurement clusterization by choosing measurements
from a coarse grid in $\Omega$ \cite{koval2020, alexanderian2021,
  attia2020, alexanderian2014, alexanderian2016,
  alexanderian2018efficient}. However, this approach incurs a
significant computational cost as it requires solving a difficult
combinatorial optimization problem for choosing measurementst from a
finite set. The combinatorial optimization problem is usually relaxed
by first assigning optimal measurement "weights" in $\mathbb{R}_+$ to
the potential measurement locations. Some researchers incorporate a
sparsifying $\ell_1$ penalty term into the design criterion, which is
subsequently thresholded to achieve the desired binary design over the
coarse grid \cite{horesh2008borehole}. Others progressively relax the
$\ell_1$ penalty to an $\ell_0$ penalty via a continuation method
\cite{alexanderian2016, alexanderian2014}. A clever alternative is to
cast the problem of finding optimal measurement weights as a
stochastic optimization problem \cite{attia2022stochastic}. All of the
aforementioned methods may indeed find a binary optimal design
restricted to a given coarse grid. However, none addresses one
fundamental issue: the restriction of measurement locations to a
coarse grid in $\Omega \subseteq \mathbb{R}^d, d=1,2,3$ inherently
results in a sub-optimal design.

Avoiding measurement clusterization is a pragmatic approach:
intuitively, we recognize that measurement clusterization is
undesirable, even though the underlying reasons may not be fully
clear. Consequently, we strive to prevent it and devise various
methodologies to avoid measurement clusterization. Yet each and every
one of these methodologies achieves this objective by imposing
restrictions on measurement locations, thereby fundamentally altering
the optimal design problem. To the best of my knowledge, no previous
study has tried to address some seemingly simple yet fundamental
questions:
%
Why does measurement clusterization occur?
%
Why does imposing correlations between observations alleviate
measurement clusterization?
%
Should we aim to avoid measurement clusterization?
%
Is it possible to substitute an optimal clustered design with an
equally optimal non-clustered design?
%Can an optimal clustered design be relpaced with an equally optimal
%non-clustered design?


\subsection{Contribution}
The primary objective of this study is to provide a comprehensive
understanding of measurement clusterization by addressing the
aforementioned questions. Our focus centers around investigating the
Bayesian D-optimality criterion, which involves maximizing the
expected Kullback-Leibler divergence between the posterior and prior
measures \cite{CoverThomas91, Chaloner1995}. We conduct an analysis of
Bayesian D-optimal designs within the context of linear inverse
problems in Hilbert spaces. We propose a novel relaxed model for
D-optimality that maintains analytical tractability and enables the
identification of D-optimal designs using Lagrange multipliers. This
analytical framework facilitates the exploration of the questions
posed at the end of the previous paragraph:


\begin{enumerate}
\item \label{q:why} \textbf{Why does measurement clusterization
  occur?} In Section \ref{section:vanishing}, we provide an insightful
  explanation for the optimality of clustered designs when no model
  error is present. We demonstrate that measurement clusterization
  poses no obstruction to D-optimality and that clustered design are
  preferred by numerical algorithms.

  %% We conjecture that the prevalence of measurement clusterization
  %% arises due to the ease of discovering clustered designs.

\item \label{q:mitigate} \textbf{Why does imposing correlations
  between observations alleviate measurement clusterization?} In
  Section \ref{section:non vanishing}, we rigorously demonstrate the
  role of model error in mitigating clusterization, thereby
  corroborating earlier observations made by various researchers.

\item \label{q:avoid} \textbf{Should we aim to avoid measurement
  clusterization?} Based on the analysis conducted in this study, we
  did not find any compelling reason to explicitly avoid optimal
  clustered designs.

\item \label{q:replace} \textbf{Is it possible to substitute an
  optimal clustered design with an equally optimal non-clustered
  design?} In Section \ref{section:vanishing}, we answer this question
  in the affirmative, although we show that numerical experiments
  conducted using our model indicate a strong preference for clustered
  designs.
\end{enumerate}

The cornerstone of our investigation into measurement clusterization
is Theorem \ref{thm:char} proven in Section
\ref{section:vanishing}. The key insight of Theorem \ref{thm:char}
lies in its fifth component, which highlights that a D-optimal design
aims to uniformly reduce posterior uncertainties across all posterior
covariance eigenvectors. A similar conclusion was reached by Koval et
al.~\cite{koval2020}, who showed that A-optimal designs should be
constructed in the space of \emph{observations}.


\begin{restatable}[D-optimal designs with vanishing model error]{theorem}{char}\label{thm:char}
  Let:
  \begin{itemize}
  \item $\hilp, \hilo$ Hilbert spaces.
  \item $\fwd:\hilp \to \hilo$ a linear compact operator.
  \item $\pr \sim \mathcal{N}(0, \prcov)$ prior Gaussian measure on $\hilp$,
    where $\prcov:\hilp \to \hilp$ is the prior covariance operator.
  \item $\obs: \hilo \to \mathbb{R}^m$ measurement operator, where $m
    \in \mathbb{N}$ is the number of measurements taken.
  \item $\sigma^2 \in \mathbb{R}_{+}$ observation noise variance,
    $\data = \obs \fwd \param + \eps$, where $\eps \in \mathbb{R}^m$
    is iid $\mathcal{N}(0, \sigma^2)$ noise.
  \item $\post$ the posterior measure, with covariance $\postcov$.
  \item A D-optimality design criterion
    \cite{AlexanderianGloorGhattas14}:
    \begin{align*}
      \begin{split}
        \tar(\obs) :&= \mathbb{E}_{\data}\left [ D_{\text{KL}} (\post || \pr ) \right ] \\
        % 
        % 
        % 
        &= \frac12 \log \det ( I + \sigma^{-2} \prcov^{1/2} \fwd ^*
        \obs^* \obs \fwd \prcov^{1/2}).
      \end{split}
    \end{align*}
  \item $\{\lambda_i\}_{i=1}^\infty$ eigenvalues of $\fwd\prcov\fwd^*$
    in decreasing order of magnitude.
  %% \item $\{\ev_i\}_{i=1}^\infty$ their corresponding eigenvectors.
  \item $\{\eta_i\}_{i=1}^\infty$ eigenvalues of $\obs^*\obs$.
  \item $\obs$ a D-optimal design operator
    \begin{equation*}
      \obs = \argmax_{\|\meas_j\| = 1, j=1,\dots,m}\tar(\obs).
    \end{equation*}
  \end{itemize}

  Then:
  \begin{enumerate}
  \item  $\tr{\obs^*\obs} = m$.
  \item $\obs^*\obs$ and $\fwd\prcov\fwd^*$ are simultaneously
    diagonalizable.
  \item $k := \rank \obs^*\obs \leq m$ and
    \begin{equation*}
      \tar(\obs) = \frac12 \sum_{i=1}^{k} \log (1 + \sigma^{-2}\lambda_i\eta_i). %= \frac12 \sum_{i=1}^{m} \log (1 + \sigma^{-2}\lambda_i\eta_i).
    \end{equation*}
  %% \item 
  %%   \begin{equation*}
  %%     k = \argmax \left \{ k:\lambda_k^{-1} < \sigma^{-2}\frac{m}{k} + \frac{1}{k} \sum_{j=1}^{k}
  %%     \lambda_j^{-1} \right \}.
  %%   \end{equation*}
  \item
    \begin{equation*}
        \eta_i = \begin{cases}
          \frac{m}{k} - \sigma^2 \lambda_i^{-1} + \sigma^2 \frac{1}{k} \sum_{j=1}^k \lambda_j^{-1} & 1 \leq i \leq k \\
          0 & i > k 
        \end{cases}.
    \end{equation*}
  \item The covariance of the pushforwad $\fwd_{*} \post$ is $\left (
    (\fwd \prcov \fwd^*)^{-1} + \sigma^{-2} \obs^*\obs \right )^{-1}$
    and has eigenvalues
    \begin{equation*}
      \theta_i =
      \begin{cases}
        \left(\frac{\sum_{j=1}^k \lambda_j^{-1} + \sigma^{-2}m}{k} \right )^{-1} & i \leq k \\
        \lambda_i &  i > k 
      \end{cases}
    \end{equation*}
  \end{enumerate}
%\end{theorem}
\end{restatable}

Other contributions include three useful lemmas: In Lemma
\ref{lemma:free} we decompose a symmetric positive definite matrix $M
\in \mathbb{R}^{k \times k}$ with $\ttr M = m \in \mathbb{N}$ as $M =
AA^t$, where $A$ has unit norm columns. In Lemma \ref{lemma:MDL} we
prove a matrix determinant lemma in Hilbert spaces. Finally in Lemma
\ref{lemma:lax} we generalize a lemma for calculating
$\frac{\der}{\der t} \log \det (I + X(t))$, where $X(t)$ is an
operator valued function \cite{Lax07}.


\subsection{Limitations}\label{subsec:limitations}
The main limitation of this study is that our generic model does not
correspond to any specific real-life problem. It is generic enough to
be analytically tractable, but one may argue it is too far removed
from any real application. To these claims I would argue that
scientists have a long history of studying models that are bare-bones
simplifications of real systems, e.g. the Ising model
\cite{cipra1987}, the Lorenz system \cite{brin}, the Lotka-Volterra
equations \cite{logan2006}, the Carnot engine \cite{kardar2007}, and
many others.


%% \subsection{An Example of Measurement Clusterization}\label{subsec:example}
