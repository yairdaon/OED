\section{Introduction}\label{section:OED intro}
Experimental design is an important part of many scientific
investigations. When considering an inverse problem, one can often
specify sensor locations (e.g.\ in geophysics and oceanography
applications), certain wavelengths (e.g.\ in MRI) or wave reflections
from the ground (e.g.\ searching for oil or using a radar). Whatever
the allowed set of observations is, one should select the optimal
observations to take, in order to increase accuracy, reduce costs, or
both.

Designing experiments is usually done by optimizing some \emph{design
criterion}. This is true both for frequentists
\cite{silvey2013,Ucinski05} as well as for Bayesians
\cite{Chaloner1995} (see \cite{Chaloner1995} for an investigation of
the analogy between the two approaches). Among the plethora of design
criteria, \emph{D-optimal design} have one of the simplest and most
appealing motivation in the Bayesian context \cite{Chaloner1995}: a
D-optimal design maximizes the expected information gain (KL
divergence \cite{CoverThomas91}) between posterior and prior. For a
linear model in finite dimensions, with Gaussian prior and noise, a
D-optimal design minimizes the determinant of the posterior covariance
matrix. In a frequentist setting, a D-optimal design minimizes the
volume of the uncertainty ellipsoid \cite[page 16]{Ucinski05}, but
this is done for the Fisher information matrix and not the posterior
covariance. However, \cite{Chaloner1995} show that the latter is just
a regularized version of the former.

The previous discussion is classical for experimental design when
inference takes place over a finite (not too large) number of
parameters. The subject of optimal experimental design on function
spaces in a Bayesian context was pioneered by
\cite{AlexanderianGloorGhattas14, AlexanderianPetraStadlerEtAl16,
  AlexanderianPetraStadlerEtAl14}. Similarly to the finite dimensional
case, a D-optimal design still maximizes the KL divergence between
posterior and prior. For a linear model with Gaussian prior and noise,
this amounts to minimizing the determinant of the posterior covariance
operator (understood as a product of its eigenvalues). Some
difficulties arise in the process, but remedies can be found as shown
in \cite{AlexanderianGloorGhattas14}.

It seems counter intuitive that when one computes an optimal design
using the D-optimality criterion, the optimization process results in
observations that are very similar. For example, if an observation is
thought of as measuring some function value at $\x \in \Omega
\subseteq \R^d, d=1,2,3$ (with added error) then the optimization
procedure sometimes places sensors in very close proximity to each
other (as can be seen in figure \ref{fig:clusterization
  illustration}). Following \cite{Ucinski05}, we refer to this
phenomenon as \emph{sensor clusterization}.

\subsection{Related Work}
The phenomenon of sensor clusterization seems to be known in several
different contexts. In a frequentist and finite-dimensional context,
\cite{fedorov1996} and \cite[chapter 2.4.3]{Ucinski05} discuss this
phenomenon and suggest an approach called clusterization-free design.
In such designs, the user enforces observation locations to be far
from each other. One way to do this is by introducing correlated
errors which account for both observation error and model
error. Another method considered is imposing distance constraints
between observations. A somewhat different approach is suggested in
\cite[page 49]{fedorov2012}, where close observations are merged --- a
procedure which obviously does not avoid clusterization. The same
problem arises in time-series analysis for pharmacokinetic
experiments. The authors of \cite{hooker2009} suggest modeling
auto-correlation time in the noise model, which is equivalent to the
correlated errors mentioned above.

Any of the above mentioned approaches might serve as a remedy and push
sensors away from each other. Yet, none offers any insight as to why
clusterization occurs. Also, as better models are employed, model
error is decreased and the clusterization phenomenon will eventually
reappear. While these approaches are practical and help us avoid the
problem, they do not provide insight as to why sensors are clustering.

In the inverse problems community, work is mostly computational and
less theoretic. Model errors were considered in \cite{attia2020,
  koval2020}. The former study is focused on inferring a Quantity of
Interest (QoI). The focus of the latter is reducing forward solves,
using randomized linear algebra. Both studies present numerical
techniques for finding optimal designs when model error is
present. Both are restricted to linear inverse problems (although in
the latter the authors use their method on a nonlinear problem by
taking a Laplace approximation for the posterior). Both find an
optimal design by first solving a continuous problem for sensor
weights. Said solution is then sparsified to give a binary
design. Both studies are successful in the task of Bayesian
inversion. However, neither of these studies mention any effect model
errors can have on sensor clusterization. This study is mostly
theoretical and aims to fill the gap of understanding sensor
clusterization.


\subsection{Contribution}
I propose and study a relaxed and analytically tractable model for
D-optimal designs. Under this model, D-optimal designs are solutions
of a constrained optimization problem, formulated using Lagrange
multipliers. This allows me to rigorously show how model error
mitigates clusterization (section \ref{section:non vanishing}). I show
that designs that exhibit clusterization are indeed optimal.

%% A beautiful mathematical structure arises in D-optimal designs

When no model error is present, the Lagrange multipliers problem boils
down to a nonlinear eigenvalue problem.
%% The operator for which eigenvectors and eigenvalues are sought is a
%% sum of two operators. The first is the prior covariance. The second
%% is an outer product of the observations (see section
%% \ref{section:vanishing} for details and exact statement).
This structure helps me characterize D-optimal designs in Theorem
\ref{thm:char}. The key idea is that a D-optimal design reduces
uncertainties where they are highest first.


\begin{restatable}[D-optimal designs with vanishing model error]{theorem}{main}\label{thm:char}
  Let:
  \begin{enumerate}
  \item $\fwd:\hilp \to \hilo$ a linear forward operator,
    $\prcov:\hilp \to \hilp$ prior covariance operator, $\obs: \hilo
    \to \mathbb{R}^m$ observation operator, where $m \in \mathbb{N}$
    is the number of observations taken.
    \item $\sigma^2 \in \mathbb{R}_{+}$ observation noise variance,
      $\data = \obs \fwd \param + \eps$, where $\eps \in \mathbb{R}^m$
      is iid $\mathcal{N}(0, \sigma^2)$ noise, $\pr \sim
      \mathcal{N}(0, \prcov)$ prior measure, $\post$ the posterior
      measure.
  \item A D-optimality utility function
    \cite{AlexanderianGloorGhattas14}:
    \begin{align*}
      \begin{split}
        \tar(\obs) :&= \mathbb{E}_{\data}\left [ D_{\text{KL}} (\post || \pr ) \right ] \\
        % 
        % 
        % 
        &= \frac12 \log \det ( I + \sigma^{-2} \prcov^{1/2} \fwd ^*
        \obs^* \obs \fwd \prcov^{1/2}).
    \end{split}
  \end{align*}
  \item A D-optimal design operator $\obs$:
    $$
    \obs = \argmax_{\|\meas_j\| = 1, j=1,\dots,m}\tar(\obs)
    $$ 
  \item $\{\lambda_i\}_{i=1}^\infty$ eigenvalues of $\fwd\prcov\fwd^*$
    in decreasing order of magnitude.
  %% \item $\{\ev_i\}_{i=1}^\infty$ their corresponding eigenvectors.
  \item $\{\eta_i\}_{i=1}^\infty$ eigenvalues of $\obs^*\obs$.
  \end{enumerate}

  Then:
  \begin{enumerate}
  \item $\obs^*\obs$ and $\fwd\prcov\fwd^*$ are simultaneously
    diagonalizable.
  \item $k := \rank \obs^*\obs \leq m$.
  \item     
    \begin{equation*}
      \tar(\obs) = \frac12 \sum_{i=1}^{k} \log (1 + \sigma^{-2}\lambda_i\eta_i). %= \frac12 \sum_{i=1}^{m} \log (1 + \sigma^{-2}\lambda_i\eta_i).
    \end{equation*}
  %% \item 
  %%   \begin{equation*}
  %%     k = \argmax \left \{ k:\lambda_k^{-1} < \sigma^{-2}\frac{m}{k} + \frac{1}{k} \sum_{j=1}^{k}
  %%     \lambda_j^{-1} \right \}.
  %%   \end{equation*}
  \item
    \begin{equation*}
        \eta_i = \begin{cases}
          \frac{m}{k} - \sigma^2 \lambda_i^{-1} + \sigma^2 \frac{1}{k} \sum_{j=1}^k \lambda_j^{-1} & 1 \leq i \leq k \\
          0 & i > k 
        \end{cases}.
    \end{equation*}
  \end{enumerate}
\end{restatable}

In the process, I generalize several lemmas from linear algebra to
infinite-dimensional settings. I prove a Matrix Determinant Lemma in
\ref{lemma:MDL}. I generalize a lemma due to Lax \cite{Lax07} for
calculating $\frac{\der}{\der t} \log \det (I + X(t))$, for an
operator valued function $X(t)$ in \ref{lemma:lax}. I show how to
decompose a symmetric positive definite matrix $m$ as $M = AA^t$,
where $A$ has unit norm columns in Lemma \ref{lemma:free}.

\subsection{Limitations}\label{subsec:limitations}
Two main drawbacks of the study presented here are: First, the relaxed
model does not consider any specific set of allowed
observations. Rather, I take observations in the unit ball in some
Hilbert space. This allows considerably less restrictive observations
than any real-life problem does. The second drawback is that I do not
show rigorously that clusterization necessarily occurs, I only show
that it is as reasonable as no clusterization.


\subsection{An Example of Clusterization}\label{subsec:example}
\begin{figure}
  \begin{tikzpicture}[thick, scale=1.3, every node/.style={scale=0.99}]
    \begin{axis}
      [
      title={Posterior Pointwise Standard Deviations and D-Optimal Sensor Locations},  
      xmin = 0,
      xmax = 3.14,
      xlabel = {$x$},
      ylabel = posterior std,
      ymin   = 0,
      %compat = 1.3,
      % ymax   = 130,
      % ytick = \empty,
      legend cell align=left,
      % legend style={at={(0.45,0.2)}}
      legend pos= outer north east 
      ]
      % \draw[black!30!white, thin] (50,0) -- (50,130);
      % 
      %% \addplot [thin, black, mark=none] table{stdv-heat-sens1-var1.txt};
      %% \addlegendentry{1 sensors};
      
      %% \addplot [thin, blue, mark=none] table{stdv-heat-sens2-var1.txt};
      %% \addlegendentry{2 sensors};
      
      %% \addplot [thin, red, mark=none] table{stdv-heat-sens3-var1.txt};
      %% \addlegendentry{3 sensors};
      
      \addplot [thin, green, mark=none] table{stdv-heat-sens4-var1.txt};
      \addlegendentry{4 sensors};
      
      \addplot [thin, purple, mark=none] table{stdv-heat-sens5-var1.txt};
      \addlegendentry{5 sensors};
      
      \addplot [thin, cyan, mark=none] table{stdv-heat-sens6-var1.txt};
      \addlegendentry{6 sensors};

  
      %% \addplot [black,  only marks, mark=x, mark size=1.5] 
      %% table{locs-heat-sens1-var1.txt}; 
      %% \addplot [blue,   only marks, mark=x, mark size=1.5]
      %% table{locs-heat-sens2-var1.txt}; 
      %% \addplot [red,    only marks, mark=x, mark size=1.5]
      %% table{locs-heat-sens3-var1.txt};
      \addplot [green,  only marks, mark=*, mark size=1.5] 
      table{locs-heat-sens4-var1.txt}; 
      \addplot [purple, only marks, mark=*, mark size=1.5] 
      table{locs-heat-sens5-var1.txt}; 
      \addplot [cyan,   only marks, mark=*, mark size=1.5] 
      table{locs-heat-sens6-var1.txt}; 
  
      
    \end{axis}
  \end{tikzpicture}
  \caption{The clusterization effect for the 1D heat equation
    described in section \ref{subsec:example}. Posterior pointwise
    standard deviations (lines) are plotted over the domain $[0,
      \pi]$, for varying numbers of sensors. Sensor locations
    (circles) were chosen to minimize (an expression analogous to) the
    determinant of the posterior covariance. The clusterization effect
    can be clearly seen for six sensors. Only four observation
    locations are used --- two pairs of sensors are so close they are
    indistinguishable.}
  \label{fig:clusterization illustration}
\end{figure}

In section \ref{section:prelim} I present a more abstract and general
formulation of the inverse problem I consider. But, for the purpose of
illustration, I present clusterization via a toy model --- the 1D heat
equation in $[0,\pi]$ with a homogeneous Dirichlet boundary condition.

The 1D heat equation is:
\begin{subequations}\label{eq:heat equation}
  \begin{alignat}{2}
    u_t &= \Delta u &&\qquad \text{in } [0,\pi] \times [0,\infty),\\
      u &= 0 &&\qquad \text{on } \{0, \pi\} \times [0,\infty),\\
        u &= u_0 &&\qquad \text{on }[0,\pi] \times \{0\}.
  \end{alignat}
\end{subequations}

The goal is to infer the initial condition $u_0$. For that purpose, we
measure $u$ at some set of locations $\x_j \in [0,\pi], j=1, \dots,m$
and a final time $T > 0$. We assume centered Gaussian observation
error, so we observe $v(\x_j,T) = u(\x_j,T) + \eps(\x_j)$ with
$\eps(\x_j) \sim \normal(0, \sigma^2)$ iid, $\sigma^2 > 0$. We model
the initial condition as $u_0 \sim \normal(0,\prcov)$, for $\prcov =
(-\Delta)^{-1}$ with a homogeneous Dirichlet boundary condition. It is
known \cite{Tarantola05} that for linear problems, with Gaussian prior
and error, the posterior is also Gaussian with a covariance that does
not depend on the observed data. The posterior covariance $\postcov$
has a closed form formula, even in infinite dimensions
\cite{Stuart10}. We denote by $\fwd$ the dynamics operator, so that
$u( \cdot,T) = \fwd u_0$, and the observation operator $\obs$ so that
$u(\x_j,T) = (\obs u)_j, j=1,\dots,m$. The posterior covariance is
known and depends only on $\prcov, \fwd, \obs$ and $\sigma^2$ (see
section \ref{section:prelim} and \eqref{eq:postcov} specifically).

We consider generalization of the information-theoretic design
criterion presented in the introduction to infinite dimensions
(section \ref{subsec:D optimal design} below). We choose
$\x_j,j=1,\dots,m$ to minimize (an expression analogous to) the
determinant of the posterior covariance operator. We will see later
how this corresponds to maximizing expected information gain.

The clusterization effect is illustrated in figure
\ref{fig:clusterization illustration}. Posterior pointwise standard
deviations are plotted over the domain $[0, \pi]$. Since the posterior
covariance does not depend on data, the plot has no reference to
actual data observed. The posterior covariance does, however, depend
on location of the observation taken. In figure
\ref{fig:clusterization illustration}, observation locations are
marked by circles. These were chosen to minimize (an expression
analogous to) the determinant of the posterior covariance. The
clusterization effect can be clearly seen for six sensors. It looks
like only four observations were taken. The reason is that two pairs
of sensors are so close they are indistinguishable.


