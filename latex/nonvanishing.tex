%%%%%%%%%%%%%%%%%%%%%%%%%%%%%%%%%%%%%%%%%%%%%%%%%%%%%%%%%%%%%%%%
%% SECTION Analysis of Optimal Designs --- Non-Vanishing Model Error
%%%%%%%%%%%%%%%%%%%%%%%%%%%%%%%%%%%%%%%%%%%%%%%%%%%%%%%%%%%%%%%%
\section{Sensor Clusterization --- Non-Vanishing Model Error}\label{section:non vanishing}
In this section we show the effect model error has on the
clusterization phenomenon. We will see that if $\modcov \neq 0$
clusterization will not occur. This is in contrast to the case of
vanishing model error presented in Section \ref{section:vanishing}.

Denote $\obs = (\meas_1,\dots,\meas_m)^t$ and $\obsm :=
(\meas_1,\dots,\meas_{m-1})^t$. Also, denote $\Sigmam := \Sigma
(\obsm)$ and $\postcovm$ the posterior covariance that arises when
$\obsm$ is our observation operator.
\begin{restatable*}{corollary}{samemeas}\label{cor:same meas}
  If $\meas_m = \meas_j$ for some $1 \leq j \leq m-1$, then
  \begin{equation*}
    \tar(\obs) - \tar(\obsm) =
    \log \left ( 1 + \frac{\sigma^2
      \langle \fwd \postcovm \fwd^* \obsm^* \Sigmam^{-1} e_j,
      \obsm^* \Sigmam^{-1}e_j \rangle
    }{
      2 - \sigma^2 e_j^t\Sigmam^{-1}e_j 
    }       
    \right ),
  \end{equation*}
  where $e_j\in \mathbb{R}^{m-1}$ is the $j^{\text{th}}$ standard unit
  vector.
\end{restatable*}
We delegate the proof of Corollary \ref{cor:same meas} to the
appendix. Recall from \eqref{eq:Sigma} that $\Sigma(\obs) = \obs
\modcov \obs^* + \sigma^2I$ and let $u := \obsm (\obsm \modcov \obsm^*
+ \sigma^2 I)^{-1}e_j$. Then

\begin{align*}
  \begin{split}
    \lim_{\sigma^2 \to 0} u &= \obsm (\obsm \modcov \obsm^*)^{-1}e_j\\
    \lim_{\sigma^2 \to 0} \postcovm &= (\prcov^{-1} + \fwd^* \obsm^* \obsm \modcov \obsm^* \obsm \fwd)^{-1} \text{ (From \eqref{eq:postcov})}.
  \end{split}
\end{align*}

Consequently, 
\begin{equation*}
   \langle \fwd \postcovm \fwd^* \obsm^* \Sigmam^{-1}
    e_j, \obsm^* \Sigmam^{-1}e_j \rangle 
  %
  %
  = \langle \fwd \postcovm \fwd^* u, u \rangle
\end{equation*}

is bounded, and



\begin{equation*}
\lim_{\sigma^2 \to 0} \tar(\obs) -\tar(\obsm) = 0.
\end{equation*}

Hence no increase in $\tar$ is achieved by taking a repeated
observation (in the limit $\sigma^2 \to 0$). This is clearly
sub-optimal and $\obs$ cannot give rise to a D-optimal design. Thus,
for small observation error levels, the clusterization effect is
mitigated by the presence of a non-zero model error. Since the design
criterion is not defined for $\sigma^2 = 0$ and identical
observations, we cannot make a statement regarding $\sigma^2 = 0$,
except in the limiting sense described above.
