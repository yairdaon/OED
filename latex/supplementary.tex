\documentclass{article}
\usepackage{amsmath}
\usepackage{amssymb}
\usepackage{amsfonts}
\usepackage{amsthm}
%% \usepackage{amsaddr}


\usepackage{thm-restate}
\usepackage{tikz,pgfplotstable}
\usetikzlibrary{patterns}
\pgfplotsset{compat=1.9} % set to 1.8 to get old behaviour
%\usepackage{hyperref}
\usepackage{xcolor}

\newcommand{\R}{\ensuremath{\mathbb{R}}}
\newcommand{\eps}{\varepsilon}
\newcommand{\M}{\mathcal{M}}

\newcommand{\der}{\text{\textup{d}}}
\newcommand{\diag}{\textup{diag}}

\providecommand{\x}{}
\renewcommand{\x}{\mathbf{x}}
\newcommand{\y}{\mathbf{y}}

\newcommand{\hil}{\mathcal{H}}
\newcommand{\hilp}{\mathcal{H}_p}
\newcommand{\hilo}{\mathcal{H}_o}
\newcommand{\obs}{\mathcal{O}}
\newcommand{\pobs}{\mathcal{P}}
\newcommand{\fwd}{\mathcal{F}}

\newcommand{\obsm}{\widehat{\obs}}
\newcommand{\Sigmam}{\widehat{\Sigma}}
\newcommand{\postcovm}{\widehat{\Gamma_{\textup{post}}}}
\newcommand{\uu}{\mathbf{u}}
\newcommand{\tar}{\Psi}
\DeclareMathOperator*{\argmin}{arg\,min}
\DeclareMathOperator*{\argmax}{arg\,max}
\DeclareMathOperator*{\conv}{conv}


% Definitions for second chapter
\newcommand{\data}{\mathbf{d}}
\newcommand{\param}{\mathbf{m}}
\newcommand{\normal}{\mathcal{N}}
\newcommand{\pr}{\mu_{\textup{pr}}} %Prior measure
%\newcommand{\post}{\mu_{\textup{post}}^{\data, \obs}} % Posterior measure
\newcommand{\post}{\mu_{\textup{post}}} % Posterior measure for slides
\newcommand{\prmean}{\param_{\textup{pr}}} % Prior mean
\newcommand{\postmean}{\param_{\textup{post}}} % Posterior mean
\newcommand{\postcov}{\Gamma_{\textup{post}}} % Posterior covariance
\newcommand{\prcov}{\Gamma_{\textup{pr}}} % Prior covariance
\newcommand{\modcov}{\Gamma_{\textup{model}}} % Model covariance
\newcommand{\tmp}{\mathcal{G}}
\newcommand{\meas}{\mathbf{o}}
\newcommand{\ev}{\mathbf{e}} % eigenvector 
\newcommand{\func}{\mathbf{a}}
\newcommand{\tr}[1]{\textup{tr}\left \{#1 \right \} }
\newcommand{\ttr}[1]{\textup{tr}\ #1}
\newcommand{\rank}{\textup{rank}\ }
\newcommand{\des}{\eta} % vector of design parameters
\newcommand{\sigsqr}{\sigma^2}

%% \newcommand{\opt}{\mathfrak{D}}
\newcommand{\opt}{\mathcal{D}}
\newcommand{\postopt}{\mu_{\textup{post}}^{\data, \opt}} % Posterior measure

%% %\usepackage{comment}% http://ctan.org/pkg/comment
%% %% %% %\excludecomment{proof}
%% %% \excludecomment{figure}
%% %% \let\endfigure\relax


%% %\overfullrule=0pt

\theoremstyle{plain}
\newtheorem{theorem}{Theorem}
\newtheorem{corollary}[theorem]{Corollary}
\newtheorem{lemma}[theorem]{Lemma}
\newtheorem{proposition}[theorem]{Proposition}
\theoremstyle{definition}
\newtheorem{definition}[theorem]{Definition}
\newtheorem{example}[theorem]{Example}
\newtheorem{conjecture}[theorem]{Conjecture}
\theoremstyle{remark}
\newtheorem{remark}[theorem]{Remark}

\usepackage{amsthm}
\usepackage{authblk}
\newtheorem{lemma}{Lemma}
%% \externaldocument[][nocite]{ms}

\title{Supplementary Material: Measurement Clusterization in D-optimal
  Designs for Bayesian Linear Inverse Problems over Hilbert Spaces}

\author{Yair Daon\thanks{yair.daon@gmail.com}}
\affil{Azrieli Faculty of Medicine, Bar Ilan University, Safed,
  Israel.}

\begin{document}
\maketitle


\section{The inverse problem of the 1D heat equation}\label{section:example}
%% The mathematical foundations of inversion in function spaces can be
%% found in \cite{Stuart10}.

\section{Generalizations of a known lemma}

\begin{lemma}[Lax, Chapter 9, Theorem 4\cite{Lax07}]\label{lemma:lax}
  Let $Y(t)$ be a differentiable operator-valued function. Assume 
  $I+Y(t)$ is invertible, $Y(t)$ self-adjoint and trace-class. Then
  \begin{equation*}
    \frac{\der \log \det (I+Y(t))}{\der t} = \tr{(I+Y(t))^{-1} \dot{Y}(t)}.
  \end{equation*}
\end{lemma}

\begin{proof}
  Consider a differentiable operator-valued function $X(t)$ such that
  $X(0) = 0$ and $X(t)$ is positive, self-adjoint and trace-class for
  every $t\in \R$. We denote the eigenvalues of this operator by
  $\lambda_k(X(t))$ and sometimes drop the dependence on $X(t)$, so
  $\lambda_k = \lambda_k(X(t))$.  Then $\det (I+X(t)) =
  \prod_{k=1}^{\infty} (1+\lambda_k) < \infty$ and this is finite
  \cite{AlexanderianGloorGhattas14}. The full derivative
  is \begin{align*} \frac{\der \det (I+X(t))}{\der t}
    % 
    % 
    % 
    &= \sum_{k=1}^{\infty} 
    \frac{\partial \det (I+X(s))}{\partial (1+\lambda_k)}\Big |_{s=t}
    \frac{\der (1+\lambda_k)}{\der t} \\
    % 
    %
    %
    &= \sum_{k=1}^{\infty} \frac{\partial \prod_{l=1}^{\infty}
      (1+\lambda_l(s))}{\partial (1+\lambda_k)}\Big |_{s=t}
    \frac{\der (1+\lambda_k)}{\der t} \\
    %
    %
    %
    &= \sum_{k=1}^{\infty} \prod_{l=1, l\neq k}^{\infty}
      (1+\lambda_l(s)) \frac{\partial (1+\lambda_k(s))}{\partial (1+\lambda_k)}\Big |_{s=t}
    \frac{\der (1+\lambda_k)}{\der t} \\
    %
    %
    %    
    &= \sum_{k=1}^{\infty} \frac{\prod_{l=1}^{\infty}
      (1+\lambda_l(s))}{(1+\lambda_k)}\Big |_{s=t}
    \dot{\lambda_k}(X(t)) \\
    % 
    % 
    % 
    &= \sum_{k=1}^{\infty} \frac{\det (I+X(t))}{1 +\lambda_k} \dot{\lambda_k}(X(t)).
  \end{align*}
  The assumption $X(0) = 0$ means $\lambda_k(X(0)) = 0,\ \forall k \geq 1$. Thus:
  \begin{align*}
    \frac{\der (I+\det X(t))}{\der t}\Big |_{t=0} 
    = \sum_{k=1}^{\infty} \dot{\lambda_k}(X(0)) 
    = \frac{\der }{\der t}\tr{X(0)}
    = \tr{\dot{X}(0)},
  \end{align*}
  where the second equality follows by monotone convergence. 
  Let $Y(t)$ a trace-class self-adjoint operator such that 
  $I+Y(t)$ is invertible.
  Define $X(t)$ via $I+X(t) = (I+Y(0))^{-1/2} (I+Y(t)) (I+Y(0))^{-1/2}$. 
  We show $X(t)$ satisfies the conditions above. It is trace-class:
  \begin{align*}
    \tr{X(t)} = \tr{(I+Y(0))^{-1} (I+Y(t)) - I}
    \leq \tr{I+Y(t) - I}< \infty,
  \end{align*}
  since $Y(t)$ is trace-class. It is also clear that
  $X(0) = 0$ and $X(t)$ is self-adjoint.
  $I+Y(t) = (I+Y(0))^{1/2}(I+X(t))(I+Y(0))^{1/2}$, so
  \begin{align*}
    \frac{\der \det (I+Y(t))}{\der t}|_{t=0} 
    &= \det (I+Y(0))\frac{\der \det (I+X(t))}{\der t}\Big |_{t=0} \\
    % 
    % 
    % 
    &= \det (I+Y(0)) \tr{\dot{X}(0)} \\
    % 
    % 
    % 
    &= \det (I+Y(0)) \tr{(I+Y(0))^{-1} \dot{Y}(0)}.
  \end{align*}
  Consequently, by the one-variable chain rule:
  \begin{align*}
    \frac{\der \log \det (I+Y(t))}{\der t}\Big |_{t=0} &=
    % 
    % 
    % 
    \frac{1}{\det (I+Y(0))}\frac{\der \det (I+Y(t))}{\der t}\Big |_{t=0} \\ 
    % 
    % 
    % 
    &= \tr{ (I+Y(t))^{-1} \dot{Y}(t)} \big |_{t=0}.
  \end{align*}
  There is nothing special about $t_0 = 0$ --- we could have chosen
  any other $t_0$ instead. Thus, the relation holds for all $t$.
\end{proof}

The following is a generalization of the Matrix Determinant Lemma to
Hilbert spaces.
\begin{lemma}%\label{lemma:MDL}
  Let $\hil$ a separable Hilbert space, $u,v\in \hil$ and $A: \hil \to
  \hil$ an invertible linear operator such that $\tr{A-I} <
  \infty$. Then $\det A$ and $\det A + uv^*$ are well defined and
  \begin{equation*}
    \det (A + uv^*) = (1 + \langle A^{-1} u, v \rangle ) \det A,
  \end{equation*}
  where $(A + uv^*)w := Aw + \langle v,w \rangle u$.
\end{lemma}
\begin{proof}
  In this proof we rely on common definitions and results
  \cite{simon1977}. First, consider $B := I + xy^*$ for some $x,y \in
  \hil$. We construct an eigenbasis for $B$ and use that to show $\det
  B = 1 + \langle x, y \rangle$. First let $x_1 := x$.  Now, if $x
  \parallel y$, take $\{x_n \}_{n=2}^{\infty}$ an orthogonal basis for
  $span\{x_1\} ^{\perp}$. If, on the other hand, $x \not \parallel y$,
  let
  \begin{equation*}
    x_2 := x - \frac{ \langle x, y\rangle}{\|y\|^2}y
  \end{equation*}
  and it is easy to verify that $x_2 \perp y$ and $span \{x,y\} = span
  \{x_1,x_2\}$. Take $\{x_n \}_{n=3}^{\infty}$ an orthogonal basis for
  $span\{x_1,x_2\} ^{\perp}$. In both cases,
  \begin{equation*}
    B x_n =
    \begin{cases}
      (1 + \langle x, y \rangle) x_n & n = 1 \\
      x_n                            & n \neq 1,
    \end{cases}
  \end{equation*}
  and so $\det B = 1 + \langle x, y \rangle$.
  
  It is easy to verify that $uv^*$ is trace-class and since $\tr{A-I}
  < \infty$, also $\tr{A + uv^* - I} < \infty$ (sum of two trace-class
  operators is trace-class). Thus $\det A$ and $\det (A+uv^*)$ are
  well defined. Let $x:=A^{-1}u$ and $y := v$:
  \begin{equation*}
    \det (A + uv^*) = \det A \ \det(I+A^{-1}uv^*) =
    (1 + \langle A^{-1}u, v \rangle) \det A .
  \end{equation*}
\end{proof}


%\bibliographystyle{siamplain}
\bibliographystyle{amsplain}
\bibliography{/home/yair/projects/bibtex.bib}

\end{document}
