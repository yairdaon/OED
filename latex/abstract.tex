Estimation of parameters in physical processes often demands costly measurements, prompting the pursuit of an optimal measurement strategy. Finding such strategy is termed the problem of \emph{optimal experimental design}, abbreviated as optimal design. Remarkably, optimal designs can yield tightly clustered measurement locations, leading researchers to fundamentally revise the design problem just to circumvent this issue. Some authors introduce error correlation among error terms that are initially independent, while others restrict measurement locations to a finite set of locations. While both approaches may prevent clusterization, they also fundamentally alter the optimal design problem.

In this study, we consider Bayesian D-optimal designs, i.e.~designs that maximize the expected Kullback-Leibler divergence between posterior and prior. We propose an analytically tractable model for D-optimal designs over Hilbert spaces. In this framework, we make several key contributions: \textbf{(a)} We establish that measurement clusterization is a generic trait of D-optimal designs with independent Gaussian measurement errors, and prove that introducing correlations among measurement error terms mitigates clusterization. \textbf{(b)} We characterize D-optimal designs as reducing uncertainty across a subset of prior covariance eigenvectors. Finally, \textbf{(c)} We leverage this characterization to argue that measurement clusterization arises as a consequence of the pigeonhole principle: when more measurements are taken than there are locations where the select eigenvectors are large and others are small --- clusterization occurs. 
%
%% In summary, in this study we shed light on an often ignored issue with
%% Bayesian D-optimal designs. We characterize such designs and show why
%% clusterization arisesit arisestheir inherent properties, propose
%% strategies to address clusterization, and provide insights into the
%% fundamental attributes underlying measurement design optimization.
