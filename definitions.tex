
%%
%% Place here your \usepackage's. Some recommended packages are already included.
%%
\usepackage{tikz,pgfplots,pgfplotstable}
\pgfplotsset{compat=1.9} % set to 1.8 to get old behaviour

% Graphics:
%% \usepackage[final]{graphicx}
\usepackage{graphicx} % use this line instead of the above to suppress graphics in draft copies
\usepackage{wrapfig}
%\usepackage{graphpap} % \defines the \graphpaper command
\usepackage[T1]{fontenc} % Use 8-bit encoding that has 256 glyphs
\usepackage[english]{babel} % English language/hyphenation
%\usepackage{amsmath, amsfonts, amsthm} % Math package

\usepackage{cite}
% makes color citations
\usepackage[colorlinks=true,urlcolor=blue,citecolor=red,linkcolor=red,bookmarks=true]{hyperref}
\usepackage{color}
\usepackage{pgfplots}
\usepackage{tikz}
\usepackage{paralist}
%% \usepackage{graphics} %% add this and next lines if pictures should be in esp format
%% \usepackage{epsfig} %For pictures: screened artwork should be set up with an 85 or 100 line screen
\usepackage{epstopdf} 

%% From slides file
\usepackage{booktabs} % Allows the use of \toprule, \midrule and \bottomrule in tables
\usepackage[normalem]{ulem}
\usepackage{xcolor}
%\usepackage{algorithm}
%\usepackage[noend]{algpseudocode}
\usepackage{proof-at-the-end}

%% % Indent first line of each section:
%% \usepackage{indentfirst}

%% % Good AMS stuff:
%% \usepackage{amsthm}
%% \usepackage[tbtags]{amsmath} % a lot of good stuff!
\usepackage{amsaddr}


%% % Fonts and symbols:
\usepackage{amsfonts}
\usepackage{amssymb}

% Formatting tools:
%\usepackage{relsize} % relative font size selection, provides commands \textsmalle, \textlarger
%\usepackage{xspace} % gentle spacing in macros, such as \newcommand{\acims}{\textsc{acim}s\xspace}

% Page formatting utility:
\usepackage{geometry}
%\usepackage[margin=0.1in]{geometry}

%%
%% Place here your \newcommand's and \renewcommand's. Some examples already included.
%%
\renewcommand{\le}{\leqslant}
\renewcommand{\ge}{\geqslant}
\renewcommand{\emptyset}{\ensuremath{\varnothing}}
\newcommand{\ds}{\displaystyle}
\newcommand{\R}{\ensuremath{\mathbb{R}}}
\newcommand{\Q}{\ensuremath{\mathbb{Q}}}
\newcommand{\Z}{\ensuremath{\mathbb{Z}}}
\newcommand{\N}{\ensuremath{\mathbb{N}}}
\newcommand{\T}{\ensuremath{\mathbb{T}}}
\newcommand{\B}{\mathcal{B}}
\newcommand{\eps}{\varepsilon}
\newcommand{\closure}[1]{\ensuremath{\overline{#1}}}
\newcommand{\M}{\mathcal{M}}
\newtheorem{theorem}{Theorem}[section]
\newtheorem{corollary}[theorem]{Corollary}
\newtheorem*{main}{Main Theorem}
\newtheorem{lemma}[theorem]{Lemma}
\newtheorem{proposition}[theorem]{Proposition}
\newtheorem{conjecture}{Conjecture}
\newtheorem*{problem}{Problem}
\theoremstyle{definition}
\newtheorem{definition}[theorem]{Definition}
\newtheorem{remark}{Remark}
\newtheorem*{notation}{Notation}
\newcommand{\ep}{\varepsilon}
%\newcommand{\eps}[1]{{#1}_{\varepsilon}}
\newcommand{\bs}{\boldsymbol}
\newcommand{\der}{\text{\textup{d}}}
\newcommand{\coder}[1]{\texttt{#1}}
\newcommand{\inner}[2]{#1 \cdot #2}
\newcommand{\var}{\textup{Var}}
\newcommand{\corr}{\textup{Corr}}
\newcommand{\cov}{\textup{Cov}}
\newcommand{\diag}{\textup{diag}}
\newcommand{\dom}{\mathcal{D}om}
\newcommand{\gsnote}[1]{{\textcolor{blue}{#1}}}
\newcommand{\ydnote}[1]{{\textcolor{cyan}{Answer: #1}}}
%\newcommand{\reftwo}[1]{{\textcolor{green}{#1}}}
%\newcommand{\refthree}[1]{{\textcolor{red}{#1}}}
\newcommand{\refthree}[1]{{\color{red} #1}}
\newcommand{\reftwo}[1]{{\color{green} #1}}



\newcommand{\precop}{\mathcal{L}}
\newcommand{\precmat}{\mathbf{L}}
\newcommand{\Op}{\mathcal{A}}
\newcommand{\OpDir}{\mathcal{A}_{\textup{Dirichlet}}}
\newcommand{\OpNeu}{\mathcal{A}_{\textup{Neumann}}}
\newcommand{\covop}{\mathcal{C}}
\newcommand{\lag}{\mathcal{L}}
\newcommand{\n}{\boldsymbol{n} }
\newcommand{\dn}{\partial \n }
\newcommand{\x}{{\boldsymbol{x}}}
\newcommand{\y}{{\boldsymbol{y}}}
\newcommand{\z}{{\boldsymbol{z}}}
\newcommand{\kxmy}{ \kappa \| \x - \y \| }
\newcommand{\xmy}{ \| \x - \y \| }
\newcommand{\proj}{\mathcal{P}}
\newcommand{\kr}{\kappa r}
\newcommand{\ivar}{\text{IntVar}}

\newcommand{\K}{\mathbb{K}}
\newcommand{\hil}{\mathcal{H}}
\newcommand{\ban}{\mathcal{B}}
\newcommand{\obs}{\mathcal{O}}
\newcommand{\fwd}{\mathcal{F}}
\newcommand{\tru}{\fwd_{\textup{True}}}
\newcommand{\err}{\fwd_{\textup{Error}}}

\newcommand{\obsm}{\widehat{\obs}}
\newcommand{\Sigmam}{\widehat{\Sigma}}
\newcommand{\postcovm}{\widehat{\postcov}}
\newcommand{\uu}{\mathbf{u}}
\newcommand{\tar}{\Psi}
\DeclareMathOperator*{\argmin}{arg\,min}
\DeclareMathOperator*{\argmax}{arg\,max}

% Definitions for second chapter
\newcommand{\data}{\mathbf{d}}
\newcommand{\param}{\mathbf{m}}
\newcommand{\sspar}{\param_{\textup{SmallScale}}}
\newcommand{\normal}{\mathcal{N}}
\newcommand{\pr}{\mu_{\textup{pr}}} %Prior measure
\newcommand{\post}{\mu_{\textup{post}}} % Posterior measure
\newcommand{\postmean}{\param_{\textup{post}}} % Posterior mean
\newcommand{\postcov}{\Gamma_{\textup{post}}} % Posterior covariance
\newcommand{\prcov}{\Gamma_{\textup{pr}}} % Prior covariance
\newcommand{\modcov}{\Gamma_{\textup{model}}} % Model covariance
\newcommand{\tmp}{\mathcal{G}}
\newcommand{\meas}{\mathbf{o}}
\newcommand{\ev}{\mathbf{e}} % eigenvector 
\newcommand{\func}{\mathbf{a}}
\newcommand{\tr}[1]{\textup{tr}\left \{#1 \right \} }
\newcommand{\ttr}[1]{\textup{tr}\ #1}
\newcommand{\rank}{\textup{rank}\ }
\newcommand{\des}{\eta} % vector of design parameters
\newcommand{\sigsqr}{\sigma^2}
%\newcommand{\acim}{\textsc{acim}\xspace}
%\newcommand{\acims}{\textsc{acim}s\xspace}


\newcommand\numberthis{\addtocounter{equation}{1}\tag{\theequation}}
%%
%% Place here your \newtheorem's:
%%

%% Some examples commented out below. Create your own or use these...
%%%%%%%%%\swapnumbers % this makes the numbers appear before the statement name.
%\theoremstyle{plain}
%\newtheorem{thm}{Theorem}[chapter]
%\newtheorem{prop}[thm]{Proposition}
%\newtheorem{lemma}[thm]{Lemma}
%\newtheorem{cor}[thm]{Corollary}
\newtheorem{observation}[theorem]{Observation}

%\theoremstyle{definition}
%\newtheorem{define}{Definition}[chapter]

%\theoremstyle{remark}
%\newtheorem*{rmk*}{Remark}
%\newtheorem*{rmks*}{Remarks}

%% This defines the "proo" environment, which is the same as proof, but
%% with "Proof:" instead of "Proof.". I prefer the former.
%\newenvironment{proo}{\begin{proof}[Proof:]}{\end{proof}}

%% Hack
\newcommand{\stkout}[1]{\ifmmode\text{\sout{\ensuremath{#1}}}\else\sout{#1}\fi}
\usepackage{cancel}
