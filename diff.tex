\documentclass{amsart}
%DIF LATEXDIFF DIFFERENCE FILE
%DIF DEL paper.tex       Wed Oct  7 01:09:38 2020
%DIF ADD new_paper.tex   Thu Oct 22 17:43:11 2020

\usepackage{amsmath}
\usepackage{amssymb}
\usepackage{amsfonts}
\usepackage{amsthm}
%% \usepackage{amsaddr}


\usepackage{thm-restate}
\usepackage{tikz,pgfplotstable}
\usetikzlibrary{patterns}
\pgfplotsset{compat=1.9} % set to 1.8 to get old behaviour
%\usepackage{hyperref}
\usepackage{xcolor}

\newcommand{\R}{\ensuremath{\mathbb{R}}}
\newcommand{\eps}{\varepsilon}
\newcommand{\M}{\mathcal{M}}

\newcommand{\der}{\text{\textup{d}}}
\newcommand{\diag}{\textup{diag}}

\providecommand{\x}{}
\renewcommand{\x}{\mathbf{x}}
\newcommand{\y}{\mathbf{y}}

\newcommand{\hil}{\mathcal{H}}
\newcommand{\hilp}{\mathcal{H}_p}
\newcommand{\hilo}{\mathcal{H}_o}
\newcommand{\obs}{\mathcal{O}}
\newcommand{\pobs}{\mathcal{P}}
\newcommand{\fwd}{\mathcal{F}}

\newcommand{\obsm}{\widehat{\obs}}
\newcommand{\Sigmam}{\widehat{\Sigma}}
\newcommand{\postcovm}{\widehat{\Gamma_{\textup{post}}}}
\newcommand{\uu}{\mathbf{u}}
\newcommand{\tar}{\Psi}
\DeclareMathOperator*{\argmin}{arg\,min}
\DeclareMathOperator*{\argmax}{arg\,max}
\DeclareMathOperator*{\conv}{conv}


% Definitions for second chapter
\newcommand{\data}{\mathbf{d}}
\newcommand{\param}{\mathbf{m}}
\newcommand{\normal}{\mathcal{N}}
\newcommand{\pr}{\mu_{\textup{pr}}} %Prior measure
%\newcommand{\post}{\mu_{\textup{post}}^{\data, \obs}} % Posterior measure
\newcommand{\post}{\mu_{\textup{post}}} % Posterior measure for slides
\newcommand{\prmean}{\param_{\textup{pr}}} % Prior mean
\newcommand{\postmean}{\param_{\textup{post}}} % Posterior mean
\newcommand{\postcov}{\Gamma_{\textup{post}}} % Posterior covariance
\newcommand{\prcov}{\Gamma_{\textup{pr}}} % Prior covariance
\newcommand{\modcov}{\Gamma_{\textup{model}}} % Model covariance
\newcommand{\tmp}{\mathcal{G}}
\newcommand{\meas}{\mathbf{o}}
\newcommand{\ev}{\mathbf{e}} % eigenvector 
\newcommand{\func}{\mathbf{a}}
\newcommand{\tr}[1]{\textup{tr}\left \{#1 \right \} }
\newcommand{\ttr}[1]{\textup{tr}\ #1}
\newcommand{\rank}{\textup{rank}\ }
\newcommand{\des}{\eta} % vector of design parameters
\newcommand{\sigsqr}{\sigma^2}

%% \newcommand{\opt}{\mathfrak{D}}
\newcommand{\opt}{\mathcal{D}}
\newcommand{\postopt}{\mu_{\textup{post}}^{\data, \opt}} % Posterior measure

%% %\usepackage{comment}% http://ctan.org/pkg/comment
%% %% %% %\excludecomment{proof}
%% %% \excludecomment{figure}
%% %% \let\endfigure\relax


%% %\overfullrule=0pt

\theoremstyle{plain}
\newtheorem{theorem}{Theorem}
\newtheorem{corollary}[theorem]{Corollary}
\newtheorem{lemma}[theorem]{Lemma}
\newtheorem{proposition}[theorem]{Proposition}
\theoremstyle{definition}
\newtheorem{definition}[theorem]{Definition}
\newtheorem{example}[theorem]{Example}
\newtheorem{conjecture}[theorem]{Conjecture}
\theoremstyle{remark}
\newtheorem{remark}[theorem]{Remark}


\numberwithin{equation}{section}
%DIF PREAMBLE EXTENSION ADDED BY LATEXDIFF
%DIF UNDERLINE PREAMBLE %DIF PREAMBLE
\RequirePackage[normalem]{ulem} %DIF PREAMBLE
\RequirePackage{color}\definecolor{RED}{rgb}{1,0,0}\definecolor{BLUE}{rgb}{0,0,1} %DIF PREAMBLE
\providecommand{\DIFadd}[1]{{\protect\color{blue}\uwave{#1}}} %DIF PREAMBLE
\providecommand{\DIFdel}[1]{{\protect\color{red}\sout{#1}}}                      %DIF PREAMBLE
%DIF SAFE PREAMBLE %DIF PREAMBLE
\providecommand{\DIFaddbegin}{} %DIF PREAMBLE
\providecommand{\DIFaddend}{} %DIF PREAMBLE
\providecommand{\DIFdelbegin}{} %DIF PREAMBLE
\providecommand{\DIFdelend}{} %DIF PREAMBLE
\providecommand{\DIFmodbegin}{} %DIF PREAMBLE
\providecommand{\DIFmodend}{} %DIF PREAMBLE
%DIF FLOATSAFE PREAMBLE %DIF PREAMBLE
\providecommand{\DIFaddFL}[1]{\DIFadd{#1}} %DIF PREAMBLE
\providecommand{\DIFdelFL}[1]{\DIFdel{#1}} %DIF PREAMBLE
\providecommand{\DIFaddbeginFL}{} %DIF PREAMBLE
\providecommand{\DIFaddendFL}{} %DIF PREAMBLE
\providecommand{\DIFdelbeginFL}{} %DIF PREAMBLE
\providecommand{\DIFdelendFL}{} %DIF PREAMBLE
%DIF LISTINGS PREAMBLE %DIF PREAMBLE
\RequirePackage{listings} %DIF PREAMBLE
\RequirePackage{color} %DIF PREAMBLE
\lstdefinelanguage{DIFcode}{ %DIF PREAMBLE
%DIF DIFCODE_UNDERLINE %DIF PREAMBLE
  moredelim=[il][\color{red}\sout]{\%DIF\ <\ }, %DIF PREAMBLE
  moredelim=[il][\color{blue}\uwave]{\%DIF\ >\ } %DIF PREAMBLE
} %DIF PREAMBLE
\lstdefinestyle{DIFverbatimstyle}{ %DIF PREAMBLE
	language=DIFcode, %DIF PREAMBLE
	basicstyle=\ttfamily, %DIF PREAMBLE
	columns=fullflexible, %DIF PREAMBLE
	keepspaces=true %DIF PREAMBLE
} %DIF PREAMBLE
\lstnewenvironment{DIFverbatim}{\lstset{style=DIFverbatimstyle}}{} %DIF PREAMBLE
\lstnewenvironment{DIFverbatim*}{\lstset{style=DIFverbatimstyle,showspaces=true}}{} %DIF PREAMBLE
%DIF END PREAMBLE EXTENSION ADDED BY LATEXDIFF

\begin{document}

\title[Sensor Clusterization in D-optimal design in infinite
  dimensions]{Sensor Clusterization in D-optimal Design in Infinite
  Dimensional \DIFaddbegin \DIFadd{Linear }\DIFaddend Bayesian Inverse Problems}

\author{Yair Daon}
\address{Porter School of the Environment and Earth
  Sciences, Tel Aviv University\\ Tel Aviv, Israel}
%% \curraddr{Porter School of Environment Studies, Tel Aviv University\\ Tel Aviv, Israel}
%% \curraddr{}
\email{yair.daon@gmail.com}
%% \thanks{}

%    The 2010 edition of the Mathematics Subject Classification is
%    the current definitive version.
\subjclass{Primary: 
  62F15, % Statistics - bayesian inference
  35R30, % PDE - Inverse problems
  Secondary:
  28C20, % Measure and integration - set functions and measures and integrals in
  %infinite-dimensional spaces (Wiener measure, Gaussian measure, etc.
}
%% Primary: 62F15, 35R30; Secondary: 28C20
\date{\today}

\DIFdelbegin %DIFDELCMD < \begin{abstract}
%DIFDELCMD <   We investigate the problem of sensor clusterization in optimal
%DIFDELCMD <   experimental design for infinite-dimensional Bayesian inverse
%DIFDELCMD <   problems. We suggest an analytically tractable model for such
%DIFDELCMD <   designs and reason how it may lead to sensor clusterization in the
%DIFDELCMD <   case of iid measurement noise. We also show that in the case of
%DIFDELCMD <   spatially correlated measurement error clusterization does not
%DIFDELCMD <   occur. As a part of the analysis we prove a matrix determinant
%DIFDELCMD <   lemma analog in infinite dimensions, as well as a lemma for
%DIFDELCMD <   calculating derivatives of $\log \det$ of operators.
%DIFDELCMD < \end{abstract}
%DIFDELCMD < %%%
\DIFdelend \DIFaddbegin \begin{abstract}
  We investigate the problem of sensor clusterization in optimal
  experimental design for infinite-dimensional Bayesian
  linear inverse problems. We suggest an analytically
  tractable model for such designs and reason how it may lead to
  sensor clusterization in the case of iid measurement noise. We also
  show that in the case of spatially correlated measurement error
  clusterization does not occur. As a part of the analysis we prove a
  matrix determinant lemma analog in infinite dimensions, as well as a
  lemma for calculating derivatives of $\log \det$ of operators.
\end{abstract}
\DIFaddend 

\maketitle

\section{Introduction}\label{section:OED intro}
Experimental design is an important part of many scientific
investigations. When considering an inverse problem, one can often
specify sensor locations (e.g.\ in geophysics and oceanography
applications), certain wavelengths (e.g.\ in MRI) or wave reflections
from the ground (e.g.\ searching for oil or using a radar). Whatever
the allowed set of measurements is, one should select the optimal
measurements to take, in order to increase accuracy, reduce costs, or
both.

Designing experiments is usually done by optimizing some \emph{design
  criterion}\DIFdelbegin \DIFdel{and this }\DIFdelend \DIFaddbegin \DIFadd{. This }\DIFaddend is true both for frequentists
\cite{Silvey13,Ucinski05} as well as for Bayesians
\cite{ChalonerVerdinelli95}\DIFdelbegin \DIFdel{, see }\DIFdelend \DIFaddbegin \DIFadd{. See }\DIFaddend \cite{ChalonerVerdinelli95} for an
investigation of the analogy between the two approaches. Although
there is a plethora of design criteria, \DIFdelbegin \DIFdel{in the current work we will
}\DIFdelend \DIFaddbegin \DIFadd{we }\DIFaddend focus on just one of these,
commonly referred to as \emph{D-optimal design}. It has a simple and
appealing motivation in the Bayesian context as explained in
\cite{ChalonerVerdinelli95}: \DIFdelbegin \DIFdel{for a linear
model, a }\DIFdelend \DIFaddbegin \DIFadd{A }\DIFaddend D-optimal design seeks to maximize the
expected information gain (KL divergence
\cite{KullbackLeibler51,CoverThomas12}) between posterior and
prior. \DIFdelbegin \DIFdel{In }\DIFdelend \DIFaddbegin \DIFadd{Consider a linear model in }\DIFaddend finite dimensions, \DIFaddbegin \DIFadd{with Gaussian
prior and noise. In this model, }\DIFaddend maximizing the expected information
gain amounts to minimizing the determinant of the posterior covariance
matrix. In a frequentist \DIFdelbegin \DIFdel{settings, an optimal
}\DIFdelend \DIFaddbegin \DIFadd{setting, a D-optimal }\DIFaddend design minimizes the
volume of the uncertainty ellipsoid \cite[page 16]{Ucinski05}, but
this is done for the Fisher information matrix and not the posterior
covariance. However, \cite{ChalonerVerdinelli95} \DIFdelbegin \DIFdel{shows }\DIFdelend \DIFaddbegin \DIFadd{show }\DIFaddend that the latter
is just a regularized version of the former.

The previous discussion is classical for experimental design when
inference is to take place over a finite (not too large) number of
parameters. The subject of optimal experimental design for function
inference in a Bayesian context was pioneered by
\DIFdelbegin \DIFdel{\mbox{%DIFAUXCMD
\cite{AlexanderianGloorGhattas14,AlexanderianPetraStadlerEtAl16,
  AlexanderianPetraStadlerEtAl14}}\hspace{0pt}%DIFAUXCMD
}\DIFdelend \DIFaddbegin \DIFadd{\mbox{%DIFAUXCMD
\cite{AlexanderianGloorGhattas14, AlexanderianPetraStadlerEtAl16,
  AlexanderianPetraStadlerEtAl14}}\hspace{0pt}%DIFAUXCMD
}\DIFaddend . Similarly to the finite dimensional
case, it can be shown that a D-optimal design arises naturally for
linear models when one wishes to maximize the KL divergence between
posterior and prior\DIFdelbegin \DIFdel{and this }\DIFdelend \DIFaddbegin \DIFadd{. This }\DIFaddend amounts to minimizing the determinant of the
posterior covariance operator (understood as a product of its
eigenvalues). Some difficulties arise in the process, but remedies can
be found as shown in \cite{AlexanderianGloorGhattas14}.

It seems counter intuitive that when one computes an optimal design
using the D-optimal criterion (and others that will not be
investigated here), the optimization process results in measurements
that are very similar. For example, if a measurement is thought of as
measuring some function value at $\x \in \Omega \subseteq \R^d,
d=1,2,3$ (with added error) then the optimization procedure sometimes
places sensors in \DIFaddbegin \DIFadd{very }\DIFaddend close proximity to each other (as can be seen
in figure \ref{fig:clusterization illustration}). Following
\cite{Ucinski05}, we refer to this phenomenon as \emph{sensor
  clusterization}.

\subsection{Related Work}
The phenomenon of sensor clusterization seems to be known in several
different contexts. In a frequentist and finite-dimensional context,
\cite{Fedorov96} and \cite[chapter 2.4.3]{Ucinski05} discuss this
phenomenon and suggest an approach called clusterization-free design\DIFdelbegin \DIFdel{,
where }\DIFdelend \DIFaddbegin \DIFadd{.
In such designs, }\DIFaddend the user enforces measurement locations to be far
from each other. One way to do this is by introducing correlated
errors which, philosophically, accounts for both measurement error and
model error. Another method considered is imposing distance
constraints between measurements. A somewhat different approach is
suggested in \cite[page 49]{Fedorov12}, where close by design
measurements are merged --- a procedure which obviously does not avoid
clusterization. The same problem arises in time-series analysis for
pharmacokinetic experiments. The authors of \cite{Hooker09} suggest
modeling auto-correlation time in the noise model, which is equivalent
to the correlated errors mentioned above.

Any of the above mentioned approaches might serve as a remedy and push
sensors away from each other. Yet, none offers any insight as to why
clusterization occurs. Also, as better models are employed, model
error is decreased and the clusterization phenomenon will eventually
reappear. While these approaches are practical and help us avoid the
problem, they do not provide insight as to why sensors are clustering.

\subsection{Contribution}
We propose and thoroughly study an analytically tractable model for
understanding D-optimal designs. In this model, \DIFdelbegin \DIFdel{an optimal design
arises as a solution to }\DIFdelend \DIFaddbegin \DIFadd{D-optimal design
are solutions of }\DIFaddend a constrained optimization problem, formulated
using Lagrange multipliers (section \ref{section:D and grad}). \DIFdelbegin \DIFdel{In the
case of no model error we show this problem }\DIFdelend \DIFaddbegin \DIFadd{Using
this model we rigorously show how model error mitigates clusterization
(section \ref{section:non vanishing}). We show how clusterization
occurs in a sequential design scenario (section
\ref{subsec:clusterization sequential}). Then, we show that designs
that exhibit clusterization are just as optimal as clusterization free
designs in a simultaneous D-optimal designs (section
\ref{subsec:clusterization simultaneous}).
}

\DIFadd{A beautiful mathematical structure arises in D-optimal designs when no
model error is present. The Lagrange multipliers problems }\DIFaddend is in fact a
\DIFdelbegin \DIFdel{set of eigenvector-eigenvalue problems (equation }\DIFdelend \DIFaddbegin \DIFadd{nonlinear eigenvalue problem for the observations and prior
covariance. The operator for which eigenvectors and eigenvalues are
sought is a sum of two operators. The first is the prior
covariance. The second is an outer product of the observations. See
}\DIFaddend \eqref{eq:eigenproblem} \DIFdelbegin \DIFdel{). We then characterize
an optimal design as one that reduces the highest
uncertainties , and show how that can give rise to repeated
measurements (section \ref{subsec:clusterization} and figure
\ref{fig:clusterization})}\DIFdelend \DIFaddbegin \DIFadd{for details and exact statement. This
structure helps us shed light into D-optimal designs. We characterize
these designs using theorem \ref{thm:char}. One insight we prove in
said theorem, is that a D-optimal design reduces uncertainties where
they are highest first. Other interestig phenomena arise, but they
require setting notation, and are discussed later}\DIFaddend .

In the process we generalize several lemmas from linear algebra to
infinite-dimensional settings\DIFdelbegin \DIFdel{--- the Matrix Determinant Lemma and a lemma for calculating the derivative of determinants}\DIFdelend \DIFaddbegin \DIFadd{. We prove a matrix determinant lemma in
\ref{lemma:MDL}. We generalize a lemma due to Lax \mbox{%DIFAUXCMD
\cite{lax97} }\hspace{0pt}%DIFAUXCMD
for
calculating $\frac{\der}{\der t} \log \det (I + X(t))$, for an
operator valued function $X(t)$ in \ref{lemma:lax}. We also prove
lemmas in linear algebra. One gives a decomposition $M = AA^t$ where
$A$ has unit norm columns \ref{lemma:free}. Another shows simultaneous
diagonizability of the prior and outer product in the nonlinear
eigenvalue problem referenced above \ref{lemma:sim diag}}\DIFaddend . We also
provide other tools for understanding D-optimal designs in
infinite-dimensional Bayesian inverse problems, including lemmas for
calculating the increase in the design criterion per observation ---
Lemmas \ref{lemma:design increase} and Corollary \ref{cor:zero mod
  err}.
%DIF < %  We also show that in our model with independent measurement
%DIF < % noise, an interesting phenomenon arises, in which the eigenvectors
%DIF < % of the measurement operator (denoted below $\obs^*\obs$) are the
%DIF < % same as the eigenvectors of the prior covariance in observation
%DIF < % space, which helps arrive at the results outlined in the previous
%DIF < % paragraph.
\DIFdelbegin \DIFdel{The arguments presented are completely generic and apply equally to
any of the scenarios mentioned in the introduction.
}\DIFdelend 


%DIF < % \subsection{Limitations}\label{subsec:limitations}
%DIF < % There are two main drawbacks of the arguments presented here. First,
%DIF < % one can argue that we allow considerably less restrictive observations
%DIF < % than any real-life problem does. This is because our observation model
%DIF < % allows taking measurements that are arbitrarily weighted linear
%DIF < % combinations of eigenvectors. The second drawback is that we do not
%DIF < % show rigorously that clusterization necessarily occurs in our model,
%DIF < % we only show that it can occur.
\DIFaddbegin \subsection{\DIFadd{Limitations}}\label{subsec:limitations}
\DIFadd{There are two main drawbacks of the study presented here. Our relaxed
model does not consider any specific set of allowed
measurements. Rather, we take measurements in the unit ball in some
Hilbert space. This allows considerably less restrictive observations
than any real-life problem does. The second drawback is that we do not
show rigorously that clusterization necessarily occurs in a
simultaneous design. We only show that it is as reasonable as no
clusterization.
}\DIFaddend 


\subsection{An Example of Clusterization}\label{subsec:example}
\DIFdelbegin %DIFDELCMD < \begin{figure}
%DIFDELCMD <   \begin{tikzpicture}[thick, scale=1.3, every node/.style={scale=0.99}]
%DIFDELCMD <     \begin{axis}
%DIFDELCMD <       [
%DIFDELCMD <       xmin = 0,
%DIFDELCMD <       xmax = 3.14,
%DIFDELCMD <       xlabel = {$x$},
%DIFDELCMD <       ylabel = posterior std,
%DIFDELCMD <       ymin   = 0,
%DIFDELCMD <       %compat = 1.3,
%DIFDELCMD <       % ymax   = 130,
%DIFDELCMD <       % ytick = \empty,
%DIFDELCMD <       legend cell align=left,
%DIFDELCMD <       % legend style={at={(0.45,0.2)}}
%DIFDELCMD <       legend pos= outer north east 
%DIFDELCMD <       ]
%DIFDELCMD <       % \draw[black!30!white, thin] (50,0) -- (50,130);
%DIFDELCMD <       % 
%DIFDELCMD <       %% \addplot [thin, black, mark=none] table{stdv-heat-sens1-var1.txt};
%DIFDELCMD <       %% \addlegendentry{1 sensors};
%DIFDELCMD <       

%DIFDELCMD <       %% \addplot [thin, blue, mark=none] table{stdv-heat-sens2-var1.txt};
%DIFDELCMD <       %% \addlegendentry{2 sensors};
%DIFDELCMD <       

%DIFDELCMD <       %% \addplot [thin, red, mark=none] table{stdv-heat-sens3-var1.txt};
%DIFDELCMD <       %% \addlegendentry{3 sensors};
%DIFDELCMD <       

%DIFDELCMD <       \addplot [thin, green, mark=none] table{stdv-heat-sens4-var1.txt};
%DIFDELCMD <       \addlegendentry{4 sensors};
%DIFDELCMD <       

%DIFDELCMD <       \addplot [thin, purple, mark=none] table{stdv-heat-sens5-var1.txt};
%DIFDELCMD <       \addlegendentry{5 sensors};
%DIFDELCMD <       

%DIFDELCMD <       \addplot [thin, cyan, mark=none] table{stdv-heat-sens6-var1.txt};
%DIFDELCMD <       \addlegendentry{6 sensors};
%DIFDELCMD < 

%DIFDELCMD <   
%DIFDELCMD <       %% \addplot [black,  only marks, mark=x, mark size=1.5] 
%DIFDELCMD <       %% table{locs-heat-sens1-var1.txt}; 
%DIFDELCMD <       %% \addplot [blue,   only marks, mark=x, mark size=1.5]
%DIFDELCMD <       %% table{locs-heat-sens2-var1.txt}; 
%DIFDELCMD <       %% \addplot [red,    only marks, mark=x, mark size=1.5]
%DIFDELCMD <       %% table{locs-heat-sens3-var1.txt};
%DIFDELCMD <       \addplot [green,  only marks, mark=x, mark size=1.5] 
%DIFDELCMD <       table{locs-heat-sens4-var1.txt}; 
%DIFDELCMD <       \addplot [purple, only marks, mark=x, mark size=1.5] 
%DIFDELCMD <       table{locs-heat-sens5-var1.txt}; 
%DIFDELCMD <       \addplot [cyan,   only marks, mark=x, mark size=1.5] 
%DIFDELCMD <       table{locs-heat-sens6-var1.txt}; 
%DIFDELCMD <   

%DIFDELCMD <       
%DIFDELCMD <     \end{axis}
%DIFDELCMD <   \end{tikzpicture}
%DIFDELCMD <   \caption{Shown are pointwise posterior standard deviations for
%DIFDELCMD <     various numbers of sensors, and location of these sensors for the
%DIFDELCMD <     1D heat equation with homogeneous Dirichlet boundary described in
%DIFDELCMD <     section \ref{subsec:example}. Note the clusterization that occurs
%DIFDELCMD <     when 6 sensors are used --- only 4 measurement locations are
%DIFDELCMD <     included. Observed data does not influence the pointwise
%DIFDELCMD <     variances, hence it is omitted.}
%DIFDELCMD < \end{figure}\label{fig:clusterization illustration}
%DIFDELCMD < %%%
\DIFdel{In section \ref{section:abstract OED} }\DIFdelend \DIFaddbegin \begin{figure}
  \begin{tikzpicture}[thick, scale=1.3, every node/.style={scale=0.99}]
    \begin{axis}
      [
      title={Posterior Pointwise Standard Deviations and D-Optimal Sensor Locations},  
      xmin = 0,
      xmax = 3.14,
      xlabel = {$x$},
      ylabel = posterior std,
      ymin   = 0,
      %compat = 1.3,
      % ymax   = 130,
      % ytick = \empty,
      legend cell align=left,
      % legend style={at={(0.45,0.2)}}
      legend pos= outer north east 
      ]
      % \draw[black!30!white, thin] (50,0) -- (50,130);
      % 
      %% \addplot [thin, black, mark=none] table{stdv-heat-sens1-var1.txt};
      %% \addlegendentry{1 sensors};

      %% \addplot [thin, blue, mark=none] table{stdv-heat-sens2-var1.txt};
      %% \addlegendentry{2 sensors};

      %% \addplot [thin, red, mark=none] table{stdv-heat-sens3-var1.txt};
      %% \addlegendentry{3 sensors};

      \addplot [thin, green, mark=none] table{stdv-heat-sens4-var1.txt};
      \addlegendentry{4 sensors};

      \addplot [thin, purple, mark=none] table{stdv-heat-sens5-var1.txt};
      \addlegendentry{5 sensors};

      \addplot [thin, cyan, mark=none] table{stdv-heat-sens6-var1.txt};
      \addlegendentry{6 sensors};

  
      %% \addplot [black,  only marks, mark=x, mark size=1.5] 
      %% table{locs-heat-sens1-var1.txt}; 
      %% \addplot [blue,   only marks, mark=x, mark size=1.5]
      %% table{locs-heat-sens2-var1.txt}; 
      %% \addplot [red,    only marks, mark=x, mark size=1.5]
      %% table{locs-heat-sens3-var1.txt};
      \addplot [green,  only marks, mark=*, mark size=1.5] 
      table{locs-heat-sens4-var1.txt}; 
      \addplot [purple, only marks, mark=*, mark size=1.5] 
      table{locs-heat-sens5-var1.txt}; 
      \addplot [cyan,   only marks, mark=*, mark size=1.5] 
      table{locs-heat-sens6-var1.txt}; 

      
    \end{axis}
  \end{tikzpicture}
  \caption{The clusterization effect for the 1D heat equation
    described in section \ref{subsec:example}. Posterior pointwise
    standard deviations are plotted over the domain $[0, \pi]$, for
    varying numbers of sensors (lines). Sensor locations (circles)
    were chosen to minimize (an expression analogous to) the
    determinant of the posterior covariance. The clusterization effect
    can be clearly seen for six sensors. Only four measurement
    locations are used --- two pairs of sensors are so close they are
    indistinguishable.}
  \label{fig:clusterization illustration}
\end{figure}
\DIFadd{In section \ref{section:prelim} }\DIFaddend we present a more abstract and general
formulation of the problem, but for the purpose of illustration, we
present the problem via a toy model --- the 1D heat equation in
$[0,\pi]$ with homogeneous Dirichlet boundary conditions.

The 1D heat equation is:
\begin{subequations}\label{eq:heat equation}
  \begin{alignat}{2}
    u_t &= \Delta u &&\qquad \text{in } [0,\pi] \times [0,\infty),\\
      u &= 0 &&\qquad \text{on } \{0, \pi\} \times [0,\infty),\\
        u &= u_0 &&\qquad \text{on }[0,\pi] \times \{0\}.
  \end{alignat}
\end{subequations}

We would like to infer the initial condition $u_0$. For that purpose,
we measure $u$ at some set of locations $\x_j \in [0,\pi], j=1, \dots,m$
and a final time $T > 0$. We assume centered Gaussian measurement
error, so we can observe $v(\x_j,T) = u(\x_j,T) + \eps(\x_j)$ with
$\eps(\x_j) \sim \normal(0, \sigma^2), \sigma > 0$ iid. We model the
initial condition as $u_0 \sim \normal(0,\prcov)$, for $\prcov =
(-\Delta)^{-1}$ with homogeneous Dirichlet boundary condition. It is
well known \cite{Tarantola05} that for linear problems, with Gaussian
prior and error, the posterior is also Gaussian with a covariance that
does not depend on the observed data. The posterior covariance
$\postcov$ is known to have a closed form formula, even in infinite
dimensions\cite{Stuart10}. If we denote by $\fwd$ the dynamics
operator, so that $u( \cdot,T) = \fwd u_0$, and the observation
operator $\obs$ so that $u(\x_j,T) = (\obs u)_j, j=1,\dots,m$, then the
posterior covariance is known and depends \DIFaddbegin \DIFadd{only }\DIFaddend on $\prcov, \fwd, \obs$ and
$\sigma^2$ (see section \DIFdelbegin \DIFdel{\ref{section:abstract OED} }\DIFdelend \DIFaddbegin \DIFadd{\ref{section:prelim} }\DIFaddend and
\eqref{eq:postcov} specifically).

We \DIFdelbegin \DIFdel{will }\DIFdelend consider generalization of the information-theoretic design
criterion presented in \DIFdelbegin \DIFdel{section \ref{section:OED intro} }\DIFdelend \DIFaddbegin \DIFadd{the introduction }\DIFaddend to infinite dimensions
(section \ref{subsec:D optimal design} \DIFaddbegin \DIFadd{below}\DIFaddend ). We choose
$\x_j,j=1,\dots,m$ \DIFdelbegin \DIFdel{so as to minimize }\DIFdelend \DIFaddbegin \DIFadd{to minimize (}\DIFaddend an expression analogous to\DIFaddbegin \DIFadd{) }\DIFaddend the
log-determinant of the posterior covariance operator\DIFdelbegin \DIFdel{(corresponding to
a D-optimal design). We can see
the clusterization effect }\DIFdelend \DIFaddbegin \DIFadd{. We will see
later how this corresponds to maximizing expected information gain.
}


\DIFadd{The clusterization effect is illustrated }\DIFaddend in figure
\ref{fig:clusterization illustration}\DIFdelbegin \DIFdel{--- the design criterion
suggests placing sensors in locations that are extremely close. }%DIFDELCMD < 

%DIFDELCMD < %%%
\subsection{\DIFdel{Remarks to the Reader}}%DIFAUXCMD
\addtocounter{subsection}{-1}%DIFAUXCMD
%DIFDELCMD < \label{subsec:plan}
%DIFDELCMD < %%%
\DIFdel{The Lagrange multipliers problem for iid measurement noise is
discussed in section \ref{section:vanishing}. In that section, it is
shown how sensor clusterization can occur. The reader may skip if they
are
constrained by time. It is advised that a reader less comfortable
with Hilbert and Banach spaces mostly ignore these terms and consider
$\hil$ and $\ban$ finite dimensional vector spaces}\DIFdelend \DIFaddbegin \DIFadd{. Posterior pointwise standard
deviations are plotted over the domain $[0, \pi]$. For a Bayesian
linear inverse problem with Gaussian prior and noise the posterior
covariance does not depend on data \mbox{%DIFAUXCMD
\cite{Stuart10}}\hspace{0pt}%DIFAUXCMD
. Thus, the plot has
no reference to actual data observed. The posterior covariance does,
however, depend on the measurement taken. In figure
\ref{fig:clusterization illustration}, measurement locations are
marked by circles. These were chosen to minimize (an expression
analogous to) the log-determinant of the posterior covariance. The
clusterization effect can be clearly seen for six sensors. It looks
like only four measurements were taken. The reason is that two pairs
of sensors are so close they are indistinguishable}\DIFaddend .


%DIF < % Seeking an optimal design amounts to maximizing $\tar(\obs)$ in
%DIF < % $\obs$, subject to constraints posed on $\obs$. These enter as
%DIF < % constraints on the allowed set of measurement vectors we can choose
%DIF < % from --- the rows of $\obs$. These constraints are determined by the
%DIF < % problem structure --- for example, in the 1D heat equation example
%DIF < % considered in section \ref{subsec:example}, we were constrained to
%DIF < % choose $\obs = (\meas_1,\dots,\meas_m)^t$ with $\meas_j = \delta_{\x_j}$
%DIF < % for some $\x_j \in [0,\pi]$. We present an approach to understanding
%DIF < % optimal experiment design based on the first-order necessary condition
%DIF < % for optimal design, where the constraints are incorporated using
%DIF < % Lagrange multipliers. This condition is used to understand what causes
%DIF < % clusterization. One problem with such an approach is that the set we
%DIF < % can choose $\obs$ from is complicated and difficult to work with. For
%DIF < % example, if $\ban = C(\Omega)$ for some compact $\Omega$, then the
%DIF < % allowed set of measurement vectors may only be point evaluations --- a
%DIF < % set that is not easy to describe as a level set of some simple
%DIF < % function, a description crucial for the use of Lagrange multipliers.
%DIF < % In order to facilitate analysis, we consider an easier constraint in
%DIF < % section \ref{subsec:unit length}. Reverting to this simpler constraint
%DIF < % is a limitation of our analysis but it provides some insight into the
%DIF < % cause of clusterization.
\DIFdelbegin %DIFDELCMD < 

%DIFDELCMD < %%%
\DIFdelend \section{Preliminaries and Notation}\label{section:prelim}

\DIFaddbegin \subsection{\DIFadd{Overview}}
\DIFadd{In this section we define notations that will be used throughout the
paper. }\DIFaddend The theoretical foundations for inverse problems over function spaces
can be found in \cite{Stuart10}.
\DIFdelbegin \DIFdel{This section is mostly devoted to
setting notation. At this point, it is useful to record a known result
regarding the posterior covariance operator of our inverse problem,
namely
}%DIFDELCMD < \begin{align}\label{eq:postcov precursor}
%DIFDELCMD <   \postcov = (\prcov^{-1} + \fwd^* \obs^* \Sigma^{-1} \obs \fwd )^{-1}.
%DIFDELCMD < \end{align}
%DIFDELCMD < %%%
\DIFdelend %DIF > % A reader mostly familiar with the
%DIF > % literature may initially want to skip this section and only consult
%DIF > % the list below, which summarizes the notation used. We denote:
%DIF > % \begin{itemize}
%DIF > %   \item Parameter space $\hilp$.
%DIF > %   \item Data space $\hilo$.
%DIF > %   \item Forward dynamics operator $\fwd: \hilp \to \hilo$.
%DIF > %   \item Observation operator $\obs: \hilo \to \mathbb{R}^m$.
%DIF > %   \item Prior measure $\pr = \mathcal{N}(0, \prcov)$.
%DIF > %   \item Data $\data \in \mathbb{R}^m$.
%DIF > %   \item Observation and model errors, captured in $\Sigma(\obs)$.
%DIF > % \end{itemize}
%DIF > % The design objective was derived in \cite{AlexanderianGloorGhattas14}
%DIF > % and recorded in \eqref{eq:objective} below.



\subsection{Bayesian \DIFaddbegin \DIFadd{Linear }\DIFaddend Inverse Problems}\DIFdelbegin %DIFDELCMD < \label{section:abstract OED}
%DIFDELCMD < %%%
\DIFdel{Let $\hil$ be a Hilbert space and $\ban$ a Banach space, both
infinite-dimensional. Consider $\fwd: \hil \to \ban$}\DIFdelend \DIFaddbegin \label{subsec:abstract OED}
\DIFadd{Let $\hilp$ and $\hilo$ be seperable Hilbert spaces (the subscripts p
and o are for ``parameter'' and ``observation'', respectfully). We
denote $\hilo^*$ the Hilbert space of all linear functionals on
$\hilo$. Consider $\fwd: \hilp \to \hilo$}\DIFaddend , a linear operator (the
``forward operator''). We are interested in forward operators that are
strongly smoothing (have fast decaying \DIFdelbegin \DIFdel{singular
values }\DIFdelend \DIFaddbegin \DIFadd{modes }\DIFaddend --- the heat operator
from section \ref{subsec:example} is a prime example). \DIFdelbegin %DIFDELCMD < 

%DIFDELCMD < %%%
\DIFdel{We want to
infer $\param \in \hil$, }\DIFdelend \DIFaddbegin \DIFadd{Our goal is to
infer $\param \in \hilp$ --- }\DIFaddend some parameter of the dynamics \DIFaddbegin \DIFadd{--- }\DIFaddend given
noisy observations of $\fwd \param$. We \DIFdelbegin \DIFdel{model $\param \sim
\normal(0,\prcov)$ }\DIFdelend \DIFaddbegin \DIFadd{take a zero mean Gaussian
prior $\pr$ for $\param$ (otherwise the non-zero mean can just be
subtracted everywhere). Hence, $\param \sim \pr = \normal(0,\prcov)$
}\DIFaddend with some appropriate covariance operator $\prcov$ \cite{Stuart10}. \DIFaddbegin \DIFadd{It
is important to note that $\fwd \prcov \fwd^*$ is the prior covariance
in $\hilo$ \mbox{%DIFAUXCMD
\cite{Stuart10}}\hspace{0pt}%DIFAUXCMD
. As such, it is assumed invertible --- an
assumption we rely heavily upon. Throughout this paper, we denote $m$
the number of measurements we are allowed to take. }\DIFaddend Observations
(measurements) are taken via the linear observation operator \DIFdelbegin \DIFdel{$\obs \in ( \ban^* )^m$, where $m$ is the number
of observations allowed (e.g. sensors at our disposal)}\DIFdelend \DIFaddbegin \DIFadd{$\obs \in
( \hilo^* )^m$}\DIFaddend . In an analogy with linear algebra (where row vectors
are thought of as linear functionals), we think of $\obs$ as having
``rows'' $\meas_j, j=1,\dots,m$:
\DIFdelbegin %DIFDELCMD < \begin{equation}\label{eq:O}
%DIFDELCMD <   \obs = (\meas_1,\dots, \meas_m)^t, \meas_j \in \ban^*, j = 1,\dots,m.
%DIFDELCMD < \end{equation}
%DIFDELCMD < %%%
\DIFdelend \DIFaddbegin \begin{equation}\label{eq:O}
  \obs = (\meas_1,\dots, \meas_m)^t, \meas_j \in \hilo^*, j = 1,\dots,m.
\end{equation}
\DIFaddend This way, for \DIFdelbegin \DIFdel{$u \in \ban$ }\DIFdelend \DIFaddbegin \DIFadd{$u \in \hilo$ }\DIFaddend we have $\obs u = (\meas_1(u), \dots,
\meas_m(u) )^t \in \R^m$.
%% but we may drop the parentheses and write $\meas_j u = \meas_j(u)$.
\DIFdelbegin \DIFdel{A few observations regarding $\obs$ are in order. First, it is good to
keep in mind that $( \ban^* )^m$ is a Banach space with norm $\| \obs
\| = \sum_{j=1}^m \|\meas_j\|$. Next, for $u \in \ban$ }\DIFdelend %DIF > % A few observations regarding $\obs$ are in order.
%DIF > % First, it is good to keep in mind that $( \hilo^* )^m$ is a Hilbert
%DIF > % space with norm $\| \obs \| = \sum_{j=1}^m \|\meas_j\|$.
\DIFaddbegin \DIFadd{For $u \in \hilo$ }\DIFaddend and $v\in \R^m$:
\begin{align*}
  \big (\obs^*v \big ) (u) &= \langle v, \obs u \rangle_{\R^m} = \sum_{j=1}^m  v_j \meas_j(u)
  = v^t \left ( \obs u \right ) = (v^t \obs) (u),
\end{align*}
and thus:
\begin{align}\label{eq:obs*}
  \obs^*v &= \sum_{j=1}^m v_j \meas_j = v^t \obs.
\end{align}

\DIFdelbegin %DIFDELCMD < \begin{observation}
%DIFDELCMD <   It is best to think of $\meas_j,j=1,\dots,m$ as row vectors and of
%DIFDELCMD <   $\obs$ as a matrix with rows $\meas_j$. 
%DIFDELCMD < \end{observation}
%DIFDELCMD < %%%
\DIFdelend %DIF > % \begin{observation}
%DIF > %   It is best to think of $\meas_j,j=1,\dots,m$ as row vectors and of
%DIF > %   $\obs$ as a matrix with rows $\meas_j$. 
%DIF > % \end{observation}

Each measurement \DIFdelbegin \DIFdel{consists of }\DIFdelend \DIFaddbegin \DIFadd{is }\DIFaddend a linear functional, chosen from \DIFdelbegin \DIFdel{an
allowed set of
linear functionals in $\ban^{*}$}\DIFdelend \DIFaddbegin \DIFadd{some subset of
$\hilo^{*}$}\DIFaddend . Data is acquired via noisy measurements
\DIFdelbegin %DIFDELCMD < \begin{align*}
%DIFDELCMD <   \data := \obs (\fwd \param + \eps') + \eps = \obs \fwd \param + \obs \eps' + \eps
%DIFDELCMD < \end{align*}
%DIFDELCMD < %%%
\DIFdelend \DIFaddbegin \begin{align*}
  \data := \obs (\fwd \param + \eps') + \eps = \obs \fwd \param + \obs \eps' + \eps,
\end{align*}
\DIFaddend with $\data \in \R^m$ and $\eps, \eps'$ defined next. \DIFdelbegin \DIFdel{Measurement
error consists of two
parts}\DIFdelend \DIFaddbegin \DIFadd{We consider two
types of noise / error terms}\DIFaddend . First, there is spatially correlated
\DIFaddbegin \DIFadd{model }\DIFaddend error $\eps' \sim \normal(0,\modcov)$, modeled as a centered
Gaussian measure on \DIFdelbegin \DIFdel{$\ban$ }\DIFdelend \DIFaddbegin \DIFadd{$\hilo$ }\DIFaddend with covariance operator \DIFdelbegin \DIFdel{$\modcov: \ban^* \to \ban$
(see \mbox{%DIFAUXCMD
\cite[section 6]{Stuart10} }\hspace{0pt}%DIFAUXCMD
for some details on Gaussian measures
on Banach spaces)}\DIFdelend \DIFaddbegin \DIFadd{$\modcov$}\DIFaddend . Then
there is measurement / independent error $\eps \sim \normal(0,
\sigma^2 I_m)$, with $I_m$ the $m \times m$ identity. Both error terms
and the prior are assumed independent of each other.
\DIFdelbegin \DIFdel{Let us now }\DIFdelend \DIFaddbegin 

\DIFadd{At this point, it is useful to record a known result regarding the
posterior covariance operator of our inverse problem, namely
}\begin{align}\label{eq:postcov}
  \postcov = (\prcov^{-1} + \fwd^* \obs^* \Sigma^{-1} \obs \fwd
  )^{-1}.
\end{align}
\DIFadd{Usually, this is written using $\tmp := \obs \fwd$ and $\tmp^*$. Since
we need to separate the observation operator $\obs$ from the forward
operator $\fwd$ we write the explicit expression
}\eqref{eq:postcov}\DIFadd{. We now turn to }\DIFaddend understand the terms present in the
above expression.



\subsubsection{Dynamics and Observation Operators}\label{subsec:dynamics}
One can \DIFdelbegin \DIFdel{think }\DIFdelend \DIFaddbegin \textit{\DIFadd{think}} \DIFaddend of $(\obs \fwd \param)_j$ as pointwise
evaluations of a continuous function\DIFdelbegin \DIFdel{if $\ban$ is some appropriate function space,
e.g.\ $C(\Omega)$, the Banach space of real-valued continuous
functions over a compact $\Omega$ with the $\| \cdot \|_{\infty}$
norm. These }\DIFdelend \DIFaddbegin \footnote{\DIFadd{Of course, this is
  impossible if, e.g., $\hilo = L^2(\Omega)$ for some $\Omega$.}}\DIFadd{. Such
}\DIFaddend measurements can be represented using the linear functionals
$\delta_{\x_j} \in C(\Omega)^*$. In a different setting, we may be
able to measure other quantities, e.g.\ some Fourier coefficients as
is the case in MRI applications. Either way, the rows of $\obs$ are
linear functionals and we refer to these as {\it measurement
  vector}s. Note that we cannot choose every measurement vector we
want\DIFdelbegin \DIFdel{, though}\DIFdelend . For example, we may be restricted only to pointwise evaluations
of $\fwd \param$ by the sensors at our disposal, in which case
measuring the mean $\ell(u) = \int_{\Omega}u$ (or any non-atomic
measure) of a function is not possible.

\subsubsection{Error Terms}
Considering the sum of the error terms it is easy to see that
\DIFdelbegin \DIFdel{$\obs
\eps' + \eps \in \R^m$}\DIFdelend \DIFaddbegin \DIFadd{$\bar{\eps} := \obs \eps' + \eps \in \R^m$}\DIFaddend . It is a centered Gaussian
random vector and its covariance matrix is
\begin{align}\label{eq:Sigma}
  \begin{split}
    \Sigma(\obs) :&= \mathbb{E}[ (\obs \eps' + \eps)  (\obs \eps' + \eps)^t ] 
    % 
    % 
    = \obs \modcov \obs^* + \sigma^2I , 
  \end{split}
\end{align}
where
\begin{equation}\label{eq:modcov explained}
  [\obs \modcov \obs^*]_{ij} = e_i^t \obs \modcov \obs^* e_j = \meas_i (\modcov \meas_j),
\end{equation}
\DIFdelbegin \DIFdel{where the }\DIFdelend \DIFaddbegin \DIFadd{with $e_j \in \R^m$ the $j$th standard basis vector. The }\DIFaddend first
equality in \eqref{eq:modcov explained} holds by definition and the
second by \eqref{eq:obs*}. The explicit dependence on $\obs$ will be
mostly dropped for notational convenience, so $\Sigma(\obs) =
\Sigma$. Thus, for a fixed $\obs$ (i.e.\ a fixed set of measurements)
we can equivalently write $\data = \obs \fwd \param + \bar{\eps}$ with
$\bar{\eps} \sim \normal(0,\Sigma)$. \DIFdelbegin \DIFdel{At this point,
it is useful to rewrite the posterior covariance operator of our
inverse problem from }%DIFDELCMD < \eqref{eq:postcov precursor}
%DIFDELCMD < \begin{align}\label{eq:postcov}
%DIFDELCMD <   \postcov = (\prcov^{-1} + \fwd^* \obs^* \Sigma^{-1} \obs \fwd
%DIFDELCMD <   )^{-1}.
%DIFDELCMD < \end{align}
%DIFDELCMD < 

%DIFDELCMD < %%%
%DIF < % Usually, this is written using $\tmp := \obs \fwd$ and $\tmp^*$ but
%DIF < % since we need to separate the observation operator $\obs$ from the
%DIF < % forward operator $\fwd$ we write the explicit expression
%DIF < % \eqref{eq:postcov}.
\DIFdelend Taking $\modcov = 0$ is common
practice \cite{Tarantola05,KaipioSomersalo05,Vogel02}\DIFdelbegin \DIFdel{and then we are reduced
}\DIFdelend \DIFaddbegin \DIFadd{. This reduces }\DIFaddend to
the case where we take iid observations\DIFdelbegin \DIFdel{and }\DIFdelend \DIFaddbegin \DIFadd{. Then }\DIFaddend $\Sigma = \sigma^2I$ is
simply a scalar matrix that does not depend on $\obs$. Note that
taking an error model with a non-scalar covariance as we do here
allows us to consider model error (modeled by $\modcov$) as well as
measurement error (modeled by $\sigma^2$). For example, say we believe
our forward model does not capture some small scale phenomenon.  Then
we may express this belief by saying $\tru = \fwd + \err$, with $\fwd$
depending on $\param$ and $\err$ depending on $\sspar$, and $\param
\perp \sspar$. We do not know much about this effect but it is
reasonable to assume it changes continuously in our domain. We (may
choose to) model it as $\normal (0, \modcov)$ and take $\modcov$ to
reflect the spatial (or other) variability we imagine $\err \sspar$
has. Such small scale phenomenon can arise as a modeling issue, where
we might not model the system in its entirety. It can also arise from
a numerical source, where our discretization of the system is not fine
enough to capture all small scale phenomena. \DIFdelbegin \DIFdel{We }\DIFdelend \DIFaddbegin \DIFadd{In section
\ref{section:non vanishing}, we }\DIFaddend will see that assuming some
correlation in the error \DIFdelbegin \DIFdel{may reduce }\DIFdelend \DIFaddbegin \DIFadd{mitigates }\DIFaddend the clusterization phenomenon, as
\DIFdelbegin \DIFdel{is also }\DIFdelend reported in the literature \cite{Ucinski05}.

\subsection{D-Optimal \DIFdelbegin \DIFdel{Design }\DIFdelend \DIFaddbegin \DIFadd{Designs }\DIFaddend in Infinite Dimensions}\label{subsec:D optimal design} 
\DIFaddbegin \DIFadd{A D-optimal design maximizes expected KL divergence between posterior
and prior. It is useful to recall the definition of KL divergence for
an arbitrary prior measure:
}$$
KL(\post^\data||\pr) = \int \log \frac{\der \post^\data}{\der \pr}(x) \der \post(x).
$$

\DIFaddend The meaning of D-optimal design in infinite-dimensional Hilbert spaces
was investigated in \cite{AlexanderianGloorGhattas14}. \DIFdelbegin \DIFdel{Note that }\DIFdelend \DIFaddbegin \DIFadd{There, }\DIFaddend the
authors make assumptions that amount to $\Sigma=I$ (implied by
$\modcov = 0,\sigma^2=1$), but we choose not to take these
simplification here. This is because $\Sigma$ can determine ``how
much'' clusterization we see. As stated
\cite[pp. 681]{AlexanderianGloorGhattas14}, the results hold for more
general covariance matrices. The conclusion is that in
infinite-dimensions, a D-optimal design is well-defined as maximizing
the expected KL divergence between posterior and prior\DIFdelbegin \DIFdel{and }\DIFdelend \DIFaddbegin \DIFadd{. The main
result }\DIFaddend is summarized, using our notation, in the following theorem:
\DIFdelbegin %DIFDELCMD < \begin{theorem}[Slightly modified Theorem 1 from \cite{AlexanderianGloorGhattas14}]
%DIFDELCMD <   Let $\pr = \normal(0,\prcov)$ be a centered Gaussian prior
%DIFDELCMD <   on $\hil$   and let $\post = \normal(\postmean,\postcov)$ 
%DIFDELCMD <   the posterior measure on $\ban$ for the Bayesian linear
%DIFDELCMD <   inverse problem $\data =  \obs \fwd\param + \obs \eps' + \eps$ discussed
%DIFDELCMD <   above. Then 
%DIFDELCMD <   \begin{align*}\label{eq:objective}
%DIFDELCMD <     \begin{split}
%DIFDELCMD <       \tar( \obs) :&= \mathbb{E}_{\data}\left [ D_{\text{KL}} (\post || \pr ) \right ] \\
%DIFDELCMD <       % 
%DIFDELCMD <       % 
%DIFDELCMD <       % 
%DIFDELCMD <       &= \frac12 \log \det 
%DIFDELCMD <       ( I + \prcov^{1/2}  \fwd ^* \obs^* \Sigma^{-1} \obs \fwd \prcov^{1/2}).
%DIFDELCMD <     \end{split}
%DIFDELCMD <   \end{align*}
%DIFDELCMD < \end{theorem}
%DIFDELCMD < %%%
\DIFdelend \DIFaddbegin \begin{theorem}[Slightly modified Theorem 1 from \cite{AlexanderianGloorGhattas14}]
  Let $\pr = \normal(0,\prcov)$ be a centered Gaussian prior
  on $\hilp$   and let $\post = \normal(\postmean,\postcov)$ 
  the posterior measure on $\hilo$ for the Bayesian linear
  inverse problem $\data =  \obs \fwd\param + \obs \eps' + \eps$ discussed
  above. Then 
  \begin{align}\label{eq:objective}
    \begin{split}
      \tar( \obs) :&= \mathbb{E}_{\data}\left [ D_{\text{KL}} (\post || \pr ) \right ] \\
      % 
      % 
      % 
      &= \frac12 \log \det 
      ( I + \prcov^{1/2}  \fwd ^* \obs^* \Sigma^{-1} \obs \fwd \prcov^{1/2}).
    \end{split}
  \end{align}
\end{theorem}
\begin{definition}\label{def:d optimality}
  A design $\obs^{\star}$ is said to be D-optimal if $\obs^{\star} = \argmax_{\obs} \tar(\obs)$.
\end{definition}
\DIFaddend 

\DIFdelbegin \section{\DIFdel{The Constrained Optimization Problem of D-Optimal Design}}%DIFAUXCMD
\addtocounter{section}{-1}%DIFAUXCMD
%DIFDELCMD < \label{section:D and grad}
%DIFDELCMD < %%%
\DIFdel{From }%DIFDELCMD < \eqref{eq:objective} %%%
\DIFdelend \DIFaddbegin \DIFadd{The intuition behind definition \ref{def:d optimality} is
straightforward. From }\eqref{eq:postcov}\DIFadd{:
}\begin{align*}
  \begin{split}
    ( I + \prcov^{1/2}  \fwd ^* \obs^* \Sigma^{-1} \obs \fwd \prcov^{1/2}) &= \prcov ( \prcov^{-1} + \fwd ^* \obs^* \Sigma^{-1} \obs \fwd) \\
    &= \prcov \postcov^{-1}.
  \end{split}
\end{align*}
\DIFadd{Since $\prcov$ is constant, a D-optimal design minimizes the posterior
covariance determinant, analogously to the finite-dimensional case.
}

\subsection{\DIFadd{Sequential vs Simultaneous Optimization}}\label{subsec:seq vs sim}
\DIFadd{From defintion \ref{def:d optimality} }\DIFaddend we wish to characterize solution(s) of the
following optimization problem for $\tar$. %DIF < %: (\ban^*)^m \to \R$:
%DIF > %: (\hilo^*)^m \to \R$:
\begin{align}\label{eq:optimization}
  \obs^{\star} := \argmax_{\obs} \tar( \obs ) 
  = \argmax_{\obs} \frac12 \log \det 
  (I + \prcov^{1/2} \fwd^*\obs^* \Sigma^{-1} \obs \fwd \prcov^{1/2}),
\end{align}
where $\obs$ is constrained to some allowed set of measurements (\DIFdelbegin \DIFdel{e.g.\ pointwise evaluationson $[0,\pi]$)%DIF < %, for the example in section \ref{subsec:example}).
}\DIFdelend \DIFaddbegin \DIFadd{we
may }\textit{\DIFadd{think}} \DIFadd{of these as pointwise evaluations, even though
these might not be in $\hilo$). We call this problem ``simultaneous
optimization'', since all measurements are decided on simulatneously.
}

\DIFadd{For computational reasons, one may prefer to find the best
measurements in a sequential manner. Denote
}\begin{equation}\label{eq:def obs_k}
  \obs_k := (\meas_1,\dots, \meas_k)^t,  k\leq m.
\end{equation}
\DIFadd{Sequential optimal design proceeds as follows. Find $\meas_1$ by
maximizing $\tar(\obs_1)$. Then, keeping $\meas_1$ fixed --- find
$\meas_2$ as the maximizer of $\tar(\obs_2)$. Then, find $\meas_3$ by
keeping $\meas_1,\meas_2$ fixed and taking $\meas_3$ the maximizer of
$\tar(\obs_3)$. Continue in this fashion until $\obs_m = \obs$ is
found, where $m$ is the number of available measurements. It is
important to notice that this scheme does not require actually
observing data --- in }\eqref{eq:objective} \DIFadd{data is averaged out.
}

\DIFadd{The analysis in this paper is conducted for the general simultaneous
optimization case. The sequential optimization case is dealt with in
section \ref{subsec:seq vs sim}. It is important to note, however that
all conclusions we arrive at for the simultaneous case easily
specialize to the sequential case by considering the posterior as the
next sequential step's prior.
}


\section{\DIFadd{The Constrained Optimization Problem of D-Optimal Design}}\label{section:D and grad}

\subsection{\DIFadd{Overview}}
\DIFaddend We seek a formulation of the optimal design \DIFaddbegin \DIFadd{problem
}\eqref{eq:optimization} \DIFaddend using Lagrange multipliers. In section
\ref{section:objective} we find the gradient of $\tar$ \DIFaddbegin \DIFadd{and record it
in }\eqref{eq:tar grad}\DIFaddend . In section \ref{subsec:unit length} we relax
the constraints on $\obs$ so that the optimization problem in
\eqref{eq:optimization} \DIFdelbegin \DIFdel{and }\DIFdelend \DIFaddbegin \DIFadd{is analytically tractable.  We }\DIFaddend find gradients
for the new constraints \DIFaddbegin \DIFadd{and record them in }\eqref{eq:grad
  constraints}\DIFadd{. Finally, in section \ref{subsec:necessary} we use
these to formulate the optimal design problem as a Lagrange
multipliers problem in equation }\eqref{eq:conditions}\DIFaddend .


\subsection{The Objective and its Gradient}\label{section:objective}
In order to use Lagrange multipliers, we need to find the gradient of
$\tar$. This section is mostly technical and devoted to calculating
said gradient. The result is recorded in \eqref{eq:tar grad}.

We start by calculating the first variation. Some calculations are
delegated to Lemma \ref{lemma:aux calc} in the appendix. \DIFdelbegin \DIFdel{Also, we
}\DIFdelend \DIFaddbegin \DIFadd{We }\DIFaddend use
the following lemma\DIFdelbegin \DIFdel{whose proof is also }\DIFdelend \DIFaddbegin \DIFadd{:
}\begin{restatable*}[Generalized from \cite{Lax97}]{lemma}{lax}\label{lemma:lax}
  Let $Y(t)$ be a differentiable operator-valued function. Assume 
  $I+Y(t)$ is invertible, $Y(t)$ self-adjoint and trace-class. Then
  \begin{equation*}
    \frac{\der \log \det (I+Y(t))}{\der t} = \tr{(I+Y(t))^{-1} \dot{Y}(t)}.
  \end{equation*}
\end{restatable*}

\DIFadd{The proof of \ref{lemma:lax} is }\DIFaddend delegated to the appendix\DIFdelbegin \DIFdel{:
}%DIFDELCMD < \begin{lemma}[Generalized from \cite{Lax97}]\label{lemma:lax}
%DIFDELCMD <   Let $Y(t)$ be a differentiable operator-valued function. Assume 
%DIFDELCMD <   $I+Y(t)$ is invertible, $Y(t)$ self-adjoint and trace-class. Then
%DIFDELCMD <   \begin{displaymath}
%DIFDELCMD <     \frac{\der \log \det (I+Y(t))}{\der t} = \tr{(I+Y(t))^{-1} \dot{Y}(t)}.
%DIFDELCMD <   \end{displaymath}
%DIFDELCMD < \end{lemma}
%DIFDELCMD < 

%DIFDELCMD < %%%
\DIFdel{First }\DIFdelend \DIFaddbegin \DIFadd{. Let
$T(\obs) := \obs^*\Sigma(\obs)\obs$. The first }\DIFaddend variation of $\tar
(\obs)$ in the direction $V$ is:
\DIFdelbegin %DIFDELCMD < \begin{align*}%%\label{eq:tar var}
%DIFDELCMD <   \begin{split}
%DIFDELCMD <     \delta \tar(\obs) V 
%DIFDELCMD <     :&= \frac{\der}{\der\tau} \Big |_{\tau=0} \tar(\obs + \tau V)\  \text{ (by definition of variation)}\\
%DIFDELCMD <     % 
%DIFDELCMD <     % 
%DIFDELCMD <     % 
%DIFDELCMD <     &= \frac12 \frac{\der}{\der \tau} \Big |_{\tau=0} \log \det 
%DIFDELCMD <     (I + \prcov^{1/2} \fwd^* T(\obs+\tau V)\fwd \prcov^{1/2} ) \text{ (by definition \eqref{eq:objective})}\\
%DIFDELCMD <     % 
%DIFDELCMD <     % 
%DIFDELCMD <     % 
%DIFDELCMD <     &= \frac12 \tr ( I + \prcov^{1/2} \fwd^* \obs^* \Sigma^{-1}
%DIFDELCMD <     \obs\fwd \prcov^{1/2} )^{-1}
%DIFDELCMD <     \frac{\der}{\der \tau} \Big |_{\tau=0}
%DIFDELCMD <     \prcov^{1/2} \fwd^* T(\obs+\tau V) \fwd \prcov^{1/2}\ \text{ (by \ref{lemma:lax})}\\
%DIFDELCMD <     % 
%DIFDELCMD <     % 
%DIFDELCMD <     % 
%DIFDELCMD <     &= \frac12 \tr \postcov \fwd^* (
%DIFDELCMD <       V^* \Sigma^{-1} \obs 
%DIFDELCMD <       - \obs^*\Sigma^{-1} V\modcov \obs^* \Sigma^{-1}\obs \text{ (by \ref{lemma:aux calc})}\\
%DIFDELCMD <       &\ \ \ - \obs^* \Sigma^{-1} \obs \modcov V^* \Sigma^{-1}\obs
%DIFDELCMD <       + \obs^* \Sigma^{-1} V ) \fwd \\
%DIFDELCMD < %%   \end{split}
%DIFDELCMD < %% \end{align} 
%DIFDELCMD < %% Taking adjoints and cyclic permutations we see that
%DIFDELCMD < %% \begin{align*}
%DIFDELCMD < %%   \tr \postcov \fwd^* V^* \Sigma^{-1} \obs \fwd 
%DIFDELCMD < %%   &= \tr \postcov \fwd^* \obs^* \Sigma^{-1} V \fwd,
%DIFDELCMD < %% \end{align*}
%DIFDELCMD < %% and
%DIFDELCMD < %% \begin{align*}
%DIFDELCMD < %%   \tr \postcov \fwd^* \obs^*\Sigma^{-1} V\modcov \obs^* \Sigma^{-1}\obs \fwd
%DIFDELCMD < %%   &= \tr \postcov \fwd^*  \obs^* \Sigma^{-1} \obs \modcov V^* \Sigma^{-1}\obs \fwd.
%DIFDELCMD < %% \end{align*}
%DIFDELCMD < %% We use these in the first equality of the following derivation,
%DIFDELCMD < %% which continues the one in \eqref{eq:precursor}.
%DIFDELCMD < %% %
%DIFDELCMD < %% \begin{align} \label{eq:tar var}
%DIFDELCMD < %%   \begin{split}
%DIFDELCMD <       %% \delta \tar(\obs) V 
%DIFDELCMD <     &= \tr \postcov \fwd^* ( \obs^* \Sigma^{-1} V 
%DIFDELCMD <     - \obs^*\Sigma^{-1} V\modcov \obs^* \Sigma^{-1}\obs ) \fwd \\
%DIFDELCMD <     %
%DIFDELCMD <     %
%DIFDELCMD <     % 
%DIFDELCMD <     &= \tr \postcov \fwd^* \obs^* \Sigma^{-1} V 
%DIFDELCMD <     ( I - \modcov \obs^* \Sigma^{-1}\obs ) \fwd \\
%DIFDELCMD <     % 
%DIFDELCMD <     %
%DIFDELCMD <     %
%DIFDELCMD <     &= \tr V ( I - \modcov \obs^* \Sigma^{-1}\obs )
%DIFDELCMD <     \fwd \postcov \fwd^* \obs^* \Sigma^{-1}.
%DIFDELCMD <   \end{split}
%DIFDELCMD < \end{align*} 
%DIFDELCMD < %%%
\DIFdelend \DIFaddbegin \begin{align*}%%\label{eq:tar var}
  \begin{split}
    \delta \tar(\obs) V 
    :&= \frac{\der}{\der\tau} \Big |_{\tau=0} \tar(\obs + \tau V)\  \text{ (by definition of variation)}\\
    % 
    % 
    % 
    &= \frac12 \frac{\der}{\der \tau} \Big |_{\tau=0} \log \det 
    (I + \prcov^{1/2} \fwd^* T(\obs+\tau V)\fwd \prcov^{1/2} ) \text{ (by definition \eqref{eq:objective})}\\
    % 
    % 
    % 
    &= \frac12 \tr{( I + \prcov^{1/2} \fwd^* \obs^* \Sigma^{-1}
    \obs\fwd \prcov^{1/2} )^{-1}
    \frac{\der}{\der \tau} \Big |_{\tau=0}
    \prcov^{1/2} \fwd^* T(\obs+\tau V) \fwd \prcov^{1/2}}\ \text{ (by \ref{lemma:lax})}\\
    % 
    % 
    % 
    &= \frac12 \ttr{\postcov \fwd^* (V^* \Sigma^{-1} \obs 
      - \obs^*\Sigma^{-1} V\modcov \obs^* \Sigma^{-1}\obs \\
      &\ \ \ - \obs^* \Sigma^{-1} \obs \modcov V^* \Sigma^{-1}\obs 
      + \obs^* \Sigma^{-1} V ) \fwd}  \text{ (by \ref{lemma:aux calc})}\\
%%   \end{split}
%% \end{align} 
%% Taking adjoints and cyclic permutations we see that
%% \begin{align*}
%%   \tr \postcov \fwd^* V^* \Sigma^{-1} \obs \fwd 
%%   &= \tr \postcov \fwd^* \obs^* \Sigma^{-1} V \fwd,
%% \end{align*}
%% and
%% \begin{align*}
%%   \tr \postcov \fwd^* \obs^*\Sigma^{-1} V\modcov \obs^* \Sigma^{-1}\obs \fwd
%%   &= \tr \postcov \fwd^*  \obs^* \Sigma^{-1} \obs \modcov V^* \Sigma^{-1}\obs \fwd.
%% \end{align*}
%% We use these in the first equality of the following derivation,
%% which continues the one in \eqref{eq:precursor}.
%% %
%% \begin{align} \label{eq:tar var}
%%   \begin{split}
      %% \delta \tar(\obs) V 
    &= \tr{\postcov \fwd^* ( \obs^* \Sigma^{-1} V 
    - \obs^*\Sigma^{-1} V\modcov \obs^* \Sigma^{-1}\obs ) \fwd} \\
    %
    %
    % 
    &= \tr{\postcov \fwd^* \obs^* \Sigma^{-1} V 
    ( I - \modcov \obs^* \Sigma^{-1}\obs ) \fwd} \\
    % 
    %
    %
    &= \tr{V ( I - \modcov \obs^* \Sigma^{-1}\obs )
    \fwd \postcov \fwd^* \obs^* \Sigma^{-1}}.
  \end{split}
\end{align*} 
\DIFaddend Denote
\begin{align}\label{eq:tar grad}
  \nabla \tar(\obs) &:= (I - \modcov \obs^* \Sigma^{-1} \obs) \fwd
  \postcov \fwd^* \obs^*\Sigma^{-1},
\end{align}
the gradient of $\tar (\obs )$ and we now justify this definition.
Since the trace of $A := V \nabla \tar(\obs) \in \R^{m \times m}$ is
just $\ttr A = \sum_{j=1}^m e_j^t A e_j$ (with $e_j$ the $j$th standard
basis vector), we see that
\DIFdelbegin %DIFDELCMD < \begin{align*}
%DIFDELCMD <   \delta \tar(\obs)V = \tr V \nabla \tar(\obs) = \sum_{j=1}^m
%DIFDELCMD <   V_j(\nabla \tar(\obs)_j),
%DIFDELCMD < \end{align*}
%DIFDELCMD < %%%
\DIFdel{with $V_j \in \ban^*$ and $\nabla \tar(\obs)_j \in \ban^{**}\subseteq
\ban, j=1,\dots,m$}\DIFdelend \DIFaddbegin \begin{align*}
  \delta \tar(\obs)V = \tr{V \nabla \tar(\obs)} = \sum_{j=1}^m
  V_j(\nabla \tar(\obs)_j),
\end{align*}
\DIFadd{with $V_j \in \hilo^*$ and $\nabla \tar(\obs)_j \in \hilo^{**} =
\hilo, j=1,\dots,m$}\DIFaddend . Thus, \DIFdelbegin \DIFdel{$\nabla \tar( \obs ) \in (\ban^{**})^m
\subseteq \ban^m$ }\DIFdelend \DIFaddbegin \DIFadd{$\nabla \tar( \obs ) \in \hilo^m$ }\DIFaddend is indeed
the correct gradient and the notation \eqref{eq:tar grad} is
justified. Note that by the comment following \eqref{eq:obs*}, we view
$V$ (defined on the same space as $\obs$) as a column vector. Then the
gradient $\nabla \tar(\obs)$ should be viewed as a row vector, as the
product $V \nabla \tar$ is in $\R^{m \times m}$. This will prove
important in section \ref{subsec:necessary}.

\subsection{Unit Length Constraints and their Gradients}\label{subsec:unit length}
In this section we suggest relaxed constraints in
\eqref{eq:constraints} and derive their gradient in \eqref{eq:grad
  constraints}, for the purpose of using them in a Lagrange
multipliers problem.

As mentioned before, we cannot choose any $\obs$ when maximizing
$\tar(\obs)$. Recall \DIFdelbegin \DIFdel{$\obs = (\meas_1,\dots,\meas_m)^t \in (\ban^*)^m$
}\DIFdelend \DIFaddbegin \DIFadd{$\obs = (\meas_1,\dots,\meas_m)^t \in
(\hilo^*)^m$ }\DIFaddend and each $\meas_j,j=1,\dots,m$ must be chosen from some
allowed set of functionals in \DIFdelbegin \DIFdel{$\ban^*$}\DIFdelend \DIFaddbegin \DIFadd{$\hilo^*$}\DIFaddend . This set differs based on the
kind of sensors we have at our disposal and the properties of \DIFdelbegin \DIFdel{$\ban$}\DIFdelend \DIFaddbegin \DIFadd{$\hilo$}\DIFaddend ,
as discussed in section \ref{subsec:dynamics}. The following
proposition \DIFdelbegin \DIFdel{will give us
}\DIFdelend \DIFaddbegin \DIFadd{gives }\DIFaddend better understanding of the constraints.
\DIFdelbegin %DIFDELCMD < \begin{proposition}\label{prop:bigger better}
%DIFDELCMD <   Let $\obs = (\meas_1,\dots,\meas_m)^t$, $j \in \{1,\dots,m\}$, $\sigma^2
%DIFDELCMD <   > 0$ and $\lambda > 1$. Then $\tar(\obs)$ increases if we use
%DIFDELCMD <   $\lambda \meas_j$ in $\obs$ instead of $\meas_j$.
%DIFDELCMD < \end{proposition}
%DIFDELCMD < %%%
\DIFdelend \DIFaddbegin \begin{restatable*}{proposition}{biggerbetter}\label{prop:bigger better}
  Let $\obs = (\meas_1,\dots,\meas_m)^t$, $j \in \{1,\dots,m\}$, $\sigma^2
  > 0$ and $\lambda > 1$. Then $\tar(\obs)$ increases if we use
  $\lambda \meas_j$ in $\obs$ instead of $\meas_j$.
\end{restatable*}
\DIFaddend Proof of Proposition \ref{prop:bigger better} is delegated to the
appendix but the idea behind it is simple: making rows of $\obs$ large
is equivalent to making $\sigma^2$ small. %DIF > %  Viewed another way, it
%DIF > % increases the Signal to Noise Ratio (SNR) --- which is always
%DIF > % desirable when taking any measurement prone to error.
This can be understood easily from the formulation of the inverse
problem $\data = \obs \fwd \param + \obs \eps' + \eps$, where $\eps'$
can be taken as model error and $\eps$ is iid measurement error.

We now suggest a \DIFdelbegin \DIFdel{more convenient-to-work-with }\DIFdelend set of constraints \DIFaddbegin \DIFadd{that will make the model
analytically tractable}\DIFaddend . By Proposition \ref{prop:bigger better}, we
cannot just take all \DIFdelbegin \DIFdel{$\ban^*$
}\DIFdelend \DIFaddbegin \DIFadd{$\hilo^*$ }\DIFaddend to be the allowed set of measurement
vectors, since this would effectively eliminate measurement error
entirely. We note that for point evaluations, the norm of the
measurement is always one:
\begin{align*}
  \| \delta_{\x} \| = \sup_{0 \neq u \in C(\Omega)} \frac{
    |\int_{\Omega}u(\y) \delta_{\x}(\y) \der \y| 
  }{
    \sup|u|}
  = \sup_{0 \neq u \in C(\Omega)} \frac{|u(\x)|}{ \sup|u|} = 1,
  \forall \x \in \Omega.
\end{align*}
Following this, we also restrict the norm of our measurement vectors
to be one. \DIFdelbegin \DIFdel{But the norm on functionals is hard to work with. Instead,
we assume there exists a Hilbert space $\bar{\hil} \subseteq \ban^*$
and take measurements $\meas_j \in \bar{\hil}^* = \bar{\hil}$ such
that $\| \meas_j\| = 1, j=1,\dots,m$. Recall that we think of $\ban$ as
a somewhat restricted space ($C(\bar{\Omega})$ with the sup-norm over
a bounded domain $\Omega$, for example) and so $\ban^*$ can be thought
of as large. Then $\ban^*$ can accommodate a reasonable Hilbert space
$\bar{\hil}$. For example, if $\ban = C(\bar{\Omega})$, we take
$\bar{\hil} = L^2(\Omega) \subseteq \ban^*$.
%DIF < % (we can ignore the
%DIF < % boundary since is has zero measure).
Then
%DIF < 
}%DIFDELCMD < \begin{equation*}
%DIFDELCMD <   \ban \subseteq \bar{\hil} \subseteq \ban^*,
%DIFDELCMD < \end{equation*}
%DIFDELCMD < %%%
\DIFdel{since members of $C(\bar{\Omega})$ are bounded, hence square integrable. }\DIFdelend The unit norm \DIFdelbegin \DIFdel{constraint }\DIFdelend \DIFaddbegin \DIFadd{constraints }\DIFaddend can be written using
\eqref{eq:obs*} as a series of $m$ equality constraints (one for each
measurement) on $\obs$ as
\DIFdelbegin %DIFDELCMD < \begin{align}\label{eq:constraints}
%DIFDELCMD <   \phi_j(\obs) :=\frac12 \| \obs^* e_j\|_{\bar{\hil}}^2 - \frac12 = 0,j=1,\dots,m
%DIFDELCMD < \end{align}
%DIFDELCMD < %%%
\DIFdel{with $e_j \in \R^m$ the $j$th standard basis vector. }\DIFdelend \DIFaddbegin \begin{align}\label{eq:constraints}
  \phi_j(\obs) :=\frac12 \| \obs^* e_j\|_{\hilp}^2 - \frac12 = 0,j=1,\dots,m.
\end{align}
\DIFaddend The first variations are:
\DIFdelbegin %DIFDELCMD < \begin{align*}
%DIFDELCMD <   \delta \phi_j(\obs)V  
%DIFDELCMD <   &= \frac12\lim_{\tau \to 0}\tau^{-1}
%DIFDELCMD <   ( \|(\obs + \tau V)^*e_j \|_{\bar{\hil}}^2 - \|\obs ^*e_j \|_{\bar{\hil}}^2  ) \\
%DIFDELCMD <   %
%DIFDELCMD <   %
%DIFDELCMD <   %
%DIFDELCMD <   &= \frac12\lim_{\tau \to 0}\tau^{-1}
%DIFDELCMD <   ( \langle (\obs + \tau V)^*e_j, (\obs + \tau V)^*e_j \rangle_{\bar{\hil}} - 
%DIFDELCMD <   \langle \obs^*e_j, \obs^*e_j \rangle_{\bar{\hil}} ) \\
%DIFDELCMD <   % 
%DIFDELCMD <   % 
%DIFDELCMD <   %
%DIFDELCMD <   &= \frac12\lim_{\tau \to 0}\tau^{-1}
%DIFDELCMD <   (2\tau \langle \obs^*e_j,V^*e_j \rangle_{\bar{\hil}} 
%DIFDELCMD <   +\tau^2 \langle V^*e_j, V^*e_j \rangle_{\bar{\hil}} ) \\
%DIFDELCMD <   %
%DIFDELCMD <   %
%DIFDELCMD <   % 
%DIFDELCMD <   &= \langle \obs^*e_j,V^*e_j \rangle_{\bar{\hil}} \\
%DIFDELCMD <   %
%DIFDELCMD <   %
%DIFDELCMD <   % 
%DIFDELCMD <   &= \langle V \obs^*e_j,e_j \rangle_{\R^m} \\
%DIFDELCMD <   %
%DIFDELCMD <   %
%DIFDELCMD <   %
%DIFDELCMD <   &= e_j^t V \obs^* e_j \\
%DIFDELCMD <   % 
%DIFDELCMD <   %
%DIFDELCMD <   %
%DIFDELCMD <   &= \tr V \obs^* e_je_j^t.
%DIFDELCMD < \end{align*}
%DIFDELCMD < %%%
\DIFdelend \DIFaddbegin \begin{align*}
  \delta \phi_j(\obs)V  
  &= \frac12\lim_{\tau \to 0}\tau^{-1}
  ( \|(\obs + \tau V)^*e_j \|_{\bar{\hilp}}^2 - \|\obs ^*e_j \|_{\bar{\hilp}}^2  ) \\
  %
  %
  %
  &= \frac12\lim_{\tau \to 0}\tau^{-1}
  ( \langle (\obs + \tau V)^*e_j, (\obs + \tau V)^*e_j \rangle_{\bar{\hilp}} - 
  \langle \obs^*e_j, \obs^*e_j \rangle_{\bar{\hilp}} ) \\
  % 
  % 
  %
  &= \frac12\lim_{\tau \to 0}\tau^{-1}
  (2\tau \langle \obs^*e_j,V^*e_j \rangle_{\bar{\hilp}} 
  +\tau^2 \langle V^*e_j, V^*e_j \rangle_{\bar{\hilp}} ) \\
  %
  %
  % 
  &= \langle \obs^*e_j,V^*e_j \rangle_{\bar{\hilp}} \\
  %
  %
  % 
  &= \langle V \obs^*e_j,e_j \rangle_{\R^m} \\
  %
  %
  %
  &= e_j^t V \obs^* e_j \\
  % 
  %
  %
  &= \tr{V \obs^* e_je_j^t}.
\end{align*}
\DIFaddend Using the same arguments we used to justify the definition of
\eqref{eq:tar grad}, we conclude that
\begin{align}\label{eq:grad constraints}
\nabla \phi_j(\obs) = \obs^* e_j e_j^t = \meas_j e_j^t , j=1,\dots,m,
\end{align}
with \DIFdelbegin \DIFdel{$\nabla \phi_j(\obs) \in \bar{\hil}^m \subseteq \ban^m$ }\DIFdelend \DIFaddbegin \DIFadd{$\nabla \phi_j(\obs) \in \hilo^m$ }\DIFaddend and where the
last equality follows since $\obs^*e_j = \meas_j$ by \eqref{eq:obs*}. As
we noted at the end of section \DIFdelbegin \DIFdel{\ref{section:D and grad}}\DIFdelend \DIFaddbegin \DIFadd{\ref{section:objective}}\DIFaddend , the
gradient $\nabla \phi_j(\obs)$ is a row vector.

\subsection{Necessary Conditions for Optimal Design}\label{subsec:necessary}
Necessary conditions for optimality are found using Lagrange
multipliers:
\begin{align}
  %\begin{split}
  &\nabla \tar(\obs) = \sum_{j=1}^m \xi_j \nabla \phi_j (\obs)
  \label{eq:Lagrange mult1} \\
    &\phi_j(\obs) = 0, j = 1,\dots,m. \label{eq:Lagrange mult2}
  %\end{split}
\end{align}
%% This means that any optimal design satisfies \eqref{eq:Lagrange mult1}
%% and \eqref{eq:Lagrange mult2}.
Recall where the objects involved are defined: the observation operator,
\DIFdelbegin \DIFdel{$\obs \in \bar{\hil}^m \subseteq (\ban^*)^m$}\DIFdelend \DIFaddbegin \DIFadd{$\obs \in (\hilo^*)^m$}\DIFaddend . The objective \DIFdelbegin \DIFdel{$\tar:
(\ban^*)^m \to \R$ }\DIFdelend \DIFaddbegin \DIFadd{$\tar:
(\hilo^*)^m \to \R$ }\DIFaddend and its gradient is \DIFdelbegin \DIFdel{$\nabla \tar(\obs) \in
\ban^m$}\DIFdelend \DIFaddbegin \DIFadd{$\nabla \tar(\obs) \in
\hilo^m$}\DIFaddend . The constraints are defined similarly --- \DIFdelbegin \DIFdel{$\phi_j: (\ban^*)^m
\to \R$ and $\nabla \phi_j(\obs) \in \ban^m$}\DIFdelend \DIFaddbegin \DIFadd{$\phi_j: (\hilo^*)^m
\to \R$ and $\nabla \phi_j(\obs) \in \hilo^m$}\DIFaddend .

Focusing on \eqref{eq:Lagrange mult1} and using the gradients of the
objective and constraints \eqref{eq:tar grad} and \eqref{eq:grad
  constraints} we have:
\begin{equation*}
  (I - \modcov \obs^* \Sigma^{-1} \obs) \fwd \postcov \fwd^* \obs^*\Sigma^{-1}
  = \sum_{j=1}^m \xi_j \obs^* e_je_j^t = (\xi_1 \meas_1,\dots,\xi_m \meas_m).
\end{equation*}
If we let $\Xi = \diag(\xi_j)$, then with some abuse of
notation, we can write this more compactly as:
\begin{equation}\label{eq:conditions}
  ( I - \modcov \obs^* \Sigma^{-1} \obs) \fwd \postcov \fwd^* \obs^*  \Sigma^{-1}
  = \obs^* \Xi.
\end{equation}






%DIF > %%%%%%%%%%%%%%%%%%%%%%%%%%%%%%%%%%%%%%%%%%%%%%%%%%%%%%%%%%%%%%%
%DIF > % SECTION Analysis of Optimal Designs --- Vanishing Model Error
%DIF > %%%%%%%%%%%%%%%%%%%%%%%%%%%%%%%%%%%%%%%%%%%%%%%%%%%%%%%%%%%%%%%
\section{Analysis of Optimal Designs --- Vanishing Model Error}\label{section:vanishing}
\DIFdelbegin \DIFdel{When $\modcov = 0$, then }\DIFdelend \DIFaddbegin \subsection{\DIFadd{Overview}}
\DIFadd{In this section we analyze optimal designs when no model error is
present. We discuss sensor clusterization in section
\ref{section:clusterization}. We prepare the ground by first deriving
a ``weird'' eigenvalue problem in $\obs$ for D-optimal designs in
section \ref{subsec:eigenproblem}. This ``weird'' eigenvalue problem
is recorded in }\eqref{eq:mod conditions}\DIFadd{. Next, we characterize
optimal designs as ones that make all eigenvalues equal in the above
mentioned weird eigenvalue problem. The derivation is presented in
section \ref{subsec:characterization}.
}


\subsection{\DIFadd{The ``Weird'' Eigenvalue Problem}}\label{subsec:eigenproblem}
\DIFadd{$\modcov = 0$ implies }\DIFaddend $\Sigma= \sigma^2I$. The necessary first-order
condition for optimality \eqref{eq:conditions} becomes
\begin{equation}\label{eq:eigenproblem}
  \sigma^{-2}\fwd \postcov \fwd^* \obs^* = \obs^* \Xi, 
\end{equation}
with $\Xi$ diagonal. This \DIFdelbegin \DIFdel{is }\DIFdelend \DIFaddbegin \DIFadd{looks like }\DIFaddend an eigenvalue problem for the
self-adjoint operator $\sigma^{-2}\fwd \postcov \fwd^*$\DIFdelbegin \DIFdel{where the }\DIFdelend \DIFaddbegin \DIFadd{. The }\DIFaddend rows of
$\obs$, namely $\meas_j,j=1,\dots, m$, are the eigenvectors. \DIFdelbegin \DIFdel{If }\DIFdelend \DIFaddbegin \DIFadd{Note,
however, that $\postcov$ depends on $\obs$ --- which makes
}\eqref{eq:eigenproblem} \DIFadd{a ``weird'' eigenvalue problem.
}

\DIFadd{Below, the expression for the posterior covariance }\eqref{eq:postcov}
\DIFadd{justifies the second equality. Some calculations delegated to lemma
\ref{lemma:twice woodbury} in the appendix justify the third (trivial
if }\DIFaddend $\fwd$ \DIFdelbegin \DIFdel{is invertible,
then we may write:
}%DIFDELCMD < \begin{align}\label{eq:mod conditions}
%DIFDELCMD <   \begin{split}
%DIFDELCMD <   \obs^* \Xi &= \sigma^{-2}\fwd \postcov \fwd^* \obs^*  \\
%DIFDELCMD <   %
%DIFDELCMD <   %
%DIFDELCMD <   %
%DIFDELCMD <   &= \sigma^{-2} \fwd( \prcov^{-1} + \sigma^{-2}  \fwd^* \obs^* \obs \fwd )^{-1} \fwd^* \obs^* \\
%DIFDELCMD <   %
%DIFDELCMD <   %
%DIFDELCMD <   %
%DIFDELCMD <   &= \sigma^{-2} \left ( (\fwd\prcov\fwd^*)^{-1} + \sigma^{-2}  \obs^* \obs \right )^{-1} \obs^*,
%DIFDELCMD <   \end{split}
%DIFDELCMD < \end{align}  
%DIFDELCMD < %%%
\DIFdel{which presents the optimal design problem as a problem on $\ban$ and
not on $\hil$ --- the }\DIFdelend \DIFaddbegin \DIFadd{is invertible):
}\begin{align}\label{eq:mod conditions}
  \begin{split}
  \obs^* \Xi &= \sigma^{-2}\fwd \postcov \fwd^* \obs^*  \\
  %
  %
  %
  &= \sigma^{-2} \fwd( \prcov^{-1} + \sigma^{-2}  \fwd^* \obs^* \obs \fwd )^{-1} \fwd^* \obs^* \\
  %
  %
  %
  &= \sigma^{-2} \left ( (\fwd\prcov\fwd^*)^{-1} + \sigma^{-2}  \obs^* \obs \right )^{-1} \obs^*
  \end{split}
\end{align}  
\DIFadd{The }\DIFaddend term $(\fwd \prcov \fwd^*)^{-1}$ is the prior precision in \DIFdelbegin \DIFdel{$\ban$ and the big parentheses enclose the posterior
precision in $\ban$. This expression makes sense even if $\fwd$ is not
invertible--- consider instead $\fwd$ restricted to $\ker
\fwd^{\perp}$. This does not change the optimal design problem at all, since we will never take measurements that have a component in $\ker
\fwd$ (this can be understood from Corollary \ref{cor:zero mod err}). }\DIFdelend \DIFaddbegin \DIFadd{$\hilo$
(and is invertible, as mentioned in section
\ref{section:prelim}). $(\fwd\prcov\fwd^*)^{-1} + \sigma^{-2} \obs^*
\obs$ is the posterior precision in $\hilo$.
}\DIFaddend 

\subsection{Characterizing D-Optimal Designs}\DIFdelbegin %DIFDELCMD < \label{subsub:characterization}
%DIFDELCMD < %%%
\DIFdel{In light of }%DIFDELCMD < \eqref{eq:mod conditions}%%%
\DIFdel{, let us denote }\DIFdelend \DIFaddbegin \label{subsec:characterization}
\DIFadd{Denote }\DIFaddend eigenvalues of $\fwd \prcov \fwd^*$ by
\DIFdelbegin \DIFdel{$\lambda_1 \geq \lambda_2 \geq \dots$ and its
}\DIFdelend \DIFaddbegin \DIFadd{$\{\lambda_i\}_{i=1}^{\infty}$. These are all positive. Denote
corresponding }\DIFaddend eigenvectors $\{\ev_i\}_{i=1}^{\infty}$. \DIFdelbegin \DIFdel{The following lemma will be useful:
}\DIFdelend \DIFaddbegin \DIFadd{Lemma
\ref{lemma:sim diag} (proved in the appendix) is crucial for the rest
of the analysis:
}\DIFaddend 

\DIFdelbegin %DIFDELCMD < \begin{lemma}[Proved as lemma \ref{lemma:sim diag} in the appendix]
%DIFDELCMD <   Let $C:\hil \to \hil$ self-adjoint and let $\func_1,\dots,\func_m \in
%DIFDELCMD <   \hil$. Denote $\func^*$ the element $\func$ acting as a linear
%DIFDELCMD <   functional. If
%DIFDELCMD <   \begin{displaymath}
%DIFDELCMD <    (C + \sum_{j=1}^m \func_j\func_j^*) \func_k = \xi_k \func_k, k = 1,\dots,m
%DIFDELCMD <   \end{displaymath}
%DIFDELCMD <   then $C$ and $\sum_{j=1}^m \func_j \func_j^*$ are simultaneously
%DIFDELCMD <   diagonalizable.
%DIFDELCMD < \end{lemma}
%DIFDELCMD < %%%
\DIFdelend \DIFaddbegin \begin{restatable*}[Simultaneous diagonizability for the nonlinear eigenvalue problem]{lemma}{simdiag}\label{lemma:sim diag}
  Let $\hil$ seperable Hilbert space, $C:\hil \to \hil$ self-adjoint
  and $\func_1,\dots,\func_m \in \hil$. Denote $\func^*$ the element
  $\func$ acting as a linear functional. If
  \begin{equation*}
   (C + \sum_{j=1}^m \func_j\func_j^*) \func_l = \xi_l \func_l, l = 1,\dots,m
  \end{equation*}
  then $C$ and $\sum_{j=1}^m \func_j \func_j^*$ are simultaneously
  diagonalizable.
\end{restatable*}
\DIFaddend 

\DIFdelbegin \DIFdel{Thus, if we take $\func_j^{*} = \meas_j$ and $C := (\fwd \prcov
\fwd^*)^{-1}$ in the above lemma, we conclude from the eigenproblem
}%DIFDELCMD < \eqref{eq:mod conditions} %%%
\DIFdel{that $\obs^*\obs$ has the same eigenvectors
as $\fwd \prcov \fwd^*$, namely $\{\ev_i\}_{i=1}^{\infty}$ (infinitely
many of them with zero eigenvalue, since }\DIFdelend \DIFaddbegin \begin{corollary}[Eigenvectors of a D-optimal design]\label{cor:same ev}
  $\obs^*\obs$ has the same eigenvectors as $\fwd \prcov \fwd^*$.
  %namely $\{\ev_i\}_{i=1}^{\infty}$.
\end{corollary}
\begin{proof}
In \eqref{eq:mod conditions}, take $\func_j^{*} = \meas_j$ and $C :=
(\fwd \prcov \fwd^*)^{-1}$ and use lemma \ref{lemma:sim diag}.
\end{proof}
\DIFadd{Only finitely many of the eigenvectors of }\DIFaddend $\obs^*\obs$ \DIFdelbegin \DIFdel{is
finite-rank)}\DIFdelend \DIFaddbegin \DIFadd{have positive
eigenvalue. This happens because $k := \rank \obs^*\obs \leq
m$}\DIFaddend . Denote the corresponding non-zero eigenvalues of $\obs^*\obs$ by
$\{\eta_i\}_{i=1}^{k}$ and let $\eta_i = 0$ for $i \geq k+1$. \DIFdelbegin \DIFdel{By the argument in the paragraph following }%DIFDELCMD < \eqref{eq:mod
%DIFDELCMD <   conditions}%%%
\DIFdel{:
}%DIFDELCMD < \begin{align}\label{eq:pre manipulations}
%DIFDELCMD <   \begin{split}
%DIFDELCMD <     I + \sigma^{-2} \prcov^{1/2}  \fwd ^* \obs^* \obs \fwd \prcov^{1/2} &= 
%DIFDELCMD <     \prcov^{1/2} ( \prcov^{-1} + \sigma^{-2} \fwd^*\obs^*  \obs \fwd ) \prcov^{1/2}\\
%DIFDELCMD <     %
%DIFDELCMD <     %
%DIFDELCMD <     %
%DIFDELCMD <     &=\prcov^{1/2} \fwd^* ( \fwd^{-*} \prcov^{-1}\fwd^{-1} + \sigma^{-2} \obs^* \obs  ) \fwd \prcov^{1/2}\\
%DIFDELCMD <     %
%DIFDELCMD <     %
%DIFDELCMD <     %
%DIFDELCMD <     &= \prcov^{1/2} \fwd^* \left ( (\fwd \prcov \fwd^* )^{-1} +\sigma^{-2} \obs^* \obs  \right ) \fwd \prcov^{1/2}.
%DIFDELCMD <   \end{split}
%DIFDELCMD < \end{align}
%DIFDELCMD < %%%
\DIFdelend \DIFaddbegin \DIFadd{No
assumption is made regarding the magnitudes the corresponding
$\{\lambda_i\}_{i=1}^{\infty}$. They are to be thought of as fixed but
unknown. Using Sylvester's determinant theorem (the standard proof
works for Hilbert spaces as well) we get:
}

\begin{align}\label{eq:pre manipulations}
  \begin{split}
    \det \left (I + \sigma^{-2} \prcov^{1/2} \fwd ^* \obs^*
    \obs \fwd \prcov^{1/2}\right ) &=
    %
    \det \left (I + \sigma^{-2} \obs^*
    \obs \fwd \prcov\fwd^* \right ) \\
    %
    %
    %
    &= \det \left \{ \left ( (\fwd \prcov \fwd^*)^{-1} +
    \sigma^{-2}\obs^*\obs \right ) (\fwd \prcov \fwd^*)\right \}\\
  \end{split}
\end{align}


\DIFaddend By definition of the objective \eqref{eq:objective} and using
\eqref{eq:pre manipulations}:
\DIFdelbegin %DIFDELCMD < \begin{align}\label{eq:mod objective}
%DIFDELCMD <   \begin{split}
%DIFDELCMD <     \tar ( \obs ) &=\frac12 \log \det \left( I + \sigma^{-2}
%DIFDELCMD <     \prcov^{1/2} \fwd ^* \obs^* \obs \fwd \prcov^{1/2} \right )\\
%DIFDELCMD <     %
%DIFDELCMD <     %
%DIFDELCMD <     %
%DIFDELCMD <     &=\frac12 \log \det
%DIFDELCMD <     \left ( (\fwd \prcov \fwd^* )^{-1} + \sigma^{-2}\obs^* \obs  \right ) \fwd \prcov \fwd^* \\ 
%DIFDELCMD <     %
%DIFDELCMD <     %
%DIFDELCMD <     %
%DIFDELCMD <     &=\frac12 \log \left (\prod_{i=1}^{\infty} ( \lambda_i^{-1} + \sigma^{-2} \eta_i ) \prod_{i=1}^{\infty} \lambda_i \right ) \\
%DIFDELCMD <     %
%DIFDELCMD <     %
%DIFDELCMD <     %
%DIFDELCMD <     &=\frac12 \log \left ( \prod_{i=1}^{k} ( \lambda_i^{-1} + \sigma^{-2} \eta_i )\prod_{i=1}^{k} \lambda_i \right ). 
%DIFDELCMD <   \end{split}
%DIFDELCMD < \end{align}
%DIFDELCMD < %%%
\DIFdel{Therefore, an optimal }\DIFdelend \DIFaddbegin \begin{align}\label{eq:mod objective}
  \begin{split}
    \tar ( \obs ) &=\frac12 \log \det \left( I + \sigma^{-2}
    \prcov^{1/2} \fwd ^* \obs^* \obs \fwd \prcov^{1/2} \right )\\
    %
    %
    %
    %
    &=\frac12 \log \det \Big ( \left ( (\fwd \prcov \fwd^* )^{-1} +
    \sigma^{-2}\obs^* \obs \right ) \fwd \prcov \fwd^* \Big )\\
    %
    %
    %
    &=\frac12 \log \left (\prod_{i=1}^{\infty} ( \lambda_i^{-1} + \sigma^{-2} \eta_i ) \prod_{i=1}^{\infty} \lambda_i \right ) \\
    %
    %
    %
    &=\frac12 \log \left ( \prod_{i=1}^{k} ( \lambda_i^{-1} + \sigma^{-2} \eta_i )\prod_{i=1}^{k} \lambda_i \right )\\
    %
    %
    %
    &=\frac12 \sum_{i=1}^{k} \log (\sigma^2\lambda_i^{-1} + \eta_i), 
  \end{split}
\end{align}
\DIFadd{up to an additive constant independent of $\obs$. Therefore, a
D-optimal }\DIFaddend design maximizes

\DIFdelbegin %DIFDELCMD < \begin{equation}\label{eq:true target}
%DIFDELCMD <   \sum_{i=1}^{k}\log (\lambda_i^{-1} + \sigma^{-2} \eta_i),
%DIFDELCMD < \end{equation}
%DIFDELCMD < 

%DIFDELCMD < \pgfplotstableread{
%DIFDELCMD <   Label    prior  $o_1$  $o_2$   $o_3$   $o_4$   $o_5$         $o_6$    topper
%DIFDELCMD <   1         0.2    0.8    0.5     0.5    0.4    0.33333333  0.3333333  0.001
%DIFDELCMD <   2         0.8    0.2    0.5     0.5    0.4    0.33333333  0.3333333  0.001
%DIFDELCMD <   3         2.2    0      0       0      0.2    0.33333333  0.3333333  0.001
%DIFDELCMD <   4         3.5    0      0       0      0      0           0          0.001
%DIFDELCMD < }\testdata
%DIFDELCMD < %%%
\DIFdelend \DIFaddbegin \begin{equation}\label{eq:true target}
  \sum_{i=1}^{k}\log (\sigma^2 \lambda_i^{-1} + \eta_i),
\end{equation}
\DIFaddend 

\DIFdelbegin %DIFDELCMD < \begin{figure}
%DIFDELCMD <   \begin{tikzpicture}
%DIFDELCMD <     \begin{axis}[
%DIFDELCMD <         ybar stacked,
%DIFDELCMD <         ymin=0,
%DIFDELCMD <         ymax=4,
%DIFDELCMD <         xtick=data,
%DIFDELCMD <         legend style={cells={anchor=east}, legend pos=north west, legend columns=-1},
%DIFDELCMD <         reverse legend=false, % set to false to get correct display, but I'd like to have this true
%DIFDELCMD <         xticklabels from table={\testdata}{Label},
%DIFDELCMD <         xticklabel style={text width=2cm,align=center},
%DIFDELCMD <         legend plot pos=right,
%DIFDELCMD <         ylabel=precision --- prior and posterior,
%DIFDELCMD <         xlabel=eigenvector,
%DIFDELCMD <       ]
%DIFDELCMD <     

%DIFDELCMD <       
%DIFDELCMD <       \addplot [fill=green!80]  table [y=prior, meta=Label, x expr=\coordindex] {\testdata};
%DIFDELCMD <       \addplot [fill=blue!60]   table [y=$o_1$, meta=Label, x expr=\coordindex] {\testdata};
%DIFDELCMD <       \addplot [fill=red!60]    table [y=$o_2$, meta=Label, x expr=\coordindex] {\testdata};
%DIFDELCMD <       \addplot [fill=black!60]  table [y=$o_3$, meta=Label, x expr=\coordindex] {\testdata};
%DIFDELCMD <       \addplot [fill=orange!60] table [y=$o_4$, meta=Label, x expr=\coordindex] {\testdata};
%DIFDELCMD <       \addplot [fill=cyan!60]   table [y=$o_5$, meta=Label, x expr=\coordindex] {\testdata};
%DIFDELCMD <       \addplot [fill=purple!60] table [y=$o_6$, meta=Label, x expr=\coordindex] {\testdata};
%DIFDELCMD < 

%DIFDELCMD <       
%DIFDELCMD <       \addlegendentry{prior}
%DIFDELCMD <       \addlegendentry{$o_1$}
%DIFDELCMD <       \addlegendentry{$o_2$}
%DIFDELCMD <       \addlegendentry{$o_3$}
%DIFDELCMD <       \addlegendentry{$o_4$}
%DIFDELCMD <       \addlegendentry{$o_5$}
%DIFDELCMD <       \addlegendentry{$o_6$}   
%DIFDELCMD <     \end{axis}
%DIFDELCMD <   \end{tikzpicture}
%DIFDELCMD <   \caption{Posterior precision per eigenvector, after taking a
%DIFDELCMD <     measurement $\meas_i$. Each $\meas_i$ increases precision
%DIFDELCMD <     uniformly across the lowest precision eigenvectors. We see that
%DIFDELCMD <     $\meas_2$ and $\meas_3$ are repeated measurements, as well as
%DIFDELCMD <     $\meas_5$ and $\meas_6$.}
%DIFDELCMD <   \label{fig:clusterization}
%DIFDELCMD < \end{figure}
%DIFDELCMD < %%%
\DIFdelend with the norm constraints on $\meas_j,j=1,\dots,m$. \DIFdelbegin \DIFdel{We show }\DIFdelend \DIFaddbegin \DIFadd{This implies that
$\meas_i \in span\{\ev_j\}_{j=1}^k,i=1,\dots, m$. The problem of
D-optimal design has now been cast to finite dimensions. This
justifies using linear algebra. Lemma \ref{lemma:free} (proved }\DIFaddend in the
appendix\DIFdelbegin \DIFdel{(Lemma \ref{lemma:free}) we can find $\meas_1,\dots,\meas_m$
such that$\{\eta_i\}_{i=1}^k$ have any value we desire, subject only
to the restriction that we do not change the trace of $\obs^*\obs$, so that $\sum_{i=1}^k \eta_i = m$. }\DIFdelend \DIFaddbegin \DIFadd{) does exactly that.
}\begin{restatable*}[Unit norm decomposition]{lemma}{free}\label{lemma:free}
  Let $M \in \R^{k \times k}$ symmetric positive definite with $\ttr M
  = m$, $m \geq k$. We can find $\func_j \in \R^k,j=1,\dots,m$
  with $\|\func_j\|=1$ and $A = (\func_1,\dots,\func_m)$ such that
  $AA^t = M$.
\end{restatable*}

\DIFadd{Take any $\{\eta_i\}_{i=1}^{k}$, with $\sum_{i=1}^k \eta_i = m$. Let
$M = \diag(\eta_1,\dots,\eta_k)$. The lemma gives us $A$ with unit
norm columns such that $AA^T = M$. Think of $A^t$ as representing
$\obs$, so we take $\meas_i :=\sum_{j=1}^k A_{ji}\ev_j^*$. The
constraints $\|\meas_i\| = 1, i=1,\dots,m$ are satisified by
construction. Verifying $\obs^*\obs \ev_i = \eta_i \ev_i$ is not too
difficult:
}

\begin{align*}
  \begin{split}
    \obs \ev_i &= (\meas_1,\dots,\meas_k)^t\ev_i\\
    &= (\sum_{j=1}^kA_{j1}\ev_j^*,\dots,\sum_{j=1}^kA_{jk}\ev_j^*)^t\ev_i\\
    &= (\sum_{j=1}^kA_{j1}\ev_j^*\ev_i,\dots,\sum_{j=1}^kA_{jk}\ev_j^*\ev_i)^t\\
    &= (A_{i1},\dots,A_{ik})^t.
  \end{split}
\end{align*}

\DIFadd{Now, 
}

\begin{align*}
  \begin{split}
    \obs^*\obs \ev_i &= \obs^*(A_{i1},\dots,A_{ik})^t\\
    &=(A_{i1},\dots,A_{ik}) \obs \text{ (by \eqref{eq:obs*})}\\
    &=(A_{i1},\dots,A_{ik}) (\sum_{j=1}^kA_{j1}\ev_j^*,\dots,\sum_{j=1}^kA_{jk}\ev_j^*)^t\\
    &=\sum_{l=1}^k A_{il} \sum_{j=1}^kA_{jl}\ev_j^* \\
    %&=\sum_{j=1}^k  \sum_{l=1}^kA_{il}A_{jl}\ev_j^* \\
    &=\sum_{j=1}^k  [AA^t]_{ij}\ev_j^* \\
    %&=\eta_i\ev_i^*\\
    &=\eta_i\ev_i^*. 
  \end{split}
\end{align*}


\DIFaddend By concavity of $\log$, the fastest increase in the design criterion
(equation \eqref{eq:true target}) is gained by increasing
\DIFdelbegin \DIFdel{$\lambda_i^{-1} + \sigma^{-2}\eta_i$ }\DIFdelend \DIFaddbegin \DIFadd{$\sigma^2\lambda_i^{-1} + \eta_i$ }\DIFaddend where it is smallest. So, no weight
should be given to (i.e. measurement taken \DIFaddbegin \DIFadd{in direction }\DIFaddend of) the \DIFdelbegin \DIFdel{$k$}\DIFdelend \DIFaddbegin \DIFadd{$n$}\DIFaddend th
eigenvector before \DIFdelbegin \DIFdel{$\lambda_i^{-1} + \sigma^{-2}\eta_i \geq
\lambda_k^{-1},\ \forall i \neq k$}\DIFdelend \DIFaddbegin \DIFadd{$\sigma^2\lambda_i^{-1} + \eta_i \geq
\sigma^2\lambda_n^{-1},\ \forall i \neq n$}\DIFaddend . Thus, we should choose
$\{\eta_i\}_{i=1}^k$ as follows. \DIFdelbegin \DIFdel{We would increase $\lambda_1^{-1} +
\sigma^{-2} \eta_1$ }\DIFdelend \DIFaddbegin \DIFadd{First, order the eigenvalues
$\lambda_1 \geq \lambda_2 \geq \dots$. We should increase $\sigma^2
\lambda_1^{-1} + \eta_1$ }\DIFaddend until it equals
\DIFdelbegin \DIFdel{$\lambda_2^{-1}$}\DIFdelend \DIFaddbegin \DIFadd{$\sigma^2\lambda_2^{-1}$}\DIFaddend . Then, \DIFdelbegin \DIFdel{$\lambda_i^{-1} + \sigma^{-2} \eta_i,i=1,2$ }\DIFdelend \DIFaddbegin \DIFadd{$\sigma^2 \lambda_i^{-1} +
\eta_i,i=1,2$ }\DIFaddend are increased until \DIFdelbegin \DIFdel{$\lambda_i^{-1} + \sigma^{-2} \eta_i = \lambda_3^{-1},i=1,2$ }\DIFdelend \DIFaddbegin \DIFadd{$\sigma^2 \lambda_i^{-1} + \eta_i =
\sigma^2\lambda_3^{-1},i=1,2$ }\DIFaddend and so forth.

\DIFaddbegin \begin{figure}%{r}{0.25\textwidth} 
    %\centering
    \includegraphics[width=5cm, height=5cm]{cylinders.jpg}
    \caption{Three graduated lab cylinders, corresponding to three
      eigenvectors. Prior eigenvalues not shown.}
    \label{fig:cylinder}
\end{figure}

\DIFaddend The argument above has the important consequence \DIFaddbegin \DIFadd{(illustrated in
figure \ref{fig:optimal vs not}) }\DIFaddend that the eigenvalues in \DIFaddbegin \DIFadd{the ``weird''
eigenvalue problem }\DIFaddend \eqref{eq:eigenproblem} are all equal! This makes
perfect sense. \DIFdelbegin \DIFdel{First, we }\DIFdelend \DIFaddbegin \DIFadd{We }\DIFaddend know the Lagrange multipliers \DIFaddbegin \DIFadd{$\xi_i,i=1,\dots,k$
}\DIFaddend tell us how much we gain by relaxing the constraints and there is no
reason for any \DIFdelbegin \DIFdel{measurement }\DIFdelend \DIFaddbegin \DIFadd{eigenvector }\DIFaddend to give us more than any other (otherwise
we would put more weight on the better \DIFdelbegin \DIFdel{measurement}\DIFdelend \DIFaddbegin \DIFadd{eigenvector}\DIFaddend , as can be
understood from the discussion in the following paragraph).
\DIFdelbegin \DIFdel{We are also used to symmetric
optimization problems having symmetric solutions \mbox{%DIFAUXCMD
\cite{Waterhouse83}}\hspace{0pt}%DIFAUXCMD
, so symmetric and identical Lagrange multipliers definitely make sense
in this context}\DIFdelend \DIFaddbegin 


\DIFadd{One can understand D-optimal designs as follows: Imagine each
eigenvector is a graduated lab cylinder (figure
\ref{fig:cylinder}). Each cylinder $i$ is filled, a-priori, with green
liquid of $\sigma^2\lambda_i^{-1}$ volume units. This liquid is the
prior precision, weighted by observation noise. The volume of the
green liquid is increasing in $i$. A D-optimal design with $m$
measurements ignores cylinders indexed $m+1$ and above. It has blue
liquid of $m$ volume units at its disposal. It distributes said liquid
by repeatedly adding a drop to whatever cylinder currently has the
lowest level of liquid in it (lowest precision). The result of such a
procedure is illustrated in figure \ref{fig:optimal vs not}. The
analysis conducted in this section is summarized in theorem
\ref{thm:char} below}\DIFaddend .


\DIFaddbegin \begin{theorem}[Characterizing D-optimal designs]\label{thm:char}
  Let:
  \begin{enumerate}
  \item $\fwd, \prcov, \obs$ the forward, prior covariance and
    observation operators, respectively.
  \item $\{\lambda_i\}_{i=1}^\infty$ the eigenvalues of
    $\fwd\prcov\fwd^*$ in decreasing order.
  \item $\{\ev_i\}_{i=1}^\infty$ the corresponding eigenvectors.
  \item $\{\eta_i\}_{i=1}^\infty$ eigenvalues of $\obs^*\obs$, where
    $\obs$ maximizes $\tar(\obs)$.
  \item $m$ the number of measurements taken.
  \item $\sigma^2$ observation noise variance.
  \end{enumerate}
  Then:
  \begin{enumerate}
  \item $\obs^*\obs$ and $\fwd\prcov\fwd^*$ are simultaneously diagonalizable.
  \item There exists $0 < k \leq m$ such that $\eta_i = 0$ for $i\geq k+1$.
  \item $\sigma^2\lambda_i^{-1} + \eta_i$ are equal for all $1 \leq i \leq k$.
  \item If $k < m$, then $\sigma^2\lambda_i^{-1} + \eta_i \leq
    \sigma^2\lambda_{k+1}^{-1}, i=1,\dots,k$.
  \end{enumerate}
\end{theorem}


\pgfplotstableread{
  Label     prior  optimal  sub-optimal 
  1         0.2    1.8           1.7
  2         0.8    1.2           0.8
  3         2.2    0             0.5
  4         3.5    0             0.0
}\optimalvsnot

\begin{figure}
  \begin{tikzpicture}[scale=0.85]
    \begin{axis}[
        ybar stacked,
        ymin=0,
        ymax=4,
        xtick=data,
        legend style={cells={anchor=east}, legend pos=north west, legend columns=-1},
        reverse legend=false, % set to false to get correct display, but I'd like to have this true
        xticklabels from table={\optimalvsnot}{Label},
        xticklabel style={text width=2cm,align=center},
        legend plot pos=right,
        ylabel=precision --- prior and posterior,
        xlabel=eigenvector $i$,
      ]

      
      \addplot [fill=green!80]  table [y=prior,   meta=Label, x expr=\coordindex] {\optimalvsnot};
      \addplot [fill=blue!60]   table [y=optimal, meta=Label, x expr=\coordindex] {\optimalvsnot};

      \addlegendentry{$\sigma^2\lambda_i^{-1}$}
      \addlegendentry{optimal $\eta_i$s}

    \end{axis}
  \end{tikzpicture}
  \qquad
  \begin{tikzpicture}[scale=0.85]
    \begin{axis}[
        ybar stacked,
        ymin=0,
        ymax=4,
        xtick=data,
        legend style={cells={anchor=east}, legend pos=north west, legend columns=-1},
        reverse legend=false, % set to false to get correct display, but I'd like to have this true
        xticklabels from table={\optimalvsnot}{Label},
        xticklabel style={text width=2cm,align=center},
        legend plot pos=right,
        ylabel=precision --- prior and posterior,
        xlabel=eigenvector $i$,
      ]   

      \addplot [fill=green!80]  table [y=prior,       meta=Label, x expr=\coordindex] {\optimalvsnot};
      \addplot [fill=blue!60]   table [y=sub-optimal, meta=Label, x expr=\coordindex] {\optimalvsnot};

      \addlegendentry{$\sigma^2\lambda_i^{-1}$}
      \addlegendentry{sub-optimal $\eta_i$s}

    \end{axis}
  \end{tikzpicture}
  \caption{Characterizing an optimal design, compared to a sub-optimal
    design. Both designs are allowed three measurements, which means
    the blue area has accumulated height of $3$ in both plots. An
    optimal design (left) increases precision (reduces uncertainty)
    where it is lowest. A sub-optimal design (right) does not. In this
    example, a measurement is taken partly in the direction of the
    third eigenvector, where precision is bigger even compared to the
    posterior precision of the optimal design on the left.}
  \label{fig:optimal vs not}
\end{figure}



\pgfplotstableread{
  Label    prior  $o_1$  $o_2$   $o_3$   $o_4$   $o_5$         $o_6$    topper
  1         0.2    0.8    0.5     0.5    0.4    0.33333333  0.3333333  0.001
  2         0.8    0.2    0.5     0.5    0.4    0.33333333  0.3333333  0.001
  3         2.2    0      0       0      0.2    0.33333333  0.3333333  0.001
  4         3.5    0      0       0      0      0           0          0.001
}\clusterization


\pgfplotstableread{
  Label    prior  $o_1$  $o_2$   $o_3$   $o_4$   $o_5$         $o_6$    topper
  1         0.2    0.8    0.7     0.3    0.4    0.2         0.4666666  0.001
  2         0.8    0.2    0.3     0.7    0.4    0.46666666  0.2        0.001
  3         2.2    0      0       0      0.2    0.33333333  0.3333333  0.001
  4         3.5    0      0       0      0      0           0          0.001
}\noclusterization








\section{\DIFadd{Sensor Clusterization --- Vanishing Model Error}}\label{section:clusterization}
\subsection{\DIFadd{Overview}}
\DIFadd{In this section we use theorem \ref{thm:char} to explain how sensor
clusterization may arise in various scenarios. We first consider
sequential designs, and then simultaneous designs.
}



\DIFaddend \subsection{\DIFdelbegin \DIFdel{Sensor Clusterization}\DIFdelend \DIFaddbegin \DIFadd{Sequential Design}\DIFaddend }\DIFdelbegin %DIFDELCMD < \label{subsec:clusterization}
%DIFDELCMD < %%%
\DIFdelend \DIFaddbegin \label{subsec:clusterization sequential}
\DIFadd{Recall the sequential optimal design scheme from section
\ref{subsec:seq vs sim} and the definition of $\obs_k$ from
}\eqref{eq:def obs_k}\DIFadd{. }\DIFaddend The way clusterization can occur \DIFdelbegin \DIFdel{is best understood from figure
\ref{fig:clusterization}. We see that each measurement increases
precision (reduces uncertainty) only for the eigenvectors of
lowest
precision . If the precision in these is lower than the precision for the next eigenvector, a repeated measurement is optimal .
}\DIFdelend \DIFaddbegin \DIFadd{in sequential
design scenario is understood by specializing thorem \ref{thm:char} to
one measurement each time. In this case $m=1$, so $k=1$, $\meas_1
\parallel \ev_1$ and $\eta_1 = 1$. As hinted in section
\ref{subsec:seq vs sim}, the posterior becomes the prior for the next
step. Similar reasoning leads to:
}\begin{align*}
  \begin{split}
    \meas_2 \parallel \ev_1  &\text{ if } \sigma^2 \lambda^{-1}_1 + 1 < \sigma^2\lambda^{-1}_2 \\
    \meas_2 \parallel \ev_2  &\text{ o.w. }
  \end{split}
\end{align*}
\DIFadd{In the first case, we get clusterization. In the second, we consider $\meas_3$:
}\begin{align*}
  \begin{split}
    \meas_3 \parallel \ev_1  &\text{ if } \sigma^2 \lambda^{-1}_1 + 1 < \sigma^2\lambda^{-1}_3 \\
    %\meas_3 \parallel  \ev_2  &\text{ if } \\
    \meas_3 \parallel  \ev_3  &\text{ o.w. }
  \end{split}
\end{align*}
\DIFadd{Again, the first case results in clusterization. The sequential design
proceeds in this fashion. Every $\meas_i$ is taken in the direction of
the eigenvector of smallest eigenvalue of the previous posterior
precision operator. Taking enough measurements we will surely arrive
at clusterization, since the prior precision increases quickly --- in
section \ref{subsec:abstract OED} we assumed $\fwd$ is strongly
smoothing, which means eigenvalues of $(\fwd \prcov\fwd^*)^{-1}$ grow
quickly.
}\DIFaddend 


\DIFdelbegin \section{\DIFdel{Analysis of Optimal Designs --- Non-Vanishing Model Error}}
%DIFAUXCMD
\addtocounter{section}{-1}%DIFAUXCMD
\DIFdel{Denote $\obs = (\meas_1,\dots,\meas_m)^t$ and $\obsm :=
(\meas_1,\dots,\meas_{m-1})^t$. The following corollary is
proved in the appendix.
}%DIFDELCMD < \begin{corollary}[Proved as corollary \ref{cor:same meas} in the appendix]
%DIFDELCMD <   If $\meas_m = \meas_j$ for some $1 \leq j \leq m-1$, then
%DIFDELCMD <   \begin{displaymath}
%DIFDELCMD <     \tar(\obs) - \tar(\obsm) =
%DIFDELCMD <     \log \left ( 1 + \frac{\sigma^2
%DIFDELCMD <       \langle \fwd \postcovm \fwd^* \obsm^* \Sigmam^{-1} e_j,
%DIFDELCMD <       \obsm^* \Sigmam^{-1}e_j \rangle
%DIFDELCMD <     }{
%DIFDELCMD <       2 - \sigma^2 e_j^t\Sigmam^{-1}e_j 
%DIFDELCMD <     }       
%DIFDELCMD <     \right ).
%DIFDELCMD <   \end{displaymath}
%DIFDELCMD < \end{corollary}
%DIFDELCMD < %%%
\DIFdelend \DIFaddbegin \subsection{\DIFadd{Simultaneous Design}}\label{subsec:clusterization simultaneous}
\DIFadd{In a simultaneous design scenario, the result is not as clear cut as
in the sequential design case. We still rely on previous results ---
namely corollary \ref{cor:same ev} and theorem \ref{thm:char}. An
illustration of two D-optimal designs scenarios for the same problem
are presented in figure \ref{fig:clusterization} --- the first
exhibits clusterization and second does not. As was discussed earlier
and illustrated in figure \ref{fig:optimal vs not}, an optimal design
makes the precision ``flat'' where it is lowest. Clusterization may
arise when two or more eigenvectors have equal precision. This
happens, for example for eigenvectors $1$ and $2$ when considering
measurements $\meas_i,i=1,2,3$. Taking three measurements, it is
possible that $\meas_1$ is such that it equalizes the precision of
eigenvectors $1$ and $2$. In this case, taking mesurements $\meas_2 =
\meas_3 = \frac{1}{2}(\ev_1 + \ev_2)$ leads to an optimal design that
exhibits clusterization. Similarly, taking $\meas_5 = \meas_6 =
\frac{1}{3}(\ev_1 + \ev_2 + \ev_3)$ also gives rise to an optimal
design that exhibits clusterization. It is important to note that
there are (infinitely) many optimal designs that actually do not
exhibit clusterization --- e.g. the second design presented in figure
\ref{fig:clusterization}. In the following section we postulate what
causes such designs to be observed in realistic experimental design
scenarios.
}\DIFaddend 





\DIFdelbegin \DIFdel{In contrast to the previous case of vanishing model error , we }\DIFdelend \DIFaddbegin \begin{figure}
  \begin{tikzpicture}[scale=0.85]
    \begin{axis}[
        ybar stacked,
        ymin=0,
        ymax=4,
        xtick=data,
        legend style={cells={anchor=east}, legend pos=north west, legend columns=-1},
        reverse legend=false, % set to false to get correct display, but I'd like to have this true
        xticklabels from table={\clusterization}{Label},
        xticklabel style={text width=2cm,align=center},
        legend plot pos=right,
        ylabel=precision --- prior and posterior,
        xlabel=eigenvector,
      ]

      
      \addplot [fill=green!80]  table [y=prior, meta=Label, x expr=\coordindex] {\clusterization};
      \addplot [fill=blue!60]   table [y=$o_1$, meta=Label, x expr=\coordindex] {\clusterization};
      \addplot [fill=red!60]    table [y=$o_2$, meta=Label, x expr=\coordindex] {\clusterization};
      \addplot [fill=black!60]  table [y=$o_3$, meta=Label, x expr=\coordindex] {\clusterization};
      \addplot [fill=orange!60] table [y=$o_4$, meta=Label, x expr=\coordindex] {\clusterization};
      \addplot [fill=cyan!60]   table [y=$o_5$, meta=Label, x expr=\coordindex] {\clusterization};
      \addplot [fill=purple!60] table [y=$o_6$, meta=Label, x expr=\coordindex] {\clusterization};

      
      \addlegendentry{prior}
      \addlegendentry{$o_1$}
      \addlegendentry{$o_2$}
      \addlegendentry{$o_3$}
      \addlegendentry{$o_4$}
      \addlegendentry{$o_5$}
      \addlegendentry{$o_6$}   
    \end{axis}
  \end{tikzpicture}
  \qquad
  \begin{tikzpicture}[scale=0.85]
    \begin{axis}[
        ybar stacked,
        ymin=0,
        ymax=4,
        xtick=data,
        legend style={cells={anchor=east}, legend pos=north west, legend columns=-1},
        reverse legend=false, % set to false to get correct display, but I'd like to have this true
        xticklabels from table={\noclusterization}{Label},
        xticklabel style={text width=2cm,align=center},
        legend plot pos=right,
        ylabel=precision --- prior and posterior,
        xlabel=eigenvector,
      ]

      
      \addplot [fill=green!80]  table [y=prior, meta=Label, x expr=\coordindex] {\noclusterization};
      \addplot [fill=blue!60]   table [y=$o_1$, meta=Label, x expr=\coordindex] {\noclusterization};
      \addplot [fill=red!60]    table [y=$o_2$, meta=Label, x expr=\coordindex] {\noclusterization};
      \addplot [fill=black!60]  table [y=$o_3$, meta=Label, x expr=\coordindex] {\noclusterization};
      \addplot [fill=orange!60] table [y=$o_4$, meta=Label, x expr=\coordindex] {\noclusterization};
      \addplot [fill=cyan!60]   table [y=$o_5$, meta=Label, x expr=\coordindex] {\noclusterization};
      \addplot [fill=purple!60] table [y=$o_6$, meta=Label, x expr=\coordindex] {\noclusterization};

      
      \addlegendentry{prior}
      \addlegendentry{$o_1$}
      \addlegendentry{$o_2$}
      \addlegendentry{$o_3$}
      \addlegendentry{$o_4$}
      \addlegendentry{$o_5$}
      \addlegendentry{$o_6$}   
    \end{axis}
  \end{tikzpicture}
  \caption{Clusterization and non-clusterization in simultaneous
    design. Posterior precision per eigenvector, after taking a
    measurement $\meas_i$. Each $\meas_i$ increases precision
    uniformly across the lowest precision eigenvectors. We see that
    $\meas_2$ and $\meas_3$ are repeated measurements, as well as
    $\meas_5$ and $\meas_6$.}
  \label{fig:clusterization}
\end{figure}







%DIF > %%%%%%%%%%%%%%%%%%%%%%%%%%%%%%%%%%%%%%%%%%%%%%%%%%%%%%%%%%%%%%%
%DIF > % SECTION Analysis of Optimal Designs --- Non-Vanishing Model Error
%DIF > %%%%%%%%%%%%%%%%%%%%%%%%%%%%%%%%%%%%%%%%%%%%%%%%%%%%%%%%%%%%%%%
\section{\DIFadd{Analysis of Optimal Designs --- Non-Vanishing Model Error}}\label{section:non vanishing}
\DIFadd{In this section we show the effect model error has on the
clusterization phenomenon. We }\DIFaddend will see that if $\modcov \neq 0$
clustering will not occur. \DIFdelbegin \DIFdel{From Corollary \ref{cor:same meas} we can observe that as $\sigma^2 \to 0$, the
}\DIFdelend \DIFaddbegin \DIFadd{This is contrary to the previous case of
vanishing model error.
}

\DIFadd{Corollary \ref{cor:same meas} below is proved in the appendix. Denote
$\obs = (\meas_1,\dots,\meas_m)^t$ and $\obsm :=
(\meas_1,\dots,\meas_{m-1})^t$. Denote $\Sigmam := \Sigma (\obsm)$ and
$\postcovm$ the posterior covariance that arises when we use $\obsm$.
}\begin{restatable*}{corollary}{samemeas}\label{cor:same meas}
  If $\meas_m = \meas_j$ for some $1 \leq j \leq m-1$, then
  \begin{equation*}
    \tar(\obs) - \tar(\obsm) =
    \log \left ( 1 + \frac{\sigma^2
      \langle \fwd \postcovm \fwd^* \obsm^* \Sigmam^{-1} e_j,
      \obsm^* \Sigmam^{-1}e_j \rangle
    }{
      2 - \sigma^2 e_j^t\Sigmam^{-1}e_j 
    }       
    \right ).
  \end{equation*}
\end{restatable*}
\DIFadd{Recall from }\eqref{eq:Sigma} \DIFadd{that $\Sigma(\obs) = \obs \modcov \obs^*
+ \sigma^2I$. Then
}

$$
\lim_{\sigma^2 \to 0} \tar(\obs) -\tar(\obsm) = 0
$$

\DIFadd{Hence no }\DIFaddend increase in the design criterion \DIFdelbegin \DIFdel{gained }\DIFdelend \DIFaddbegin \DIFadd{is achieved }\DIFaddend by taking a
\DIFdelbegin \DIFdel{measurement
identical to a previous one approaches zero. We may conclude that }\DIFdelend \DIFaddbegin \DIFadd{repeated measurement (in the limit $\sigma^2 \to 0$). This is clearly
sub-optimal and $\obsm$ cannot be an optimal design. Thus, }\DIFaddend for small
measurement error levels, the clusterization effect will be mitigated
by the presence of a non-zero model error. Since the design criterion
is not defined for $\sigma^2 = 0$ and identical measurements, we
cannot make a statement regarding $\sigma^2 = 0$, except in the
limiting sense described above.


\DIFdelbegin \section{\DIFdel{Conclusion}}%DIFAUXCMD
\addtocounter{section}{-1}%DIFAUXCMD
%DIFDELCMD < \label{section:conclusion}
%DIFDELCMD < %%%
\DIFdel{Our relaxed model gives us insight to D-optimal designs. We see that
in our setting, uncertainty is not reduced in any direction
(eigenvector) before it is the largest uncertainty present. We show
how and why repeated measurements can give rise to a D-optimal
design. There is more work to be done in understanding exactly what
causes such designs to be strictly better than others.
}\DIFdelend %DIF > % \section{Conclusion and Future Research Directions}\label{section:conclusion}
%DIF > % Our relaxed model gives us insight to D-optimal designs. We see that
%DIF > % in our setting, uncertainty is not reduced in any direction
%DIF > % (eigenvector) before it is the largest uncertainty present. We show
%DIF > % how and why repeated measurements can give rise to a D-optimal
%DIF > % design. There is more work to be done in understanding exactly what
%DIF > % causes such designs to be strictly better than others.


\section{Acknowledgements}
This study is a part of my PhD thesis\DIFdelbegin \DIFdel{\mbox{%DIFAUXCMD
\cite{mine} }\hspace{0pt}%DIFAUXCMD
which }\DIFdelend \DIFaddbegin \DIFadd{. It }\DIFaddend was written under the
instruction of Prof. Georg Stadler in New York University's Courant
Institute. I would like to thank him for his great mentorship.

\appendix
\section{\DIFdelbegin \DIFdel{Widely Applicable }\DIFdelend \DIFaddbegin \DIFadd{General }\DIFaddend Lemmas}

The following lemma is generalized from \cite[Chapter 9, Theorem 4, pp. 127]{Lax97}
\DIFdelbegin %DIFDELCMD < \begin{lemma}\label{lemma:lax}
%DIFDELCMD <   Let $Y(t)$ be a differentiable operator-valued function. Assume 
%DIFDELCMD <   $I+Y(t)$ is invertible, $Y(t)$ self-adjoint and trace-class. Then
%DIFDELCMD <   \begin{displaymath}
%DIFDELCMD <     \frac{\der \log \det (I+Y(t))}{\der t} = \tr{ (I+Y(t))^{-1} \dot{Y}(t)}.
%DIFDELCMD <   \end{displaymath}
%DIFDELCMD < \end{lemma}
%DIFDELCMD < \begin{proof}
%DIFDELCMD <   Consider a differentiable operator-valued function $X(t)$ such that
%DIFDELCMD <   $X(0) = 0$ and $X(t)$ is positive, self-adjoint and trace-class for
%DIFDELCMD <   every $t\in \R$. We denote the eigenvalues of this operator by
%DIFDELCMD <   $\lambda_k(X(t))$ and sometimes drop the dependence on $X(t)$, so
%DIFDELCMD <   $\lambda_k = \lambda_k(X(t))$.  Then $\det (I+X(t)) =
%DIFDELCMD <   \prod_{k=1}^{\infty} (1+\lambda_k) < \infty$ where the bound holds
%DIFDELCMD <   by the arguments given in \cite{AlexanderianGloorGhattas14}. The
%DIFDELCMD <   full derivative is
%DIFDELCMD <   \begin{align*}
%DIFDELCMD <     \frac{\der \det (I+X(t))}{\der t} 
%DIFDELCMD <     % 
%DIFDELCMD <     % 
%DIFDELCMD <     % 
%DIFDELCMD <     &= \sum_{k=1}^{\infty} 
%DIFDELCMD <     \frac{\partial \det (I+X(s))}{\partial (1+\lambda_k)}\Big |_{s=t}
%DIFDELCMD <     \frac{\der (1+\lambda_k)}{\der t} \\
%DIFDELCMD <     % 
%DIFDELCMD <     %
%DIFDELCMD <     %
%DIFDELCMD <     &= \sum_{k=1}^{\infty} \frac{\partial \prod_{l=1}^{\infty}
%DIFDELCMD <       (1+\lambda_l(s))}{\partial (1+\lambda_k)}\Big |_{s=t}
%DIFDELCMD <     \frac{\der (1+\lambda_k)}{\der t} \\
%DIFDELCMD <     %
%DIFDELCMD <     %
%DIFDELCMD <     %
%DIFDELCMD <     &= \sum_{k=1}^{\infty} \frac{\prod_{l=1}^{\infty}
%DIFDELCMD <       (1+\lambda_l(s))}{(1+\lambda_k)}\Big |_{s=t}
%DIFDELCMD <     \dot{\lambda_k}(X(t)) \\
%DIFDELCMD <     % 
%DIFDELCMD <     % 
%DIFDELCMD <     % 
%DIFDELCMD <     &= \sum_{k=1}^{\infty} \frac{\det (I+X(t))}{1 +\lambda_k} \dot{\lambda_k}(X(t)).
%DIFDELCMD <   \end{align*}
%DIFDELCMD <   The assumption $X(0) = 0$ means $\lambda_k(X(0)) = 0,\ \forall k \geq 1$. Thus:
%DIFDELCMD <   \begin{align*}
%DIFDELCMD <     \frac{\der (I+\det X(t))}{\der t}\Big |_{t=0}
%DIFDELCMD <     = \sum_{k=1}^{\infty} \dot{\lambda_k}(X(0))
%DIFDELCMD <     = \frac{\der }{\der t}\tr X(0)
%DIFDELCMD <     = \ttr \dot{X}(0),
%DIFDELCMD <   \end{align*}
%DIFDELCMD <   where the second equality follows by monotone convergence. 
%DIFDELCMD <   Let $Y(t)$ a trace-class self-adjoint operator such that 
%DIFDELCMD <   $I+Y(t)$ is invertible.
%DIFDELCMD <   Define $X(t)$ via $I+X(t) = (I+Y(0))^{-1/2} (I+Y(t)) (I+Y(0))^{-1/2}$. 
%DIFDELCMD <   We show $X(t)$ satisfies the conditions above. It is trace-class:
%DIFDELCMD <   \begin{align*}
%DIFDELCMD <     \ttr X(t) = \tr{(I+Y(0))^{-1} (I+Y(t)) - I}
%DIFDELCMD <     \leq \tr{I+Y(t) - I} < \infty,
%DIFDELCMD <   \end{align*}
%DIFDELCMD <   since $Y(t)$ is trace-class. It is also clear that
%DIFDELCMD <   $X(0) = 0$ and $X(t)$ is self-adjoint.
%DIFDELCMD <   $I+Y(t) = (I+Y(0))^{1/2}(I+X(t))(I+Y(0))^{1/2}$, so
%DIFDELCMD <   \begin{align*}
%DIFDELCMD <     \frac{\der \det (I+Y(t))}{\der t}|_{t=0} 
%DIFDELCMD <     &= \det (I+Y(0))\frac{\der \det (I+X(t))}{\der t}\Big |_{t=0} \\
%DIFDELCMD <     % 
%DIFDELCMD <     % 
%DIFDELCMD <     % 
%DIFDELCMD <     &= \det (I+Y(0)) \ttr \dot{X}(0) \\
%DIFDELCMD <     % 
%DIFDELCMD <     % 
%DIFDELCMD <     % 
%DIFDELCMD <     &= \det (I+Y(0)) \tr (I+Y(0))^{-1} \dot{Y}(0).
%DIFDELCMD <   \end{align*}
%DIFDELCMD <   Consequently, by the one-variable chain rule:
%DIFDELCMD <   \begin{align*}
%DIFDELCMD <     \frac{\der \log \det (I+Y(t))}{\der t}\Big |_{t=0} &=
%DIFDELCMD <     % 
%DIFDELCMD <     % 
%DIFDELCMD <     % 
%DIFDELCMD <     \frac{1}{\det (I+Y(0))}\frac{\der \det (I+Y(t))}{\der t}\Big |_{t=0} \\ 
%DIFDELCMD <     % 
%DIFDELCMD <     % 
%DIFDELCMD <     % 
%DIFDELCMD <     &= \tr (I+Y(t))^{-1} \dot{Y}(t) \big |_{t=0}.
%DIFDELCMD <   \end{align*}
%DIFDELCMD <   Since there is nothing special about $t=0$, the relation holds for all $t$.
%DIFDELCMD < \end{proof}
%DIFDELCMD < %%%
\DIFdelend \DIFaddbegin \lax
\begin{proof}
  Consider a differentiable operator-valued function $X(t)$ such that
  $X(0) = 0$ and $X(t)$ is positive, self-adjoint and trace-class for
  every $t\in \R$. We denote the eigenvalues of this operator by
  $\lambda_k(X(t))$ and sometimes drop the dependence on $X(t)$, so
  $\lambda_k = \lambda_k(X(t))$.  Then $\det (I+X(t)) =
  \prod_{k=1}^{\infty} (1+\lambda_k) < \infty$ where the bound holds
  by the arguments given in \cite{AlexanderianGloorGhattas14}. The
  full derivative is
  \begin{align*}
    \frac{\der \det (I+X(t))}{\der t} 
    % 
    % 
    % 
    &= \sum_{k=1}^{\infty} 
    \frac{\partial \det (I+X(s))}{\partial (1+\lambda_k)}\Big |_{s=t}
    \frac{\der (1+\lambda_k)}{\der t} \\
    % 
    %
    %
    &= \sum_{k=1}^{\infty} \frac{\partial \prod_{l=1}^{\infty}
      (1+\lambda_l(s))}{\partial (1+\lambda_k)}\Big |_{s=t}
    \frac{\der (1+\lambda_k)}{\der t} \\
    %
    %
    %
    &= \sum_{k=1}^{\infty} \prod_{l=1, l\neq k}^{\infty}
      (1+\lambda_l(s)) \frac{\partial (1+\lambda_k(s))}{\partial (1+\lambda_k)}\Big |_{s=t}
    \frac{\der (1+\lambda_k)}{\der t} \\
    %
    %
    %    
    &= \sum_{k=1}^{\infty} \frac{\prod_{l=1}^{\infty}
      (1+\lambda_l(s))}{(1+\lambda_k)}\Big |_{s=t}
    \dot{\lambda_k}(X(t)) \\
    % 
    % 
    % 
    &= \sum_{k=1}^{\infty} \frac{\det (I+X(t))}{1 +\lambda_k} \dot{\lambda_k}(X(t)).
  \end{align*}
  The assumption $X(0) = 0$ means $\lambda_k(X(0)) = 0,\ \forall k \geq 1$. Thus:
  \begin{align*}
    \frac{\der (I+\det X(t))}{\der t}\Big |_{t=0} 
    = \sum_{k=1}^{\infty} \dot{\lambda_k}(X(0)) 
    = \frac{\der }{\der t}\tr{X(0)}
    = \tr{\dot{X}(0)},
  \end{align*}
  where the second equality follows by monotone convergence. 
  Let $Y(t)$ a trace-class self-adjoint operator such that 
  $I+Y(t)$ is invertible.
  Define $X(t)$ via $I+X(t) = (I+Y(0))^{-1/2} (I+Y(t)) (I+Y(0))^{-1/2}$. 
  We show $X(t)$ satisfies the conditions above. It is trace-class:
  \begin{align*}
    \tr{X(t)} = \tr{(I+Y(0))^{-1} (I+Y(t)) - I}
    \leq \tr{I+Y(t) - I}< \infty,
  \end{align*}
  since $Y(t)$ is trace-class. It is also clear that
  $X(0) = 0$ and $X(t)$ is self-adjoint.
  $I+Y(t) = (I+Y(0))^{1/2}(I+X(t))(I+Y(0))^{1/2}$, so
  \begin{align*}
    \frac{\der \det (I+Y(t))}{\der t}|_{t=0} 
    &= \det (I+Y(0))\frac{\der \det (I+X(t))}{\der t}\Big |_{t=0} \\
    % 
    % 
    % 
    &= \det (I+Y(0)) \tr{\dot{X}(0)} \\
    % 
    % 
    % 
    &= \det (I+Y(0)) \tr{(I+Y(0))^{-1} \dot{Y}(0)}.
  \end{align*}
  Consequently, by the one-variable chain rule:
  \begin{align*}
    \frac{\der \log \det (I+Y(t))}{\der t}\Big |_{t=0} &=
    % 
    % 
    % 
    \frac{1}{\det (I+Y(0))}\frac{\der \det (I+Y(t))}{\der t}\Big |_{t=0} \\ 
    % 
    % 
    % 
    &= \tr{ (I+Y(t))^{-1} \dot{Y}(t)} \big |_{t=0}.
  \end{align*}
  There is nothing special about $t_0 = 0$ --- we could have chosen
  any other $t_0$ instead. Thus, the relation holds for all $t$.
\end{proof}
\DIFaddend 


\DIFdelbegin %DIFDELCMD < \begin{lemma}[Matrix Determinant Lemma in Hilbert Spaces]\label{lemma:MDL}
%DIFDELCMD <   Let $\hil$ a separable Hilbert space, $u,v\in \hil$ and $A: \hil \to
%DIFDELCMD <   \hil$ an invertible linear operator such that $\tr A-I <
%DIFDELCMD <   \infty$. Then $\det A$ and $\det A + uv^*$ are well defined and
%DIFDELCMD <   \begin{displaymath}
%DIFDELCMD <     \det (A + uv^*) = (1 + \langle A^{-1} u, v \rangle ) \det A,
%DIFDELCMD <   \end{displaymath}
%DIFDELCMD <   where $(A + uv^*)w := Aw + \langle v,w \rangle u$.
%DIFDELCMD < \end{lemma}
%DIFDELCMD < \begin{proof}
%DIFDELCMD <   In this proof we rely on definitions and results from
%DIFDELCMD <   \cite{Simon77}. First, consider $B := I + xy^*$ for some $x,y \in
%DIFDELCMD <   \hil$. We construct an eigenbasis for $B$ and use that to show $\det
%DIFDELCMD <   B = 1 + \langle x, y \rangle$. First let $x_1 := x$.  Now, if $x
%DIFDELCMD <   \parallel y$, take $\{x_n \}_{n=2}^{\infty}$ an orthogonal basis for
%DIFDELCMD <   $span\{x_1\} ^{\perp}$. If, on the other hand, $x \nparallel y$, let
%DIFDELCMD <   \begin{displaymath}
%DIFDELCMD <     x_2 := x - \frac{ \langle x, y\rangle}{\|y\|^2}y
%DIFDELCMD <   \end{displaymath}
%DIFDELCMD <   and it is easy to verify that $x_2 \perp y$ and $span \{x,y\} = span
%DIFDELCMD <   \{x_1,x_2\}$. Take $\{x_n \}_{n=3}^{\infty}$ an orthogonal basis for
%DIFDELCMD <   $span\{x_1,x_2\} ^{\perp}$. In both cases,
%DIFDELCMD <   \begin{displaymath}
%DIFDELCMD <     B x_n =
%DIFDELCMD <     \begin{cases}
%DIFDELCMD <       (1 + \langle x, y \rangle) x_n & n = 1 \\
%DIFDELCMD <       x_n                            & n \neq 1,
%DIFDELCMD <     \end{cases}
%DIFDELCMD <   \end{displaymath}
%DIFDELCMD <   and so $\det B = 1 + \langle x, y \rangle$.
%DIFDELCMD <   

%DIFDELCMD <   It is easy to verify that $uv^*$ is trace-class and since $\tr A - I
%DIFDELCMD <   < \infty$, also $\tr A + uv^* - I < \infty$ (sum of two trace-class
%DIFDELCMD <   operators is trace-class). Thus $\det A$ and $\det (A+uv^*)$ are
%DIFDELCMD <   well defined. Let $x:=A^{-1}u$ and $y := v$:
%DIFDELCMD <   \begin{displaymath}
%DIFDELCMD <     \det (A + uv^*) = \det A \ \det(I+A^{-1}uv^*) =
%DIFDELCMD <     (1 + \langle A^{-1}u, v \rangle) \det A .
%DIFDELCMD <   \end{displaymath}
%DIFDELCMD < \end{proof}
%DIFDELCMD < %%%
\DIFdelend \DIFaddbegin \begin{lemma}[Matrix Determinant Lemma in Hilbert Spaces]\label{lemma:MDL}
  Let $\hil$ a separable Hilbert space, $u,v\in \hil$ and $A: \hil \to
  \hil$ an invertible linear operator such that $\tr{A-I} <
  \infty$. Then $\det A$ and $\det A + uv^*$ are well defined and
  \begin{equation*}
    \det (A + uv^*) = (1 + \langle A^{-1} u, v \rangle ) \det A,
  \end{equation*}
  where $(A + uv^*)w := Aw + \langle v,w \rangle u$.
\end{lemma}
\begin{proof}
  In this proof we rely on definitions and results from
  \cite{Simon77}. First, consider $B := I + xy^*$ for some $x,y \in
  \hil$. We construct an eigenbasis for $B$ and use that to show $\det
  B = 1 + \langle x, y \rangle$. First let $x_1 := x$.  Now, if $x
  \parallel y$, take $\{x_n \}_{n=2}^{\infty}$ an orthogonal basis for
  $span\{x_1\} ^{\perp}$. If, on the other hand, $x \nparallel y$, let
  \begin{equation*}
    x_2 := x - \frac{ \langle x, y\rangle}{\|y\|^2}y
  \end{equation*}
  and it is easy to verify that $x_2 \perp y$ and $span \{x,y\} = span
  \{x_1,x_2\}$. Take $\{x_n \}_{n=3}^{\infty}$ an orthogonal basis for
  $span\{x_1,x_2\} ^{\perp}$. In both cases,
  \begin{equation*}
    B x_n =
    \begin{cases}
      (1 + \langle x, y \rangle) x_n & n = 1 \\
      x_n                            & n \neq 1,
    \end{cases}
  \end{equation*}
  and so $\det B = 1 + \langle x, y \rangle$.

  It is easy to verify that $uv^*$ is trace-class and since $\tr{A-I}
  < \infty$, also $\tr{A + uv^* - I} < \infty$ (sum of two trace-class
  operators is trace-class). Thus $\det A$ and $\det (A+uv^*)$ are
  well defined. Let $x:=A^{-1}u$ and $y := v$:
  \begin{equation*}
    \det (A + uv^*) = \det A \ \det(I+A^{-1}uv^*) =
    (1 + \langle A^{-1}u, v \rangle) \det A .
  \end{equation*}
\end{proof}


\free
\begin{proof}
  Let us diagonalize $M$, so that $M = U D U^t$ with $D =
  \diag(d_1,\dots,d_k)$ and $U \in \R^{k \times k }$ orthogonal. Let
  $S \in \R^{k \times m}$ with $S_{ii} = \sqrt{d_{i}}$ and zeros
  otherwise. Define $A:= U S V^t$, where $V \in \R^{m \times m}$ is
  orthogonal and will be further restricted later. Then $AA^t = U
  SV^tVS^t U^t = UDU^t$, so $AA^t$ has the required eigenvalues and
  eigenvectors by construction. If we can choose $V$ such that $A$
  also satisfies the unit norm constraints we are done. These
  constraints are, for $j=1,\dots,m$:
  \begin{equation}\label{eq:V constraints}
   1 = [A^tA]_{jj} = [V S^tS V^t]_{jj},
  \end{equation}
  and we can expect to do this since we assumed $\ttr D = m$.

  Define $C = S^tS - I \in \R^{m \times m}$. Note that $\ttr C = 0$ and
  $C$ is diagonal with non-zero entries $d_i-1,i=1,\dots,k$. It suffices
  to find $V$ orthogonal such that $V C V^t$ has zero diagonal. We
  construct such $V$ by sequentially inserting zeros in the diagonal
  and not destroying zeros we already introduced, starting from the
  last diagonal entry and moving to the first. Since $c_{mm} \neq 0$ ,
  let $p < m$ such that $c_{pp}c_{mm} < 0$ (such $p$ exists because
  the trace is zero) and let $\theta \in (0,\pi)$. Define a Givens
  rotation $R^{(m)} \in \R^{m \times m}$ by
  \begin{equation*}
    r^{(m)}_{ab} :=
    \begin{cases}
      1 & a = b \neq p \text{ or } a = b \neq m \\
      \cos \theta & a = b = p  \\
     -\sin \theta & a = p, b = m\\
      \cos \theta & a = b = m \\
      \sin \theta & a = m, b = p \\ 
      0 & o.w
    \end{cases}
  \end{equation*}
  Note that conjugating a matrix by $R^{(m)}$ changes only its $m$ and
  $p$ rows and columns. We want to choose $\theta$ such that
  \begin{equation}\label{eq:mm}
    0 = [R^{(m)} C (R^{(m)})^t]_{mm} = \cos^2 \theta c_{mm} + 2\cos \theta \sin
    \theta c_{mp} + \sin^2\theta c_{pp},
  \end{equation}
  and it suffices to choose $\theta$ such that
  \begin{equation*}
    c_{mm} \cot^2 \theta + 2 c_{mp} \cot \theta + c_{pp} = 0.
  \end{equation*}
  This quadratic in $\cot\theta$ has a real solution, since
  $c_{pp}c_{mm} < 0$ by assumption and we can find $\theta \in
  (0,\pi)$ such that \eqref{eq:mm} is satisfied. We continue to find
  $R^{(m-1)}$ that leaves row and column $m$ unchanged and
  continue introducing zeros to the diagonal. The assumption $\ttr D =
  m \Rightarrow \ttr C = 0$ guarantees we can do that. Taking $V:=
  R^{(1)} R^{(2)} \dots R^{(m-1)}R^{(m)}$ completes the proof.
\end{proof}


\simdiag
\begin{proof}
  First, enumerate the eigenvalues of $C + \sum_{j=1}^m
  \func_j\func_j^*$ as $\xi_1,\dots,\xi_\ell$. Denote the
  indices of the eigenvectors corresponding to $\xi_i$
  \begin{equation*}
    S_i := \{ 1 \leq k \leq m | (C + \sum_{j=1}^m \func_j\func_j^* )\func_k = \xi_i \func_k \}.
  \end{equation*}
  Define further
  \begin{equation*}
    A_i := \sum_{k \in S_i} \func_k \func_k^*,
  \end{equation*}
  which is self-adjoint. Two observations are in order. First,
  $\sum_{j=1}^m \func_j\func_j^* = \sum_{i=1}^\ell A_i$. Second, $A_i
  \func_k = 0$ if $k\not \in S_i$, since eigenvectors of different
  eigenvalue are orthogonal. For $k \in S_i$
  \begin{equation}\label{eq:on vi}
    \xi_i \func_k = (C + \sum_{j=1}^m \func_j \func_j^* ) \func_k = (C + A_i) \func_k.
  \end{equation}
  Let $V_i := span \{\func_k \}_{k\in S_i}$. Observe that $V_i$ is
  invariant under $A_i$, by definition, and under $C$, by \eqref{eq:on
    vi}. Using \eqref{eq:on vi} again, we conclude that on $V_i$, $A_i
  = \xi_iI - C$. This immediately implies $A_i$ and $C$ have the
  same eigenvectors on $V_i$. This holds for every $1 \leq i \leq
  \ell$ and we conclude that $C$ and $A$ have the same eigenvectors.
\end{proof}
\DIFaddend 


\section{Specific Lemmas}

%DIF > %%%%%%%%%%%%%%%%%%%%%%%%%%%%%%%%%%%%%%%%%%%%%%%%%%%%%%%%%%%%%%%%%%%%%%%%
\DIFaddbegin \begin{lemma}\label{lemma:twice woodbury}
  Assume $\fwd \prcov \fwd^*$ is invertible. Then
\begin{align*}
  \begin{split}
    \fwd( \prcov^{-1} + \sigma^{-2}  \fwd^* \obs^* \obs \fwd )^{-1} \fwd^* 
    %
    %
    = \left ( (\fwd\prcov\fwd^*)^{-1} + \sigma^{-2}  \obs^* \obs \right )^{-1},
  \end{split}
\end{align*}  
\end{lemma}
\begin{proof}
  The proof is boring and amounts to using Woodbury's matrix identity
  twice and a regularization trick. The standard proof for this
  identity works in infinite dimensions, as long as all terms are well
  defined. Unfortunately, $\obs^*\obs$ is not invertible, so we force
  it to be. Throughout, we let $X := \fwd\prcov \fwd^*$ and $Y_{\eps}
  := \obs^*\obs + \eps I$, for some small $\eps > 0$. Note that now
  $Y_{\eps}$ is invertible. Recall Woodbury's matrix identity:
  $$
  (A + UCV)^{-1} = A^{-1} + A^{-1}U(C^{-1} + VA^{-1}U)^{-1}VA^{-1}.
  $$

  Taking $A := \prcov^{-1}, U := \fwd^*, V := \fwd$ and $C :=
  \sigma^{-2} (\obs^*\obs+\eps I)$ we get:
  \begin{align*}
    \begin{split}
      \fwd( \prcov^{-1} + \sigma^{-2}  \fwd^* (\obs^* \obs +\eps I) \fwd )^{-1}\fwd^* &=
      \fwd ( \prcov - \prcov \fwd^* ( \sigma^2(\obs^*\obs + \eps I)^{-1} + \fwd \prcov \fwd^* )^{-1} \fwd \prcov ) \fwd^* \\
      %
      %
      &= X - X(\sigma^2Y_{\eps}^{-1} + X)^{-1}X
    \end{split}
  \end{align*}
  Note that $X + \sigma^2 Y_{\eps}^{-1}$ is invertible, as the sum of
  a two positive definite operators. Now, taking $A := X^{-1}, C :=
  \sigma^2Y_{\eps}^{-1}, U := I$ and $V := I$ and using Wodbury's matrix
  identity in reverse order:
  \begin{align*}
    \begin{split}
      X - X(\sigma^2Y_{\eps}^{-1} + X)^{-1}X &= (X^{-1} + \sigma^{-2}Y_{\eps})^{-1} \\
      %
      %
      %
      &= (\fwd \prcov \fwd^* + \sigma^{-2} (\obs^*\obs + \eps I))^{-1}.
    \end{split}
  \end{align*}

  We conclude that $\forall \eps > 0$
  \begin{align*}
    \begin{split}
      \fwd( \prcov^{-1} + \sigma^{-2}  \fwd^* (\obs^* \obs +\eps I) \fwd )^{-1}\fwd^* 
     &= (\fwd \prcov \fwd^* + \sigma^{-2} (\obs^*\obs + \eps I))^{-1}.
    \end{split}
  \end{align*}
  Letting $\eps \to 0$ completes the proof.
\end{proof}



%DIF > %%%%%%%%%%%%%%%%%%%%%%%%%%%%%%%%%%%%%%%%%%%%%%%%%%%%%%%%%%%%%%%%%%%%%%%%
\DIFaddend \begin{lemma}[Increase due to a Measurement]\label{lemma:design increase}
  Let $\obs = (\meas_1,\dots,\meas_m)^t$ and $\obsm := (\meas_1,\dots,\meas_{m-1})^t$. Then
  \begin{align*}
    \tar( \obs ) - \tar (\obsm ) &=
    \frac12 \log \left ( 1 + \frac{
      \langle \fwd \postcovm \fwd^* (\obsm^* \Sigmam^{-1} \modcov - I ) \meas_m,
      (\obsm^* \Sigmam^{-1} \modcov - I ) \meas_m \rangle
    }{
      \sigma^2 + \meas_m \modcov \meas_m - \meas_m \modcov \obsm^* \Sigmam^{-1} \obsm \modcov \meas_m 
    }       
    \right ).
  \end{align*}
\end{lemma}
\begin{proof}
  We use the Schur complement to write one inverse in terms of the other and
  introduce notations to make the derivation cleaner. Note that we think of
  $\obsm$ and $\obsm^*$ as column and row vectors (respectively).
  \begin{align*}
    \Sigma( \obs ) &= \Sigma = 
    \begin{bmatrix}
      \Sigma (\obsm )           & \obsm \modcov \meas_m \\
      \meas_m \modcov \obsm^*   & \sigma^2 + \meas_m \modcov \meas_m
    \end{bmatrix}
    =
    \begin{bmatrix}
      \Sigmam   & w \\
      w^t       & c
    \end{bmatrix}\\
    %
    %
    %
    \Sigma^{-1} &=
    \begin{bmatrix}
      \Sigmam^{-1} + \Sigmam^{-1} w ( c - w^t \Sigmam^{-1} w)^{-1} w^t \Sigmam^{-1} & - \Sigmam^{-1} w ( c - w^t \Sigmam^{-1} w)^{-1} \\
      -( c - w^t \Sigmam^{-1} w)^{-1} w^t \Sigmam^{-1}                            &  ( c - w^t \Sigmam^{-1} w)^{-1}
    \end{bmatrix} \\
    &=
    \begin{bmatrix}
      \Sigmam^{-1} & 0 \\
      0           & 0 
    \end{bmatrix}
    + (c -w^t \Sigmam^{-1} w )^{-1}
    \begin{bmatrix}
      \Sigmam^{-1} w \\
      -1
    \end{bmatrix}
    \begin{bmatrix}
      w^t \Sigmam^{-1} & -1 
    \end{bmatrix}
  \end{align*}
  %
  Further, define
  %
  \begin{align*}
    \M (\obs ):&= \prcov^{\frac12}\fwd^* \obs^* \Sigma^{-1} \obs \fwd
    \prcov^{\frac12}    
  \end{align*}
  %
  and note that, using our understanding of what is a column vector and
  what is a row vector:
  %
  \begin{align*}
    \M(\obs) &= \prcov^{1/2} \fwd^* \obs^* \Sigma^{-1} \obs \fwd \prcov^{1/2} \\
    %
    %
    %
    &= \prcov^{1/2} \fwd^* \obs^* \left \{
    \begin{bmatrix}
      \Sigmam^{-1} & 0 \\
      0           & 0 
    \end{bmatrix}
    + (c -w^t \Sigmam^{-1} w )^{-1}
    \begin{bmatrix}
      \Sigmam^{-1} w \\
      -1
    \end{bmatrix}
    \begin{bmatrix}
      w^t \Sigmam^{-1} & -1 
    \end{bmatrix} 
    \right \} \obs \fwd \prcov^{1/2} \\
    %
    %
    %
    &= \M (\obsm) + (c -w^t \Sigmam^{-1} w )^{-1}
    \prcov^{1/2} \fwd^* \obs^*
    \begin{bmatrix}
      \Sigmam^{-1} w \\
      -1
    \end{bmatrix}
    \begin{bmatrix}
      w^t \Sigmam^{-1} & -1 
    \end{bmatrix} 
    \obs \fwd \prcov^{1/2}
  \end{align*}
  %
  Now, denote:
  %
  \begin{align}\label{eq:u}
    \begin{split}
      u :&= (c -w^t \Sigmam^{-1} w )^{-1/2}
      \prcov^{1/2} \fwd^* \obs^* 
      \begin{bmatrix}
        \Sigmam^{-1} w \\
        -1 
      \end{bmatrix} \\
      %
      %
      %
      & = (c -w^t \Sigmam^{-1} w )^{-1/2} ( \prcov^{1/2}\fwd^* \obsm^* \Sigmam^{-1} \obsm  \modcov \meas_m - \prcov^{1/2} \fwd^* \meas_m )\\
      %
      %
      %
      u^* :&=  (c -w^t \Sigmam^{-1} w )^{-1/2} (\meas_m \modcov \obsm^* \Sigmam^{-1} \obsm \fwd \prcov^{1/2} - \meas_m \fwd \prcov^{1/2} ),
    \end{split}
  \end{align}
  %
  so that
  %
  \begin{equation}\label{eq:M plus I}
    I + \M( \obs ) = I + \M (\obsm ) + uu^*.
  \end{equation}
  %
  Note that
  \begin{equation}\label{eq:M postcov}
    \prcov^{1/2} \left (I + \M( \obsm ) \right )^{-1} \prcov^{1/2} = \postcovm.
  \end{equation}
  The increase in the design criterion gained by including $\meas_m$
  in the design is found by using Lemma \ref{lemma:MDL} the above
  results:
  %
  \begin{align*}
    \tar( \obs ) - \tar( \obsm )
    %
    %
    %
    &= \frac12 \log \det \Big ( I + \M ( \obs ) \Big ) / \det \Big ( I + \M (\obsm) \Big ) \\
    %
    %
    %
    &= \frac12  \log \det \left ( I + \M(\obsm) + uu^* \right ) / \det \Big ( I + \M (\obsm) \Big ) \\
    %
    %
    %
    &= \frac12 \log \left ( 1 + \left \langle \left ( I+\M(\obsm) \right )^{-1} u, u  \right \rangle \right ).
  \end{align*}
  Using \eqref{eq:u}:
  \begin{align*}
    &\left \langle \left (I+\M (\obsm)\right )^{-1}u, u \right \rangle\\
    &= \frac{
      \langle \fwd \postcovm \fwd^* (\obsm^* \Sigmam^{-1} \obsm \modcov - I ) \meas_m,
      (\obsm^* \Sigmam^{-1} \obsm \modcov - I ) \meas_m \rangle
    }{
      c- w^t \Sigmam^{-1} w
    }\\
    %
    %
    %
    &= 
    \frac{
      \langle \fwd \postcovm \fwd^* (\obsm^* \Sigmam^{-1} \obsm \modcov - I ) \meas_m,
      (\obsm^* \Sigmam^{-1} \obsm \modcov - I ) \meas_m \rangle
    }{
      \sigma^2 + \meas_m \modcov \meas_m - \meas_m \modcov \obsm^* \Sigmam^{-1} \obsm \modcov \meas_m 
    }
  \end{align*}
  and the conclusion follows.
\end{proof}

Lemma \ref{lemma:design increase} implies the following two corollaries.
\begin{corollary}[Gain for No Model Error]\label{cor:zero mod err}
  If $\modcov = 0$, then
  \begin{equation*}
    \tar( \obs ) - \tar (\obsm )
    = -\frac12 \log (1 - \sigma^{-2} \langle \fwd \postcov \fwd^* \meas_m, \meas_m \rangle ).
  \end{equation*}
\end{corollary}
\begin{proof}
  Note that this is not immediate by substituting $\modcov = 0$ in the
  conclusion of Lemma \ref{lemma:design increase}, since we make a
  claim for $\postcov$, and not $\postcovm$. Let us first review
  \eqref{eq:u} and note that since $\modcov = 0$ the covariance
  $\Sigmam = \sigma^2I_{m-1}$ and $w = 0$, so $c - w^t\Sigmam^{-1}w =
  \sigma^2$:
  \begin{align*}
    u :&=-\sigma^{-1}\prcov^{1/2} \fwd^* \meas_m\\
    %
    %
    u^* :&= -\sigma^{-1} \meas_m \fwd \prcov^{1/2}.
  \end{align*}
  From \eqref{eq:M plus I}:
  \begin{equation*}
    I + \M(\obsm) = I +\M(\obs) - uu^*,
  \end{equation*}
  and thus:
  \begin{equation*}
    \left \langle \left ( I +\M(\obs) \right )^{-1} u, u \right \rangle
    = \sigma^{-2} \langle \fwd \postcov \fwd^* \meas_m, \meas_m \rangle.
  \end{equation*}
  Analogously to \eqref{eq:M postcov} we note that
  \begin{equation*}
    \prcov^{1/2} \left ( I + \M(\obs) \right )^{-1} \prcov^{1/2} = \postcov.
  \end{equation*}
  Using Lemma \ref{lemma:MDL} we conclude
  \begin{align*}
    \tar( \obs ) - \tar( \obs )
    &= \frac12 \log \det \left (I +\M(\obs) \right ) / \det \left (I + \M(\obsm) \right ) \\
    %
    %
    %
    &= \frac12 \log \det \left (I +\M(\obs) \right ) / \det \left (I + \M(\obs) - uu^* \right ) \\
    %
    %
    %
    &=-\frac12 \log (1 - \langle (I+\M(\obs))^{-1}u, u \rangle \\
    %
    %
    %
    &= -\frac12 \log (1 - \sigma^{-2} \langle \fwd \postcov \fwd^* \meas_m, \meas_m \rangle ).
  \end{align*}
\end{proof}

\DIFdelbegin %DIFDELCMD < \begin{corollary}[Gain for Identical Measurement]\label{cor:same meas}
%DIFDELCMD <   If $\meas_m = \meas_j$ for some $1 \leq j \leq m-1$, then
%DIFDELCMD <   \begin{displaymath}
%DIFDELCMD <     \tar(\obs) - \tar(\obsm) =
%DIFDELCMD <     \log \left ( 1 + \frac{\sigma^2
%DIFDELCMD <       \langle \fwd \postcovm \fwd^* \obsm^* \Sigmam^{-1} e_j,
%DIFDELCMD <       \obsm^* \Sigmam^{-1}e_j \rangle
%DIFDELCMD <     }{
%DIFDELCMD <       2 - \sigma^2 e_j^t\Sigmam^{-1}e_j 
%DIFDELCMD <     }       
%DIFDELCMD <     \right ).
%DIFDELCMD <   \end{displaymath}
%DIFDELCMD < \end{corollary}
%DIFDELCMD < \begin{proof}
%DIFDELCMD <   Denote $A:= \obs \modcov \obs^*$ and $v_j$ the $j$th column of $A$.
%DIFDELCMD <   Note that $v_j = \obsm \modcov \meas_m$, since $(\obsm \modcov
%DIFDELCMD <   \obsm^*)_{ij} = \meas_i(\modcov \meas_j)$, as explained in
%DIFDELCMD <   \eqref{eq:modcov explained}. One can now verify that
%DIFDELCMD <   \begin{displaymath}\label{eq:observation}
%DIFDELCMD <     \Sigmam^{-1} \obsm \modcov \meas_m = \Sigmam^{-1}v_j = (A +\sigma^2I_{m-1})^{-1} v_j =
%DIFDELCMD <     e_j -\sigma^2 \Sigmam^{-1}e_j.
%DIFDELCMD <   \end{displaymath}
%DIFDELCMD <   %
%DIFDELCMD <   Using \eqref{eq:observation}:
%DIFDELCMD <   \begin{align*}\label{eq:denominator}
%DIFDELCMD <     \begin{split}
%DIFDELCMD <       \meas_m \modcov \obsm^* \Sigmam^{-1} \obsm \modcov \meas_m
%DIFDELCMD <       &= \meas_m \modcov \obsm^* ( e^j - \sigma^2 \Sigmam^{-1} e_j )\\
%DIFDELCMD <       %
%DIFDELCMD <       %
%DIFDELCMD <       %
%DIFDELCMD <       &= \meas_m \modcov \meas_j - \sigma^2 \meas_m \modcov \obsm^* \Sigmam^{-1}e_j \\
%DIFDELCMD <       %
%DIFDELCMD <       %
%DIFDELCMD <       %
%DIFDELCMD <       &= \meas_m \modcov \meas_j -\sigma^2 (e_j - \sigma^2 \Sigmam^{-1}e_j)^t e_j \\
%DIFDELCMD <       %
%DIFDELCMD <       %
%DIFDELCMD <       %
%DIFDELCMD <       &= \meas_m \modcov \meas_m -\sigma^2 + \sigma^4 e_j^t\Sigmam^{-1}e_j.
%DIFDELCMD <     \end{split}
%DIFDELCMD <   \end{align*}
%DIFDELCMD <   We use \eqref{eq:observation} to simplify the terms in the enumerator of
%DIFDELCMD <   the conclusion of Lemma \ref{lemma:design increase}:
%DIFDELCMD <   \begin{align*}\label{eq:enumerator}
%DIFDELCMD <     \begin{split}
%DIFDELCMD <       (\obsm^* \Sigmam^{-1} \obsm \modcov - I ) \meas_m
%DIFDELCMD <       &= \obsm^* \Sigmam^{-1} \obsm \modcov \meas_m - \meas_m \\
%DIFDELCMD <       %
%DIFDELCMD <       %
%DIFDELCMD <       %
%DIFDELCMD <       &= \obsm^* (e_j - \sigma^2 \Sigmam^{-1} e_j) -\meas_j \\ 
%DIFDELCMD <       %
%DIFDELCMD <       %
%DIFDELCMD <       %
%DIFDELCMD <       &= -\sigma^2 \obsm^* \Sigma^{-1}e_j. 
%DIFDELCMD <     \end{split}
%DIFDELCMD <   \end{align*}
%DIFDELCMD <   %
%DIFDELCMD <   Substitute \eqref{eq:enumerator} and \eqref{eq:denominator} to
%DIFDELCMD <   the enumerator and denominator (respectively) of the conclusion of
%DIFDELCMD <   Lemma \ref{lemma:design increase}:
%DIFDELCMD <   %
%DIFDELCMD <   \begin{align*}
%DIFDELCMD <     \tar( \obs ) - \tar (\obsm ) &=
%DIFDELCMD <     \log \left ( 1 + \frac{
%DIFDELCMD <       \langle \fwd \postcovm \fwd^* (\obsm^* \Sigmam^{-1} \modcov - I ) \meas_m,
%DIFDELCMD <       (\obsm^* \Sigmam^{-1} \modcov - I ) \meas_m \rangle
%DIFDELCMD <     }{
%DIFDELCMD <       \sigma^2 + \meas_m \modcov \meas_m - \meas_m \modcov \obsm^* \Sigmam^{-1} \obsm \modcov \meas_m 
%DIFDELCMD <     }       
%DIFDELCMD <     \right ) \\
%DIFDELCMD <     %
%DIFDELCMD <     %
%DIFDELCMD <     %
%DIFDELCMD <     &= \log \left ( 1 + \frac{\sigma^4
%DIFDELCMD <       \langle \fwd \postcovm \fwd^* \obsm^* \Sigmam^{-1} e_j,
%DIFDELCMD <       \obsm^* \Sigmam^{-1}e_j \rangle
%DIFDELCMD <     }{
%DIFDELCMD <       2\sigma^2 - \sigma^4 e_j^t\Sigmam^{-1}e_j 
%DIFDELCMD <     }       
%DIFDELCMD <     \right ) \\
%DIFDELCMD <     %
%DIFDELCMD <     %
%DIFDELCMD <     %
%DIFDELCMD <     &= \log \left ( 1 + \frac{\sigma^2
%DIFDELCMD <       \langle \fwd \postcovm \fwd^* \obsm^* \Sigmam^{-1} e_j,
%DIFDELCMD <       \obsm^* \Sigmam^{-1}e_j \rangle
%DIFDELCMD <     }{
%DIFDELCMD <       2 - \sigma^2 e_j^t\Sigmam^{-1}e_j 
%DIFDELCMD <     }       
%DIFDELCMD <     \right ).
%DIFDELCMD <   \end{align*}
%DIFDELCMD < \end{proof}
%DIFDELCMD < 

%DIFDELCMD < \begin{lemma}\label{lemma:free}
%DIFDELCMD <   Let $M \in \R^{k \times k}$ symmetric positive definite with $\tr M =
%DIFDELCMD <   m$, $m > k$. Then we can find $\func_j \in \R^k,j=1,\dots,m$ with
%DIFDELCMD <   $\|\func_j\|=1$ and $A = (\func_1,\dots,\func_m)$ such that $AA^t =
%DIFDELCMD <   M$.
%DIFDELCMD < \end{lemma}
%DIFDELCMD < \begin{proof}
%DIFDELCMD <   Let us diagonalize $M$, so that $M = U D U^t$ with $D =
%DIFDELCMD <   \diag(d_1,\dots,d_k)$ and $U \in \R^{k \times k }$ orthogonal. Let $S
%DIFDELCMD <   \in \R^{k \times m}$ with $S_{ii} = \sqrt{d_{i}}$ and zeros
%DIFDELCMD <   otherwise. Define $A:= U S V^t$, where $V \in \R^{m \times m}$ is
%DIFDELCMD <   orthogonal and will be further restricted later. Then $AA^t = U
%DIFDELCMD <   SV^tVS^t U^t = UDU^t$, so $AA^t$ has the required eigenvalues and
%DIFDELCMD <   eigenvectors by construction. If we can choose $V$ such that $A$
%DIFDELCMD <   also satisfies the unit norm constraints we are done. These
%DIFDELCMD <   constraints are, for $j=1,\dots,m$:
%DIFDELCMD <   \begin{displaymath}\label{eq:V constraints}
%DIFDELCMD <    1 = [A^tA]_{jj} = [V S^tS V^t]_{jj},
%DIFDELCMD <   \end{displaymath}
%DIFDELCMD <   and we can expect to do this since we assumed $\tr D = m$.
%DIFDELCMD < 

%DIFDELCMD <   Define $C = S^tS - I \in \R^{m \times m}$. Note that $\tr C = 0$ and
%DIFDELCMD <   $C$ is diagonal with non-zero entries $d_i-1,i=1,\dots,k$. It suffices
%DIFDELCMD <   to find $V$ orthogonal such that $V C V^t$ has zero diagonal. We
%DIFDELCMD <   construct such $V$ by sequentially inserting zeros in the diagonal
%DIFDELCMD <   and not destroying zeros we already introduced, starting from the
%DIFDELCMD <   last diagonal entry and moving to the first. If $c_{kk} \neq 0$ ,
%DIFDELCMD <   let $p < k$ such that $c_{pp}c_{kk} < 0$ (such $p$ exists because
%DIFDELCMD <   the trace is zero) and let $\theta \in (0,\pi)$. Define a Givens
%DIFDELCMD <   rotation $R^{(k)} \in \R^{m \times m}$ by
%DIFDELCMD <   \begin{displaymath}
%DIFDELCMD <     r^{(k)}_{ab} :=
%DIFDELCMD <     \begin{cases}
%DIFDELCMD <       1 & a = b \neq p \text{ or } a = b \neq k \\
%DIFDELCMD <       \cos \theta & a = b = p  \\
%DIFDELCMD <      -\sin \theta & a = p, b = k\\
%DIFDELCMD <       \cos \theta & a = b = k \\
%DIFDELCMD <       \sin \theta & a = k, b = p \\ 
%DIFDELCMD <       0 & o.w
%DIFDELCMD <     \end{cases}
%DIFDELCMD <   \end{displaymath}
%DIFDELCMD <   Note that conjugating a matrix by $R^{(k)}$ changes only its $k$ and
%DIFDELCMD <   $p$ rows and columns. Specifically, it leaves rows and columns
%DIFDELCMD <   $k+1,\dots,m$ unchanged. We want to choose $\theta$ such that
%DIFDELCMD <   \begin{displaymath}\label{eq:mm}
%DIFDELCMD <     0 = [R^{(k)} C (R^{(k)})^t]_{kk} = \cos^2 \theta c_{kk} + 2\cos \theta \sin
%DIFDELCMD <     \theta c_{kp} + \sin^2\theta c_{pp},
%DIFDELCMD <   \end{displaymath}
%DIFDELCMD <   and it suffices to choose $\theta$ such that
%DIFDELCMD <   \begin{displaymath}
%DIFDELCMD <     c_{kk} \cot^2 \theta + 2 c_{kp} \cot \theta + c_{pp} = 0.
%DIFDELCMD <   \end{displaymath}
%DIFDELCMD <   This quadratic in $\cot\theta$ has a real solution, since
%DIFDELCMD <   $c_{pp}c_{kk} < 0$ by assumption and we can find $\theta \in
%DIFDELCMD <   (0,\pi)$ such that \eqref{eq:mm} is satisfied. We continue to find
%DIFDELCMD <   $R^{(k-1)}$ that leaves rows and columns $k,\dots,m$ unchanged and
%DIFDELCMD <   continue introducing zeros to the diagonal. The assumption $\tr D =
%DIFDELCMD <   m \Rightarrow \tr C = 0$ guarantees we can do that. Taking $V:=
%DIFDELCMD <   R^{(1)} R^{(2)} \dots R^{(k-1)}R^{(k)}$ completes the proof.
%DIFDELCMD < \end{proof}
%DIFDELCMD < %%%
\DIFdelend \DIFaddbegin \samemeas
\begin{proof} \label{cor:same meas proof}
  Denote $A:= \obs \modcov \obs^*$ and $v_j$ the $j$th column of $A$.
  Note that $v_j = \obsm \modcov \meas_m$, since $(\obsm \modcov
  \obsm^*)_{ij} = \meas_i(\modcov \meas_j)$, as explained in
  \eqref{eq:modcov explained}. One can now verify that
  \begin{equation}\label{eq:observation}
    \Sigmam^{-1} \obsm \modcov \meas_m = \Sigmam^{-1}v_j = (A +\sigma^2I_{m-1})^{-1} v_j =
    e_j -\sigma^2 \Sigmam^{-1}e_j.
  \end{equation}
  %
  Using \eqref{eq:observation}:
  \begin{align}\label{eq:denominator}
    \begin{split}
      \meas_m \modcov \obsm^* \Sigmam^{-1} \obsm \modcov \meas_m
      &= \meas_m \modcov \obsm^* ( e^j - \sigma^2 \Sigmam^{-1} e_j )\\
      %
      %
      %
      &= \meas_m \modcov \meas_j - \sigma^2 \meas_m \modcov \obsm^* \Sigmam^{-1}e_j \\
      %
      %
      %
      &= \meas_m \modcov \meas_j -\sigma^2 (e_j - \sigma^2 \Sigmam^{-1}e_j)^t e_j \\
      %
      %
      %
      &= \meas_m \modcov \meas_m -\sigma^2 + \sigma^4 e_j^t\Sigmam^{-1}e_j.
    \end{split}
  \end{align}
  We use \eqref{eq:observation} to simplify the terms in the enumerator of
  the conclusion of Lemma \ref{lemma:design increase}:
  \begin{align}\label{eq:enumerator}
    \begin{split}
      (\obsm^* \Sigmam^{-1} \obsm \modcov - I ) \meas_m
      &= \obsm^* \Sigmam^{-1} \obsm \modcov \meas_m - \meas_m \\
      %
      %
      %
      &= \obsm^* (e_j - \sigma^2 \Sigmam^{-1} e_j) -\meas_j \\ 
      %
      %
      %
      &= -\sigma^2 \obsm^* \Sigma^{-1}e_j. 
    \end{split}
  \end{align}
  %
  Substitute \eqref{eq:enumerator} and \eqref{eq:denominator} to
  the enumerator and denominator (respectively) of the conclusion of
  Lemma \ref{lemma:design increase}:
  %
  \begin{align*}
    \tar( \obs ) - \tar (\obsm ) &=
    \log \left ( 1 + \frac{
      \langle \fwd \postcovm \fwd^* (\obsm^* \Sigmam^{-1} \modcov - I ) \meas_m,
      (\obsm^* \Sigmam^{-1} \modcov - I ) \meas_m \rangle
    }{
      \sigma^2 + \meas_m \modcov \meas_m - \meas_m \modcov \obsm^* \Sigmam^{-1} \obsm \modcov \meas_m 
    }       
    \right ) \\
    %
    %
    %
    &= \log \left ( 1 + \frac{\sigma^4
      \langle \fwd \postcovm \fwd^* \obsm^* \Sigmam^{-1} e_j,
      \obsm^* \Sigmam^{-1}e_j \rangle
    }{
      2\sigma^2 - \sigma^4 e_j^t\Sigmam^{-1}e_j 
    }       
    \right ) \\
    %
    %
    %
    &= \log \left ( 1 + \frac{\sigma^2
      \langle \fwd \postcovm \fwd^* \obsm^* \Sigmam^{-1} e_j,
      \obsm^* \Sigmam^{-1}e_j \rangle
    }{
      2 - \sigma^2 e_j^t\Sigmam^{-1}e_j 
    }       
    \right ).
  \end{align*}
\end{proof}
\DIFaddend 


\begin{lemma}[Auxilliary Calculations]\label{lemma:aux calc}
  Let $T(\obs) := \obs^* \Sigma^{-1}(\obs)\obs$, with $\Sigma(\obs)$
  defined as in \eqref{eq:Sigma}. Then:
  \begin{align*}
    \delta T(\obs)V &= V^* \Sigma^{-1} \obs 
    - \obs^*\Sigma^{-1} V\modcov \obs^* \Sigma^{-1}\obs \\
    &\ \ \ - \obs^* \Sigma^{-1} \obs \modcov V^* \Sigma^{-1}\obs
    + \obs^* \Sigma^{-1} V.
  \end{align*}
\end{lemma}

\begin{proof}
  We need a few supplementary calculations. First:
  \begin{align}\label{eq:der sig}
    \begin{split}
      \frac{\der}{\der \tau} \Big |_{\tau=0} \Sigma( \obs + \tau V )
      &= \frac{\der}{\der \tau} \Big |_{\tau=0} 
      (\obs + \tau V ) \modcov (\obs + \tau V )^*  + \sigma^2I\\
      % 
      % 
      % 
      &= V \modcov \obs^* + \obs \modcov V^*.
    \end{split}
  \end{align}
  By the standard trick for the derivative of an operator: 
  \begin{align*}
    0 &= \frac{\der}{\der \tau} \Big |_{\tau=0} I \\
    % 
    % 
    % 
    &= \frac{\der}{\der \tau} \Big |_{\tau=0}
    \left (\Sigma(\obs+\tau V)^{-1} \Sigma(\obs+\tau V) \right ) \\
    % 
    % 
    % 
    &= \frac{\der \Sigma(\obs+\tau V)^{-1}}{\der \tau} \Big |_{\tau=0} \Sigma+
    \Sigma^{-1} \frac{\der \Sigma(\obs+\tau V)}{\der \tau} \Big |_{\tau=0}\\  
    %
    %
    %
    &= \frac{\der \Sigma(\obs+\tau V)^{-1}}{\der \tau} \Big |_{\tau=0} \Sigma+
    \Sigma^{-1} (V\modcov \obs^* + \obs \modcov V^*) 
    \text{, by \eqref{eq:der sig}. }
  \end{align*}
  Now we can write the variation of $\Sigma^{-1}$:
  \begin{align}\label{eq:der sig inv}
    \frac{\der \Sigma(\obs+\tau V)^{-1}}{\der \tau} \Big |_{\tau=0}  
      &= -\Sigma^{-1} (V \modcov \obs^* + \obs \modcov V^*) \Sigma^{-1}.
    \end{align}
  % 
  Finally, we can do the calculation we need to. By Leibnitz (product) rule
  and \eqref{eq:der sig inv}:
  % 
  \begin{align*}
    \delta T(\obs) V 
    &= \frac{\der T(\obs + \tau V)}{\der \tau} \Big |_{\tau=0} \\
    %
    %
    %
    &= V^* \Sigma^{-1} \obs 
    - \obs^*\Sigma^{-1} V\modcov \obs^* \Sigma^{-1}\obs \\
    &\ \ \ - \obs^* \Sigma^{-1} \obs \modcov V^* \Sigma^{-1}\obs
    + \obs^* \Sigma^{-1} V,
  \end{align*}
  as required.
\end{proof}

\DIFdelbegin %DIFDELCMD < \begin{lemma}\label{lemma:sim diag}
%DIFDELCMD <   Let $C:\hil \to \hil$ self-adjoint and let $\func_1,\dots,\func_m \in
%DIFDELCMD <   \hil$. Denote $\func^*$ the element $\func$ acting as a
%DIFDELCMD <   functional. If
%DIFDELCMD <   \begin{displaymath}
%DIFDELCMD <    (C + \sum_{j=1}^m \func_j\func_j^*) \func_k = \xi_k \func_k, k = 1,\dots,m
%DIFDELCMD <   \end{displaymath}
%DIFDELCMD <   then $C$ and $\sum_{j=1}^m \func_j \func_j^*$ are simultaneously
%DIFDELCMD <   diagonalizable.
%DIFDELCMD < \end{lemma}
%DIFDELCMD < \begin{proof}
%DIFDELCMD <   First, enumerate the eigenvalues of $C + \sum_{j=1}^m
%DIFDELCMD <   \func_j\func_j^*$ as $\xi_1,\dots,\xi_\ell$. Denote the
%DIFDELCMD <   indices of the eigenvectors corresponding to $\xi_i$
%DIFDELCMD <   \begin{displaymath}
%DIFDELCMD <     S_i := \{ 1 \leq k \leq m | (C + \sum_{j=1}^m \func_j\func_j^* )\func_k = \xi_i \func_k \}.
%DIFDELCMD <   \end{displaymath}
%DIFDELCMD <   Define further
%DIFDELCMD <   \begin{displaymath}
%DIFDELCMD <     A_i := \sum_{k \in S_i} \func_k \func_k^*,
%DIFDELCMD <   \end{displaymath}
%DIFDELCMD <   which is self-adjoint. Two observations are in order. First,
%DIFDELCMD <   $\sum_{j=1}^m \func_j\func_j^* = \sum_{i=1}^\ell A_i$. Second, $A_i
%DIFDELCMD <   \func_k = 0$ if $k\not \in S_i$, since eigenvectors of different
%DIFDELCMD <   eigenvalue are orthogonal. For $k \in S_i$
%DIFDELCMD <   \begin{displaymath}\label{eq:on vi}
%DIFDELCMD <     \xi_i \func_k = (C + \sum_{j=1}^m \func_j \func_j^* ) \func_k = (C + A_i) \func_k.
%DIFDELCMD <   \end{displaymath}
%DIFDELCMD <   Let $V_i := span \{\func_k \}_{k\in S_i}$. Observe that $V_i$ is
%DIFDELCMD <   invariant under $A_i$, by definition, and under $C$, by \eqref{eq:on
%DIFDELCMD <     vi}. Using \eqref{eq:on vi} again, we conclude that on $V_i$, $A_i
%DIFDELCMD <   = \xi_iI - C$. This immediately implied $A_i$ and $C$ have the
%DIFDELCMD <   same eigenvectors on $V_i$. This holds for every $1 \leq i \leq
%DIFDELCMD <   \ell$ and we conclude that $C$ and $A$ have the same eigenvectors.
%DIFDELCMD < \end{proof}
%DIFDELCMD < 

%DIFDELCMD < \begin{proposition}\label{prop:bigger better}
%DIFDELCMD <   Let $\obs = (\meas_1,\dots,\meas_m)^t$, $j \in \{1,\dots,m\}$, $\sigma^2
%DIFDELCMD <   > 0$ and $\lambda > 1$. Then $\tar(\obs)$ increases if we use
%DIFDELCMD <   $\lambda \meas_j$ in $\obs$ instead of $\meas_j$.
%DIFDELCMD < \end{proposition}
%DIFDELCMD < \begin{proof} 
%DIFDELCMD <   Fix an arbitrary $j=1,\dots,m$ and take $V:= e_j e_j^t \obs$. We see
%DIFDELCMD <   that for $u \in \ban$
%DIFDELCMD <   \begin{displaymath}
%DIFDELCMD <     Vu = e_je_j^t (\meas_1(u),\dots,\meas_m(u) )^t = e_j \meas_j(u)
%DIFDELCMD <     = (0,\dots,0,\meas_j(u),0,\dots,0)^t.
%DIFDELCMD <   \end{displaymath}
%DIFDELCMD <   This way, $V$ has the same $j$th entry as $\obs$ while the rest are
%DIFDELCMD <   set to zero. We calculate the variation in this direction and show
%DIFDELCMD <   it is positive --- $\delta \tar(\obs)V > 0$. This means that
%DIFDELCMD <   increasing the magnitude of the $j$th measurement functional
%DIFDELCMD <   increases $\tar(\obs)$. We start with the last line of \eqref{eq:tar
%DIFDELCMD <     var}, and denote $\tmp : = \fwd \postcov \fwd^*: \ban^* \to \ban$:
%DIFDELCMD <   \begin{align*}
%DIFDELCMD <      \delta \tar(\obs) V 
%DIFDELCMD <     &= \tr V ( I - \modcov \obs^*\Sigma^{-1}\obs) \tmp \obs^* \Sigma^{-1} \\
%DIFDELCMD <     % 
%DIFDELCMD <     %
%DIFDELCMD <     %
%DIFDELCMD <     &= \tr e_je_j^t \obs ( I - \modcov \obs^*\Sigma^{-1}\obs) \tmp \obs^* \Sigma^{-1} \\
%DIFDELCMD <     %
%DIFDELCMD <     % 
%DIFDELCMD <     %
%DIFDELCMD <     &= e_j^t \obs ( I - \modcov \obs^*\Sigma^{-1}\obs) \tmp \obs^* \Sigma^{-1}e_j \\
%DIFDELCMD <     %
%DIFDELCMD <     % 
%DIFDELCMD <     %
%DIFDELCMD <     &= e_j^t ( I - \obs \modcov \obs^*\Sigma^{-1})\obs \tmp \obs^* \Sigma^{-1}e_j \\  
%DIFDELCMD <     % 
%DIFDELCMD <     %
%DIFDELCMD <     %
%DIFDELCMD <     &=  e_j^t(\Sigma-\obs \modcov \obs^*) \Sigma^{-1}\obs \tmp \obs^* \Sigma^{-1}e_j \\
%DIFDELCMD <     %
%DIFDELCMD <     %
%DIFDELCMD <     %
%DIFDELCMD <     &=\sigma^2 e_j^t \Sigma^{-1}\obs \tmp \obs^* \Sigma^{-1}e_j
%DIFDELCMD <     \text{ by \eqref{eq:Sigma} }\\
%DIFDELCMD <     %
%DIFDELCMD <     % 
%DIFDELCMD <     %
%DIFDELCMD <     &=\sigma^2 e_j^t \Sigma^{-1}\obs \fwd \postcov \fwd^* \obs^* \Sigma^{-1}e_j.
%DIFDELCMD <   \end{align*} 
%DIFDELCMD <   Since $\postcov$ is positive definite, $\delta \tar(\obs) V > 0$.
%DIFDELCMD < \end{proof}
%DIFDELCMD < %%%
\DIFdelend \DIFaddbegin \biggerbetter
\begin{proof} 
  Fix an arbitrary $j=1,\dots,m$ and take $V:= e_j e_j^t \obs$. We see
  that for $u \in \hilo$
  \begin{equation*}
    Vu = e_je_j^t (\meas_1(u),\dots,\meas_m(u) )^t = e_j \meas_j(u)
    = (0,\dots,0,\meas_j(u),0,\dots,0)^t.
  \end{equation*}
  This way, $V$ has the same $j$th entry as $\obs$ while the rest are
  set to zero. We calculate the variation in this direction and show
  it is positive --- $\delta \tar(\obs)V > 0$. This means that
  increasing the magnitude of the $j$th measurement functional
  increases $\tar(\obs)$. We start with the last line of \eqref{eq:tar
    var}, and denote $\tmp : = \fwd \postcov \fwd^*$.
  \begin{align*}
     \delta \tar(\obs) V 
    &= \tr{V ( I - \modcov \obs^*\Sigma^{-1}\obs) \tmp \obs^* \Sigma^{-1}} \\
    % 
    %
    %
    &= \tr{e_je_j^t \obs ( I - \modcov \obs^*\Sigma^{-1}\obs) \tmp \obs^* \Sigma^{-1}} \\
    %
    % 
    %
    &= e_j^t \obs ( I - \modcov \obs^*\Sigma^{-1}\obs) \tmp \obs^* \Sigma^{-1}e_j \\
    %
    % 
    %
    &= e_j^t ( I - \obs \modcov \obs^*\Sigma^{-1})\obs \tmp \obs^* \Sigma^{-1}e_j \\  
    % 
    %
    %
    &=  e_j^t(\Sigma-\obs \modcov \obs^*) \Sigma^{-1}\obs \tmp \obs^* \Sigma^{-1}e_j \\
    %
    %
    %
    &=\sigma^2 e_j^t \Sigma^{-1}\obs \tmp \obs^* \Sigma^{-1}e_j
    \text{ by \eqref{eq:Sigma} }\\
    %
    % 
    %
    &=\sigma^2 e_j^t \Sigma^{-1}\obs \fwd \postcov \fwd^* \obs^* \Sigma^{-1}e_j.
  \end{align*} 
  Since $\postcov$ is positive definite, $\delta \tar(\obs) V > 0$.
\end{proof}
\DIFaddend 



%% HERE
%% Since a measurement that takes into account $\ker \fwd$ is
%% necessarily sub-optimal, we may assume $\fwd$ is in fact
%% invertible (it is, on $\ker \fwd ^{\perp}$) and rewrite the
%% necessary condition \eqref{eq:eigenproblem} using
%% \eqref{eq:postcov} as \begin{equation*} \sigma^{-2} \left ( (\fwd
%% \prcov \fwd^*)^{-1} + \sigma^{-2}\obs\obs^* \right )^{-1} =
%% \obs^* \Xi.  \end{equation*} By Lemma \ref{lemma:sim diag},
%% taking $C = (\fwd \prcov \fwd^*)^{-1}$ we observe that $\meas_j,
%% j=1,\dots,m$ are eigenvectors of $\fwd \prcov \fwd^*$. If they have
%% different eigenvalue, they are orthogonal. If not, this means, by
%% the common interpretation of the Lagrange multipliers, that an
%% infinitesimal increase in one will cause the same increase in
%% design criterion as others with the same eigenvalue. So if these
%% eigenvectors do not have the same angle between every pair, an
%% infinitesimal increase in one will give more increase than
%% others, which is a contradiction. So in every subspace, vectors
%% will be restricted to these forms and if there will be more
%% vectors than the dimensionality of the subspace, clusterization
%% will occur.










%% \begin{figure}
%%   \begin{tikzpicture}[thick, scale=1.4, every node/.style={scale=0.99}]
%%     \begin{axis}
%%       [
%%         xmin = 0,
%%         xmax = 3.14,
%%         xlabel = {$x$},
%%         ylabel = posterior std,
%%         ymin   = 0,
%%         %compat = 1.3,
%%         % ymax   = 130,
%%         % ytick = \empty,
%%         legend cell align=left,
%%         % legend style={at={(0.45,0.2)}}
%%         legend pos= outer north east 
%%       ]
%%       % \draw[black!30!white, thin] (50,0) -- (50,130);
%%       % 
%%       \addplot [thin, black, mark=none] table{data/stdv-trunc-sens1-var1.txt};
%%       \addlegendentry{1 sensor};

%%       \addplot [thin, blue, mark=none] table{data/stdv-trunc-sens2-var1.txt};
%%       \addlegendentry{2 sensors};

%%       \addplot [thin, red, mark=none] table{data/stdv-trunc-sens3-var1.txt};
%%       \addlegendentry{3 sensors};

%%       \addplot [thin, green, mark=none] table{data/stdv-trunc-sens4-var1.txt};
%%       \addlegendentry{4 sensors};

%%       \addplot [thin, purple, mark=none] table{data/stdv-trunc-sens5-var1.txt};
%%       \addlegendentry{5 sensors};

%%       \addplot [thin, cyan, mark=none] table{data/stdv-trunc-sens6-var1.txt};
%%       \addlegendentry{6 sensors};

%%       \addplot [black,  only marks, mark=x, mark size=1.5] 
%%       table{data/locs-trunc-sens1-var1.txt}; 
%%       \addplot [blue,   only marks, mark=x, mark size=1.5]
%%       table{data/locs-trunc-sens2-var1.txt}; 
%%       \addplot [red,    only marks, mark=x, mark size=1.5]
%%       table{data/locs-trunc-sens3-var1.txt};
%%       \addplot [green,  only marks, mark=x, mark size=1.5] 
%%       table{data/locs-trunc-sens4-var1.txt}; 
%%       \addplot [purple, only marks, mark=x, mark size=1.5] 
%%       table{data/locs-trunc-sens5-var1.txt}; 
%%       \addplot [cyan,   only marks, mark=x, mark size=1.5] 
%%       table{data/locs-trunc-sens6-var1.txt}; 
%%     \end{axis}
%%   \end{tikzpicture}
%%   \caption{Analogously to figure \ref{fig:clusterization
%%       illustration}, shown are posterior pointwise standard
%%     deviations. Note the exact clustering. Here the dynamics is not
%%     the one governed by the 1D heat equation but a dynamics that sets
%%     to zero all modes except the first three. This dynamics simulates
%%     a finite-rank operator as per the conditions of Theorem
%%     \ref{theorem:finite rank}. Although the conditions of the theorem
%%     allow the use of a different constraint on measurements (see
%%     section \ref{subsec:unit length}), the conclusion of the theorem
%%     still seems to hold.}
%%   \label{fig:finite rank illustration}
%% \end{figure}




%% \section{Counterexample}\label{section:counterexample}
%% Applying Lemma \ref{lemma:sim diag} to
%% \begin{equation*}
%%   (\fwd \postcov \fwd^*)^{-1} = (\fwd \prcov \fwd^*)^{-1} +
%%   \sigma^{-2}\obs^*\obs
%% \end{equation*}
%% implies $(\fwd \prcov \fwd^*)^{-1}$ and $\obs^*\obs$ have the same
%% eigenspaces. Since $\meas_j$ are eigenvectors of $\fwd \postcov
%% \fwd^*$, we may expect this implies that $\meas_j$ are also
%% eigenvectors of $(\fwd \prcov \fwd^*)^{-1}$ and assuming $(\fwd \prcov
%% \fwd^*)^{-1}$ has simple eigenvalues completes the proof. Our
%% expectations, however, are let down.

%% Here is a counterexample. Let $e_1, e_2$ denote two
%% eigenvectors of $C := (\fwd \prcov \fwd^*)^{-1}$ of different
%% eigenvalues such that $C + \obs^*\obs$ is scalar on span
%% $V:=\{e_1,e_2\}$ (we took $\sigma^2 = 1$ for simplicity). Then,
%% on $V$, we have  
%% \begin{equation*}
%%   (\fwd \prcov \fwd^*)^{-1} = diag( d_1, d_2),
%% \end{equation*}
%% and we assume $d_1 > d_2$ without loss of generality. For $\meas \in
%DIF < % \hil$, we denote $\meas^*$ its Riesz representer, making a departure
%DIF > % \hilp$, we denote $\meas^*$ its Riesz representer, making a departure
%% from the standard notation employed, where $\meas$ is usually taken to
%DIF < % be a linear functional on $\hil$. Let $\obs = (\meas_1^*, \meas_2^*,
%DIF > % be a linear functional on $\hilp$. Let $\obs = (\meas_1^*, \meas_2^*,
%% \meas_3^*)^t$. In this setup, we would like that $\meas_j,j=1,2,3$ are
%% either orthogonal or parallel. We now outline construction that shows
%% this is not necessarily true.

%% We know, from Corollary \ref{cor:sim diag} that $\meas_j \in V$, so
%% let
%% \begin{align*}
%%   \meas_1 &=  c_1e_1 + c_2e_2 \\
%%   \meas_2 &= -c_1e_1 + c_2e_2 \\
%%   \meas_3 &=     e_1.
%% \end{align*}
%% Then
%% \begin{align*}
%%   \meas_1 \meas_1^*
%%   &= c_1^2e_1e_1^* + c_1c_2( e_1e_2^* + e_2e_1^*) +
%%   c_2^2 e_2e_2^* \\
%%   \meas_2 \meas_2^* &= c_1^2e_1e_1^* - c_1c_2( e_1e_2^* +
%%   e_2e_1^*) + c_2^2 e_2e_2^*.
%% \end{align*}
%% Note that
%% \begin{equation*}
%%   \obs^* \obs = \meas_1 \meas_1^* + \meas_2\meas_2^* +
%%   \meas_3\meas_3^* = (2c_1^2+1) e_1e_1^* + 2c_2^2e_2e_2^*,
%% \end{equation*}
%% is diaonal in $\{e_1,e_2\}$. In order for $\meas_j,j=1,2,3$ to be
%% eigenvectors of $C + \obs^*\obs$ with the same eigenvalue, $C +
%% \obs^*\obs$ must be a scalar operator on $V$. Thus the following
%% must hold:
%% \begin{align*}
%%   d_1 + 2c_1^2 + 1 &= d_2 + 2c_2^2 \\
%%   c_1^2 + c_2^2 &= 1 \text{ (arises from $\| \meas_i \| = 1$).}
%% \end{align*}
%% Equivalently,
%% \begin{equation*}
%%   \begin{bmatrix}
%%     2 & -2 \\
%%     1 & 1
%%   \end{bmatrix}
%%   \begin{bmatrix}
%%     c_1^2 \\
%%     c_2^2
%%   \end{bmatrix}
%%   =
%%   \begin{bmatrix}
%%     d_2 - d_1 - 1 \\
%%     1
%%   \end{bmatrix}.
%% \end{equation*}
%% We can always find a solution for this system, for any $d_1,d_2$.
%% However, it is necessary that either $c_1^2 = 1$ or $c_2^2 = 1$ for
%% $\meas_j, j=1,2,3$ to all be either orthogonal or parallel. The first
%% conditiion implies $d_2 - d_1 = 3$, which is impossible, since $d_1 >
%% d_2$. The second requires $d_1 - d_2 = 1$, a condition that does not
%% hold generically.


%    Text of article.

%    Bibliographies can be prepared with BibTeX using amsplain,
%    amsalpha, or (for "historical" overviews) natbib style.
\bibliographystyle{amsplain}
\DIFdelbegin %DIFDELCMD < \bibliography{refs.bib,georg_refs}
%DIFDELCMD < %%%
\DIFdelend \DIFaddbegin \bibliography{refs,georg_refs}
\DIFaddend 

\end{document}





%%%%%%%%%%%%%%%%%%%%%%%%%%%%%%%%%%%%%%%%%%%%%%%%%%%%%%%%%%%%%%%%%%%%%%%%

%    Templates for common elements of an article; for additional
%    information, see the file Author_Handbook_ProcColl.pdf, included
%    in every AMS author package, and the amsthm user's guide, linked
%    from http://www.ams.org/tex/amslatex.html .

%    Section headings
\section{}
\subsection{}

%    Ordinary theorem and proof
\begin{theorem}[Optional addition to theorem head]
% text of theorem
\end{theorem}

\begin{proof}[Optional replacement proof heading]
% text of proof
\end{proof}

%    Figure insertion; default placement is top; if the figure occupies
%    more than 75% of a page, the [p] option should be specified.
\begin{figure}
\includegraphics{filename}
\caption{text of caption}
\label{}
\end{figure}

%    Mathematical displays; for additional information, see the amsmath
%    user's guide:  texdoc amamath  or
%  http://mirror.ctan.org/tex-archive/macros/latex/required/amsmath/amsldoc.pdf

% Numbered equation
\begin{equation}
\end{equation}

% Unnumbered equation
\begin{equation*}
\end{equation*}

% Aligned equations
\begin{align}
  &  \\
  &
\end{align}

%-----------------------------------------------------------------------
% End of article-template.tex
%-----------------------------------------------------------------------
